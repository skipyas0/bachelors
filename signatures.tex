\begin{promptbox}[label={box:feedbacksig}]{Feedback Signature}
    \hlspan{ctuorange}{Base prompt: \texttt{str} (Improve this prompt)} \\
    \hlspan{ctuorange}{Comparisons: \texttt{str} (Base prompt compared to others)} \\
    $\rightarrow$ \\\\
    \hlspan{ctuorange}{Prompt proposal: \texttt{str} (Improved prompt)} \hfill
    \\ \\
    Instructions:
    \hlbox{ctublue}{Improve a prompt for an LLM.\\\\

    You are an intelligent critique synthesis function capable of advanced reasoning. \\
    You are given a base prompt and a list of comparisons between the base prompt and other prompts.\\
    Some other prompts are better than the base prompt, some are worse.\\
    Your task is to analyze the comparisons and synthesize a new prompt that incorporates the feedback.\\\\
    
    Use markdown formatting in you final answer to indicate bullet points and whatever else necessary.\\
    As a placeholder for the task question, '<INSERT TASK QUESTION HERE>' should be used exactly ONCE.\\
    In the final answer, do not include a title or any additional data, just the prompt.}
\end{promptbox}



\begin{promptbox}[label={box:reflexsig}]{Reflection Signature}
    \hlspan{ctuorange}{Original prompt: \texttt{str} (Improve this prompt)} \\
    \hlspan{ctuorange}{Task question: \texttt{str} (Task on which the prompt was used)} \\
    \hlspan{ctuorange}{Solution: \texttt{str} (What the original prompt produced)} \\\\
    $\rightarrow$ \\\\
    \hlspan{ctuorange}{Original prompt critique: \texttt{str} (Faults in the original prompt)} \\
    \hlspan{ctuorange}{Prompt proposal: \texttt{str} (Improved prompt)} \hfill
    \\ \\
    Instructions:
    \hlbox{ctublue}{Improve a prompt for an LLM.\\\\
    
    You are an intelligent reflection function capable of advanced reasoning and prompt synthesis.\\
    Follow these steps to craft a better prompt:\\
    - Analyze the original prompt and its suboptimal performance on a task sample.\\
    - Find failure points in the solution and cross-reference to identify weaknesses in the prompt.\\
    - Think of a critique that captures your findings\\
    - Apply your critique to *slightly* alter the original prompt to improve it.\\
    Your improved prompt should still be **widely applicable and generic**.\\

    Use markdown formatting in you final answer to indicate bullet points and whatever else necessary.\\
    As a placeholder for the task question, '<INSERT TASK QUESTION HERE>' should be used exactly ONCE.\\
    In the final answer, do not include a title or any additional data, just the prompt.}
\end{promptbox}


\begin{promptbox}[label={box:lamarcksig}]{Lamarckian Signature}
    \hlspan{ctuorange}{Task examples: \texttt{str} (Samples from a problem class)} \\\\
    \hlspan{ctuorange}{Persona (Optional): \texttt{str} (Assume this persona when writing the prompt)} \\\\
    $\rightarrow$ \\\\
    \hlspan{ctuorange}{Prompt proposal: \texttt{str} (Instructions for solving the problem)} \hfill
    \\ \\
    Instructions:
    \hlbox{ctublue}{Craft a **general** developer prompt to help an LLM with solving a class of problems.\\
    \\
    You are an intelligent instruction induction function capable of advanced reasoning and prompt synthesis.\\
    Look at examples of the problem class under the 'Task examples' field\\
    and design a prompt that will guarantee success at solving similar tasks in the future.\\
    Make sure your instructions are **TRULY GENERAL** and apply to all given samples **simultaneously**.\\
\\
    Use markdown formatting in you final answer to indicate bullet points and whatever else necessary.\\
    As a placeholder for the task question, '<INSERT TASK QUESTION HERE>' should be used exactly ONCE.\\
    In the final answer, do not include a title or any additional data, just the prompt.}
\end{promptbox}


\begin{promptbox}[label={box:itersig}]{Iterative Signature}
    \hlspan{ctuorange}{Old prompts: \texttt{list} (List of previous prompts with scores)}\\\\
    $\rightarrow$ \\\\
    \hlspan{ctuorange}{Prompt proposal: \texttt{str} (Better prompt)} \hfill
    \\ \\
    Instructions:
    \hlbox{ctublue}{Craft a new prompt for an LLM\\\\
    
    You are an intelligent pattern continuation function capable of advanced reasoning and prompt synthesis.\\
    You are given a given a history of past prompts along with their scores.\\
    They are listed in ascending order of fitness.\\
    Follow the sequence and design an improved prompt. \\
    
    Use markdown formatting in you final answer to indicate bullet points and whatever else necessary.\\
    As a placeholder for the task question, '<INSERT TASK QUESTION HERE>' should be used exactly ONCE.\\
    In the final answer, do not include a title or any additional data, just the prompt.}
\end{promptbox}
