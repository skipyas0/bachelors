% arara: pdflatex: { synctex: yes }
% arara: makeindex: { style: ctuthesis }
% arara: bibtex

% The class takes all the key=value arguments that \ctusetup does,
% and a couple more: draft and oneside
\documentclass[twoside]{ctuthesis}

\ctusetup{
	preprint = \ctuverlog,
	mainlanguage = english,
	titlelanguage = english,
	otherlanguages = {czech},
	title-czech = {abcd},
	title-english = {Meta-prompts for LLM Prompt Optimization},
	subtitle-czech = {abcd},
	subtitle-english = {abcd},
	doctype = B,
	faculty = F3,
	department-english = {Artificial Intelligence Center},
	department-czech = {Centrum pro umělou inteligenci},
	author = {Vojtěch Klouda},
	supervisor = {Ing. Jan Drchal PhD.},
	supervisor-address = {Resslova 307/9 Praha, E-322},
%	supervisor-specialist = {John Doe},
	fieldofstudy-english = {Artificial Intelligence},
	subfieldofstudy-english = {Natural Language Processing},
	fieldofstudy-czech = {Umělá inteligence},
	subfieldofstudy-czech = {Zpracování přirozeného jazyka},
	keywords-czech = {jazykový model, optimalizace},
	keywords-english = {language model, optimization},
	day = 10,
	month = 5,
	year = 2025,
%	specification-file = {ctutest-zadani.pdf},
%	front-specification = true,
%	front-list-of-figures = false,
%	front-list-of-tables = false,
%	monochrome = true,
%	layout-short = true,
}

\ctuprocess
\usepackage[linesnumbered, ruled, vlined]{algorithm2e}

\addto\ctucaptionsczech{%
	\def\supervisorname{Vedoucí}%
	\def\subfieldofstudyname{Studijní program}%
}

\ctutemplateset{maketitle twocolumn default}{
	\begin{twocolumnfrontmatterpage}
		\ctutemplate{twocolumn.thanks}
		\ctutemplate{twocolumn.declaration}
		\ctutemplate{twocolumn.abstract.in.titlelanguage}
		\ctutemplate{twocolumn.abstract.in.secondlanguage}
		\ctutemplate{twocolumn.tableofcontents}
		\ctutemplate{twocolumn.listoffigures}
	\end{twocolumnfrontmatterpage}
}

% todo command
\newcommand{\todo}[1]{\textsuperscript{\textbf{\textcolor{red}{#1}}}}
\newcommand{\vsep}{\, \vert \,}

\newcommand{\maxmean}[2]{#1 {\color{gray}\scriptsize (#2)}}


% Theorem declarations, this is the reasonable default, anybody can do what they wish.
% If you prefer theorems in italics rather than slanted, use \theoremstyle{plainit}
\theoremstyle{plain}
\newtheorem{theorem}{Theorem}[chapter]
\newtheorem{corollary}[theorem]{Corollary}
\newtheorem{lemma}[theorem]{Lemma}
\newtheorem{proposition}[theorem]{Proposition}

\theoremstyle{definition}
\newtheorem{definition}[theorem]{Definition}
\newtheorem{example}[theorem]{Example}
\newtheorem{conjecture}[theorem]{Conjecture}

\theoremstyle{note}
\newtheorem*{remark*}{Remark}
\newtheorem{remark}[theorem]{Remark}

\setlength{\parskip}{5ex plus 0.2ex minus 0.2ex}

% Abstract in Czech
\begin{abstract-czech}
	TODO
\end{abstract-czech}

% Abstract in English
\begin{abstract-english}
	TODO
\end{abstract-english}

% Acknowledgements / Podekovani
\begin{thanks}
	= )))
\end{thanks}

% Declaration / Prohlaseni
\begin{declaration}
Prohlašuji, že jsem předloženou práci vypracoval samostatně, a že jsem uvedl veškerou použitou literaturu.

V Praze, \ctufield{day}.~\monthinlanguage{title}~\ctufield{year}
\end{declaration}

% Only for testing purposes
\listfiles
\usepackage[pagewise]{lineno}
\usepackage{lipsum,blindtext}
\usepackage{mathrsfs} % provides \mathscr used in the ridiculous examples
\usepackage{pdflscape}
\usepackage{tabularx}
\usepackage{booktabs}
\usepackage{array} % put in preamble if not already there
\usepackage[table]{xcolor}
\usepackage[most]{tcolorbox}
\usepackage{xparse}
\usepackage{expl3}

\ExplSyntaxOn

\NewDocumentCommand{\hlboxx}{m m m}
 {
  \hlboxx_inner:nnn { #1 } { #2 } { #3 }
 }

\cs_new_protected:Nn \hlboxx_inner:nnn
 {
  % store and transform the title
  \tl_set:Nn \l_tmpa_tl { #2 }
  \tl_lower_case:N \l_tmpa_tl
  \tl_remove_all:Nn \l_tmpa_tl { ~ }

  % generate label string manually
  \tl_set:Nx \l_tmpb_tl { box:\l_tmpa_tl }

  % now insert into tcolorbox
  \begin{tcolorbox}[
    colback=#1!10!white,
    colframe=#1!80!black,
    boxrule=0.8pt,
    arc=4pt,
    left=6pt,
    right=6pt,
    top=4pt,
    bottom=4pt,
    enhanced,
    breakable,
    label={\tl_use:N \l_tmpb_tl},
    title={#2}
  ]
  #3
  \end{tcolorbox}
 }

\ExplSyntaxOff

\newcommand{\hlbox}[2]{%
  \begin{tcolorbox}[colback=#1!10!white,
                    colframe=#1!80!black,
                    boxrule=0.8pt,
                    arc=4pt,
                    left=6pt,
                    right=6pt,
                    top=4pt,
                    bottom=4pt,
                    enhanced,
                    breakable]
  #2
  \end{tcolorbox}
}
\newcommand{\hlspan}[2]{%
  \tcbox[colback=#1!10!white,
         colframe=#1!80!black,
         on line,
         boxrule=0.6pt,
         arc=3pt,
         boxsep=1pt,
         left=2pt,
         right=2pt,
         enhanced]{#2}
}
\newtcolorbox[auto counter, number within=section]{promptbox}[2][]{%
  colback=gray!5!white, colframe=ctublue,
  title={\hlspan{ctublue}{\thetcbcounter} #2},
  fonttitle=\bfseries,
  enhanced,
  breakable,
  #1
}
\begin{document}

\maketitle

\chapter{Introduction}
\section{Background}
The central focus of this work is an instance of an LLM, denoted $\mathcal{M}$. 
When appropriate, we can differentiate between instances with a lower index, specifying its purpose. 
For example, when using separate LLM instances for optimizing and task-solving, we will denote them $\mathcal{M}_{optim}$ and $\mathcal{M}_{solve}$ respectively.
This way, we put emphasis on the fact we can choose a different LLM provider and hyperparameters for each instance.


In general, $\mathcal{M}: \mathbb{T} \times \mathbb{T}$ is a stochastic (for a positive sampling temperature) mapping on the space of text sequences $\mathbb{T}$.
A prompt $p \in \mathbb{T}$ is a text sequence that, when inputted into an LLM, produces an output
\begin{equation}
    y \sim \mathcal{M}(p).
\end{equation} 

We can use LLMs to solve a general task
\begin{equation}
    t \in \mathcal{D} \vsep \mathcal{D} = \{ (q_1, g_1), (q_2, g_2), ... , (q_n, g_n), \},
\end{equation}
where $\mathbb{D}$ is a dataset consisting of $n$ pairs of queries $q$ and gold-labels $g$. 
For open-ended tasks, the gold-label does not exist.


We can further define a prompt as
\begin{equation}
    p = \mathbf{i}(q),
\end{equation}
where $\mathbf{i} \in \mathbf{T}$ is a set of text instructions into which a task query $q$ is inserted. 

\begin{algorithm}
    \caption{General optimization loop}
    \label{alg:genoptimloop}
    \KwIn{Initialization Operator $\mathcal{O}_I$, Selection Operator $\mathcal{O}_S$, Expansion Operator $\mathcal{O}_E$, Termination Condition $\Phi_{stop}$}
    \KwOut{Optimized Population $\mathcal{P}$}
    \KwData{$\mathcal{P} \gets \mathcal{O}_I$} \tcp{Initialize the population}
    \While{$\neg \Phi_{stop}(\mathcal{P})$}{
        \tcp{Selection and Expansion Steps}
        $\mathcal{P}_{\text{selected}} \gets \mathcal{O}_S(\mathcal{P})$ \\ 
        $\mathcal{P}_{\text{expanded}} \gets \mathcal{O}_E(\mathcal{P}_{\text{selected}})$ \\
        $\mathcal{P} \gets \mathcal{P}_{\text{expanded}}$ \tcp{Update the population} 
        }
        \Return{$\mathcal{P}$} \tcp{Return the optimized population}
    \end{algorithm}
    
Next, we move onto the optimization notation. In Algorithm \ref{alg:genoptimloop} we can see the general outline of a population-based optimization method.
The initialization operator $\mathcal{O}_I$ creates an initial population of individuals $\mathcal{P}$. 
Then, in each step, a selection operator $\mathcal{O}_S$ selects a portion of the population according to some criteria. 
These selected individuals are then used by the expansion operator $\mathcal{O}_E$ to create new individuals.
This process continues until a termination condition $\Phi_{stop}$ is reached.

\chapter{Literature}


\section{Inference-time scaling}\label{sec:inference}
Inference-time scaling or test-time scaling is a paradigm that has gained traction in recent years
with the advent of dedicated reasoning models\cite{openai2024openaio1card}\cite{deepseekai2025deepseekr1incentivizingreasoningcapability}. 
As opposed to training-time scaling, where the performance of models scales with 
training times, model parameter counts and dataset sizes\cite{kaplan2020scalinglawsneurallanguage},
inference-time scaling aims to improve performance by dedicating more resources to each inference call.

At their heart LLMs are probabilistic models over sequences and to generate a sequence they employ generation algorithms. 
Welleck et al.\cite{welleck2024decodingmetagenerationinferencetimealgorithms} provide an overview of these generation algorithms
and then frame more advanced inference-time techniques as meta-generations, or strategies that employ sub-generators.
Most generation algorithms attempt to find either highly probable sequences (MAP algorithms) or sample from the model's distribution.
The simplest MAP algorithm is greedy decoding, which recursively finds the next token with the highest probability in the distribution.
An example of algorithms that sample from the model's distribution is the ancestral sampling algorithm.

A generalization of greedy decoding is the beam search algorithm which maintains a structure of possible prefixes and each step expands them and scores them.
An example\cite{wang2024chainofthoughtreasoningprompting} of a beam search algorithm can identify decoding branches where the model 
employs a reasoning chain to solve a given task. Authors of this algorithm found that answer tokens found in the decoding paths with a reasoning chain 
have greater token probabilities. This means that the model shows greater confidence in its answer having reasoned about it beforehand.
In general beam search improves on simple greedy decoding but at a high computational cost\cite{welleck2024decodingmetagenerationinferencetimealgorithms}.

Another class of generation algorithms are those which interpolate between more categories of sampling algorithms.
Temperature sampling, which outperforms other adapters in input-output tasks like code generation and translation,
is an interpolation between greedy sampling and uniform sampling. 
Interpolating between ancestral sampling and simple greedy sampling gave rise to decoding algorithms such as
nucleus, top-k and $\eta$- and $\epsilon$-sampling. When we require a structured output, for example a JSON data 
structure following a JSON schema, we can utilize parser-based decoding, which enforces a structural requirement.
This can however come at worsened performance when using inflexible templates.
\newpage
These low-level generator can be interconnected into more complex technique, which Welleck et al. call meta-generators\cite{welleck2024decodingmetagenerationinferencetimealgorithms}.
We will stick to their terminology and discuss different sequence-level meta-generation algorithms. We will omit further discussion of token-level methods as
they are irrelevant to the main topic of this thesis. These strategies can be divided into the categories of chained, parallel, step-level, and refinement-based meta-generators.

\subsection{Chained meta-generation}

Chained meta-generation is the composition of several subgenerators in sequence. 
These can be LLM calls or other functions that use previous inputs, such as a code execution function\cite{chen2023programthoughtspromptingdisentangling}
or a tool for interaction with and arbitrary environment or a data source\cite{yao2023reactsynergizingreasoningacting}.
The subgenerators can be implemented as several LLM calls or with a single call given sufficient instructions in the prompt. \cite{khattab2023dspycompilingdeclarativelanguage}
Some examples include Program of Thoughts\cite{chen2023programthoughtspromptingdisentangling}, ReAct\cite{yao2023reactsynergizingreasoningacting} 
and Chain-of-Thought\cite{NEURIPS2022_8bb0d291}\cite{wei2023chainofthoughtpromptingelicitsreasoning} techniques.

In its essence, the model is a left-to-right text completion engine. We can make the analogy with human thinking 
modes, where it is said that humans have a fast automatic "System 1" mode and a slow and deliberate "System 2" mode\cite{yao2023treethoughtsdeliberateproblem}. 
In direct-QA mode, the LLM can underestimate the difficulty of the task\cite{wang2024chainofthoughtreasoningprompting} and stay in the "System 1" thinking mode.
Simple greedy decoding paths mostly do not contain a reasoning chain\cite{wang2024chainofthoughtreasoningprompting}, which means the model tends to make a guess, staying in "System 1".
By crafting a good prompt that instructs the model to reason we can shift the model from "System 1" to "System 2" thinking.
Another reason for the effectiveness of chained generation is that in LLM training
some concepts and variables are observed more frequently than others\cite{prystawski2023thinkstepstepreasoning}. 
This discrepancy hurts performance in direct-QA scenarios where the relevant
variables are rarely seen together in training. With CoT, models can incrementally chain known dependencies and bridge conceptional gaps.

\subsubsection{Chain-of-thought}\label{sec:cot}
Chain-of-Thought\cite{wei2023chainofthoughtpromptingelicitsreasoning} (CoT) is a LLM prompting technique that works by inducing a coherent series of intermediate 
reasoning steps that lead to the final answer for a problem, thereby increasing computation time. 
Upon its discovery, it brought a dramatic performance increase on arithmetic tasks, where models previously struggled.
This enhanced capability comes with the cost of longer and more computationally expensive outputs\cite{brown2024largelanguagemonkeysscaling} and
is more noticeable for more complicated problems\cite{wei2023chainofthoughtpromptingelicitsreasoning}. 

CoT can been elicited by prompting techniques - few-shot with steps demonstrations or 
zero-shot with specific instructions\cite{wang2024chainofthoughtreasoningprompting}
First CoT methods\cite{wei2023chainofthoughtpromptingelicitsreasoning} involved one/few-shot prompting, 
Although effective, this requires human engineering of multi-step reasoning prompts.
This method is also highly sensitive to prompt design with performance deteriorating 
for mismatched prompt example and task question types\cite{NEURIPS2022_8bb0d291}.
For this method, authors found that CoT is an emergent capability of model scale 
and did not observe benefits for small models\cite{wei2023chainofthoughtpromptingelicitsreasoning}.
where the prompt included examples of CoT reasoning in the prompt in facilitate a reasoning chain response.

On the other hand, zero-shot prompting induces a reasoning chain with a simple prompt like "Let's think step-by-step",
making it versatile and task-agnostic\cite{NEURIPS2022_8bb0d291}. Similar prompts also improve reasoning performance and 
some research\cite{yang2024largelanguagemodelsoptimizers} has been done on finding the optimal CoT prefix prompt.

Apart from prompting, CoT can be elicited and improved by model training or tuning. 
This method, requiring a significant amount of reasoning data\cite{wang2024chainofthoughtreasoningprompting},
has gained traction with the development of dedicated reasoning models like OpenAI's o1\cite{openai2024openaio1card} or Deepseek-R1\cite{deepseekai2025deepseekr1incentivizingreasoningcapability}.
Using methods such as supervised fine-tuning (SFT) or reinforcement learning (RL), the model is trained to
automatically produce longer reasoning chains, often bound in dedicated "thought" tags or tokens. 
These models have shown significant performance boosts on reasoning benchmarks\cite{openai2024openaio1card}\cite{deepseekai2025deepseekr1incentivizingreasoningcapability}.

Models similar to o1 all primarily extend solution length by self-revision\cite{zeng2025revisitingtesttimescalingo1like}.
After finishing a thought process, the model tries to self-revise, which is marked by words such as "Wait" or "Alternatively". 
The model then tries to spot mistakes or inconsistencies in its reasoning or propose an alternative solution. 
Self-revision ability is thus a key factor in the effectiveness of sequential scaling for reasoning models\cite{zeng2025revisitingtesttimescalingo1like}.

Longer reasoning chains mean more computing power spent at inference. How far can we take this sequential scaling?
In their study, Zeng et al.\cite{zeng2025revisitingtesttimescalingo1like} argue that longer CoTs do not consistently improve accuracy of reasoning models.
Furthermore, they find that the average length of correct solutions is shorter than that of incorrect ones. 
\newpage
Because self-revision accounts for most of the CoT length, the effectiveness of the method relies on the model's ability to self-revise.
Authors of this paper argue that the self-revision ability of models is insufficient as they demonstrate limited capacity to correct their answers
during self-revision. Some models on some tasks are even more likely to change a correct answer to an incorrect one than vice-versa.

Further research by Liu et al.\cite{liu2024mindstepbystep} suggests that for some tasks CoT can be detrimental.
Their experiments proved their hypothesis that CoT hurts performance on tasks where humans do better without deliberation
and where the nature of LLM, like the much greater context memory, does not provide an advantage over human thinking. 
This phenomenon was observed on tasks like facial recognition, implicit statistical learning or pattern recognition.

\subsection{Parallel meta-generation}

Parallel meta-generation involves multiple generations concurrently. 
The final answer can then be chosen - with a reward model or with 
voting - or constructed from the ensemble of generations\cite{welleck2024decodingmetagenerationinferencetimealgorithms}.

Parallel meta-generation allows weaker models to outperform bigger and more expensive models\cite{brown2024largelanguagemonkeysscaling}.
This can sometimes reduce cost as multiple samples with a smaller model are cheaper than a single sample with a more capable model.
This is helped by the fact that parallel sampling can make use of batching and other system throughput optimization
available for parallel inference\cite{brown2024largelanguagemonkeysscaling}.

One of the simplest such techniques is self-consistency\cite{wang2023selfconsistencyimproveschainthought} (SC),
a method which builds upon CoT to aggregate answers from diverse reasoning 
chains and selects the best one based on majority voting. 
It significantly improves accuracy in a range of arithmetic and commonsense reasoning tasks 
at the cost of increased computation expenditure\cite{wang2023selfconsistencyimproveschainthought}.
The effectiveness of SC with majority-voting comes from the fact that, for tasks with objective answers, there are often more ways to be wrong than to be right.
\newpage
For our next discussion of SC and related methods we will compare the 
terms \textit{coverage} $\mathrm{C}_{\mathbb{D}}$ and \textit{accuracy} $\mathrm{A}_{\mathbb{D}}$ for a dataset ${\mathbb{D}}$.
Given a language model $\mathcal{M}$, a task query $q_k \in {\mathbb{D}}$ and a task 
instruction $\mathbf{i}$, we can define the generation collection of length $n$ as
\begin{equation}
    Y_k = \{y_{jk}\mid j \in 1, ..., n\},
\end{equation}
\begin{equation}
    y_{jk} \sim \mathcal{M}(\mathbf{i}(q_k)).
\end{equation}
For objective tasks we can check the correctness with a metric $\mathcal{G}$
\begin{equation}
    \mathcal{G}_{k}(y_{jk}, q_k) = 
    \begin{cases}
        1.0 & y_{jk} \text{ is the correct answer for } q_k\\
        0.0 & y_{jk} \text{ is an incorrect answer for } q_k.
    \end{cases}
\end{equation}
To choose the final answer, we will define an answer selection function $\mathcal{S}(Y)$. 
This can be a majority vote selection function or some reward-based method.
We can now define \textit{coverage} $\mathrm{C}_{\mathbb{D}}$ and \textit{accuracy} $\mathrm{A}_{\mathbb{D}}$ as
\begin{align}
    \mathrm{C}_{\mathbb{D}} &= \frac{1}{|\mathbb{D}|} \sum_{q_k \in \mathbb{D}} \max_{j=1,...,n} \mathcal{G}_k(y_{jk}, q_k) \\
    \mathrm{A}_{\mathbb{D}} &= \frac{1}{|\mathbb{D}|} \sum_{q_k \in \mathbb{D}} \mathcal{G}_k\left( \mathcal{S}(Y_k), q_k \right).
\end{align}
In plain language, coverage is the fraction of the tasks where at least one sample results in a correct answer,
whereas accuracy is the fraction of the tasks where a correct answer is selected by the algorithm as a final answer.

It is easy to see why coverage might rise as we increase the amount of samples $n$ in SC generation.
One could imagine that as letting students answer with their top $n$ guesses for each question on a test. 
Indeed research\cite{brown2024largelanguagemonkeysscaling} has found that the relationship of coverage and the 
number of samples can be modeled by an exponentiated power law, suggesting a scaling law for inference
similar to the training scaling laws\cite{kaplan2020scalinglawsneurallanguage}.

However coverage alone is not enough to paint the complete picture. What good is it to have a large collection which contains a correct answer
if we cannot verify which one is correct. Parallel scaling with large sample collections is only useful 
if the correct samples in a collection can be identified\cite{brown2024largelanguagemonkeysscaling}\cite{zeng2025revisitingtesttimescalingo1like}.
The accuracy gain of SC tends to saturate quickly as we increase the number of paths\cite{wang2023selfconsistencyimproveschainthought}.
Although coverage rises, it diverges\cite{brown2024largelanguagemonkeysscaling} from accuracy as the algorithm is unable to select the correct answer from the collection.
This highlight the necessity to develop better answer selection mechanisms than simple majority voting and automatic answer verification methods.
\newpage
Zeng et al.\cite{zeng2025revisitingtesttimescalingo1like} make use of the fact that
correct solutions have shorter CoT on average and develop a length-weighted majority vote that outperforms simple majority voting on 
the challenging math benchmarks. GLaPE\cite{zhang2024glapegoldlabelagnosticprompt} is a method for gold label-agnostic evaluation which makes use
of the fact that incorrect answers tend to be inconsistent. 

\subsection{Step-level meta-generation}
Maintaining the terminology of Welleck et al.\cite{welleck2024decodingmetagenerationinferencetimealgorithms}, step-level meta-generation
algorithms implement search on the generation state-space. This can be done on the token level or on the level of longer sequences,
but in this section we will focus on the latter.

Previously discussed inference-time scaling techniques all relied on sequential or parallel linear thought processes. 
They do not explore different continuations within a thought process and do not make use of 
planning, lookahead, or backtracking\cite{yao2023treethoughtsdeliberateproblem}. These methods also do not allow 
combining the flow of reasoning upon discovering new insights, something humans utilize when solving problems\cite{Besta_2024}.

By generalizing CoT\cite{wei2023chainofthoughtpromptingelicitsreasoning}\cite{NEURIPS2022_8bb0d291} into a tree structure,
Yao et al.\cite{yao2023treethoughtsdeliberateproblem} present Tree of Thoughts (ToT), a technique which maintains a tree of thought.
In this tree, each node is a thought in a form of a coherent language sequence, serving as an intermediate step
in the reasoning process. For traversing the tree, a general tree-search algorithm, such as breadth-first or depth-first search, can be employed.

An important parameter in ToT is the branching factor. Unlike the standard tasks typically tackled by tree search algorithms 
where the number of possible actions at each node is finite, each call to LLM can yield a new output even for 
the same input, making each node's branching factor theoretically infinite\cite{misaki2025widerdeeperscalingllm}.
Misaki et al.\cite{misaki2025widerdeeperscalingllm} argue that fixed-width multi-turn methods exhibit diminishing gains 
and develop a tree search method with an adaptive branching factor, leading to a more balanced exploration and exploitation capability. 
\newpage
Although ToT allows for planning and backtracking from unpromising thought chains, its structure is still too rigid\cite{Besta_2024}.
For example, it is not possible to combine thoughts from independent branches from the tree. Graph of Thoughts\cite{Besta_2024} (GoT) is 
a framework that models the reasoning process as a heterogenous directed graph where each vertex is a thought containing a (partial) solution
and edges are dependencies between these thoughts\cite{Besta_2024}. 

In both chain- and tree-based inference-time scaling methods, a substantial amount of compute power is allocated to
processing historical information that is not beneficial to the reasoning process. To alleviate this,
Atom of Thoughts\cite{teng2025atomthoughtsmarkovllm} (AoT) iteratively decomposes the current question into a directed acyclic graph.
The graph consists of subquestions which, depending on whether they have dependencies, are dependent or independent.
All the independent questions can be answered directly and their answers added combined as context with the remaining subquestions
to be contracted into a new current question. 

\subsection{Refinement meta-generation} \label{sec:refine}
The last category of meta-generation algorithms are refinement algorithms. Refinement algorithms work by alternating between generation and refinement.
The refiner generates a revised version of the output based on past versions and additional information such as intrinsic or extrinsic feedback or environment observations\cite{welleck2024decodingmetagenerationinferencetimealgorithms}. 
Intrinsic refinement comes from the model inspecting its own answers. As we discussed in \ref{sec:cot}, models struggle with self-revision and rarely modify their answers in long reasoning chains.
Feedback from general models is ineffective compared to dedicated feedback models or other quality feedback sources\cite{wang2025dedicatedfeedbackeditmodels}. 

For extrinsic refinement, the model can utilize external information which can lead to a potential gain with refinement\cite{welleck2024decodingmetagenerationinferencetimealgorithms}.
One example of a refinement-based framework, Reflexion\cite{shinn2023reflexionlanguageagentsverbal}, converts binary or scalar feedback from the environment into verbal feedback in the form 
of a textual summary. This feedback is then added as additional context for the LLM agent, e.g. CoT or ReAct module, in the next episode. 
Reflexion improves performance over strong baselines on sequential decision making, reasoning and programming tasks\cite{shinn2023reflexionlanguageagentsverbal}.




\section{Prompt engineering}
Maintaining the notation outlined in \ref{sec:notation}, prompt $p = \mathbf{i}(q)$ is a combination 
of a set of instruction $\mathbf{i}$ and a query $q$. 

By prompt engineering we mean crafting a instruction set which 
transforms the query into a result according to our task requirements.
Our task requirements can for example be
\begin{itemize}
    \item obtaining the correct answer for a mathematical problem
    \item fixing a bug in a code base
    \item explaining the contents of an image.
\end{itemize}
Each of these tasks needs a separate instruction set $\mathbf{i}$ which can then be used with multiple queries,
representing specific task instances. This signifies a shift from the training and fine-tuning paradigm, where 
a base model is first trained on a large corpus of data and then adapted for a specific task with supervised fine-tuning.
This process requires a substantial amount of training data and computation power, making specialized LLMs unsuitable
for many users and for applications, where extensive data collection is infeasible.


Since the inception of modern LLMs, prompt engineering has evolved into a field of its own. Current LLM systems, often containing
multiple chained and interlinked models, require robust and well thought-out prompts at each step. 
Indeed, in many modern LLM applications, prompts have become programs themselves\cite{schnabel2024symbolicpromptprogramsearch}, 
marking a huge leap from the basic text messages of the early LLM days.

In this section, we will briefly cover the most notable prompt engineering techniques, which we will then
be able to utilize in our study of automatic prompt optimization.

\subsection{Components of a prompt}
We can dissect a prompt into several components\cite{schulhoff2024promptreportsystematicsurvey}.
\begin{itemize}
    \item \textbf{Directive:} The main task of the prompt, e.g., \textit{"Write an email to a coworker."}
    \item \textbf{Context:} Everything necessary or beneficial to completing the directive, e.g., \textit{"I was supposed to send a report to my boss, but I forgot."}
    \item \textbf{Examples:} How you would have solved a similar task, e.g. a past email on a similar topic.
    \item \textbf{Output specifications:} Style and format instructions, e.g., \textit{"Respond with three paragraphs in formal style with tasteful emojis."}
\end{itemize}

Although flexible and sometimes blended together, high-performing prompts often follow this structure and order. 
In more technical applications, it is beneficial to use tags or delimiters to explicitly separate components.
Models often reward prompts with a more code-like structure\cite{10.1145/3544548.3581388}. 
We now discuss each component in detail.

\subsubsection{Directive}
The directive should be a clear and objective description of the task. 
Specific requirements that narrow the scope should be avoided and left for other components.
A good rule of thumb is to treat the directive as a subject to an email. 

In cases where the prompt serves as a prompt template, meaning it can be reused with various data points,
the directive can include a placeholder. For example consider this directive
\begin{verbatim}
    Write a limerick about {topic}.
\end{verbatim}
This directive, and the prompt to which it belongs, could be reused for multiple values of \textit{topic}.

\subsubsection{Context}
Context should provide all the background information relevant to the task at hand.
The user can include more information about why they are using the model for the task,
define the target audience or attach relevant documents. For example, a prompt that asks the LLM to explain a code snippet:

In this case, the first part is the directive and the second part is the context, which specifies the user's situation.
Without the context, the model might explain each line of the snippet in too much detail and not explain the dictionary unwrapping operator specifically.

The context can become the endpoint for retrieval pipelines, which search a data source for 
relevant documents, or memory mechanisms, which gather personal information about the user from other conversations. 
\begin{verbatim}
Explain this Python code snippet.
```python
    user_info = {'name': 'Alice', 'age': 30, 'city': 'New York'}
    def greet_user(**kwargs):
        print(f"Hello, {kwargs['name']} from {kwargs['city']}!")
    greet_user(**user_info)
```

I am a beginner to Python programming. 
I understand the function definition and that user_info is a dictionary, 
but I don't know what the double asterisk does in the function call. 
Can you explain how the double asterisk 
works in this context and what it does step by step?
\end{verbatim} 

Another possible feature of the context component is role-assignment. 
We can instruct the model to adopt an identity or an expertise level.
For example, we can tell the model something like "You are a experienced business analyst".
This primes the model to use a more technical language in its response.

\subsubsection{Examples}
Sufficiently large models trained on massive datasets a
Prompts are distinguished based on the number of included examples.

\begin{table}[ht!]
    \centering
    \begin{tabular}{|c|p{8cm}|}
    \hline
    \textbf{Prompting Type} & \textbf{Description} \\
    \hline
    Zero-shot Prompting & Prompt has no examples. Model relies on its pre-trained knowledge. \\
    \hline
    One-shot Prompting & Prompt has one example to guide the model. \\
    \hline
    Few-shot Prompting & Prompt includes a few examples (typically 2 to 5). \\
    \hline
    \end{tabular}
    \caption{Comparison of Zero-shot, One-shot, and Few-shot Prompting}
\end{table}        
Research\cite{brown2020languagemodelsfewshotlearners} has shown that with growing model size the 
knowledge-generalizing ability of the model increases. Instead of expensive fine-tuning
models can reuse knowledge from pre-training and solve many tasks when provided just by a few examples.

Few-shot prompting highlights that LLMs can be seen as powerful pattern-completion engines. \cite{meyerson2024languagemodelcrossovervariation}

Providing a prompt of examples from a distribution can condition the LLM to generate further 
high-probability examples from that distribution \cite{meyerson2024languagemodelcrossovervariation}

\subsubsection{Output specifications}


\section{Prompt optimization}
\subsection{Soft prompt tuning}
Prompts for models which allow access to gradients, which is not the case for proprietary models accessed via APIs, can be optimized in the high-dimensional embedding space.

This makes the optimization problem continous. Soft prompts however pose the problem of interpretability and are non-transferable across different LLMs \cite{deng2022rlpromptoptimizingdiscretetext}.

Continuous prompt-optimization techniques, although effective, require parameters of LLMs inaccessible to black-box APIs and often fall short of interpretability. \cite{guo2024connectinglargelanguagemodels}

\subsection{Discrete prompt tuning}
The area of optimizing prompts discretely while utilizing language models as optimization operators has attracted significant research interest in recent years.

Natural language prompt engineering is particularly interesting because it is a natural interface for humans to communicate with machines, but plain language prompts do not always produce the desired result. \cite{zhou2023largelanguagemodelshumanlevel}

Natural language program synthesis search space is infinitely large. \cite{zhou2023largelanguagemodelshumanlevel}


Meta-prompts are flexible but studies lack principled guidelines about their design. \cite{tang2024unleashingpotentiallargelanguage}

Reproduces key model parameter learning factors - update direction and update method - in LLMs to seek theoretical foundations. \cite{tang2024unleashingpotentiallargelanguage}

OPRO\cite{yang2024largelanguagemodelsoptimizers} and APO\cite{pryzant2023automaticpromptoptimizationgradient} introduced analogical "gradient" forms. \cite{tang2024unleashingpotentiallargelanguage}

Analogical momentum forms inspired by the momentum method involve including the optimization trajectory in the meta-prompt. To fit into the context limit and reduce noise, trajectory can be summarized or k most recent/relevant/important gradients can be retrieved. \cite{tang2024unleashingpotentiallargelanguage}

To mimic effects or learning rate, prompt variation can be limited by edit distance (maximum words to be changed). Warm-up and decay strategies can be applied to this constraint. \cite{tang2024unleashingpotentiallargelanguage}

New prompt can be created by editing a previous prompt or generate a new one by following a demonstration. \cite{tang2024unleashingpotentiallargelanguage}

In an experiment on BBH, authors found that optimization without reflection performs better and the best momentum method being relevance. For prompt variation control, the best combination was cosine decay and no warm-up.  \cite{tang2024unleashingpotentiallargelanguage}

Summarization-based trajectory is less helpful because it tends to only capture common elements. \cite{tang2024unleashingpotentiallargelanguage}

Task input-output examples are beneficial in the meta-prompt to provide additional context to the LLM to understand the task. \cite{tang2024unleashingpotentiallargelanguage}

GPT-4 can consistently find better task prompts than GPT-3.5-turbo, which suggests the need for a capable model as the prompt optimizer \cite{tang2024unleashingpotentiallargelanguage}

Trajectory-based methods perform very well possible because trajectory helps the prompt optimizer pay more attention to the important information instead of the noise in the current step. \cite{tang2024unleashingpotentiallargelanguage}

\textbf{APE}
LLMs are used to construct a good set of candidate solutions by inferring the most likely instructions from input/output demonstrations. \cite{zhou2023largelanguagemodelshumanlevel}

Local search around the best candidates by resampling - asking the LLM to paraphrase the candidate prompt - this however only provides marginal improvements over just choosing the best-performing prompt from instruction induction. \cite{zhou2023largelanguagemodelshumanlevel}

APE was used to improve on Zero-Shot-CoT \cite{NEURIPS2022_8bb0d291} universal "Let's think by step" prompt"on GSM8k.\cite{zhou2023largelanguagemodelshumanlevel}


Prompt to the LLM optimizer is called the meta-prompt and includes previous prompts with their training accuracies sorted in ascending order along with the task description and training set samples. \cite{yang2024largelanguagemodelsoptimizers}

The main advantage of LLMs for optimization is their ability of understanding natural language, which allows people to describe their optimization tasks without formal specifications. \cite{yang2024largelanguagemodelsoptimizers}

Motivated by linear regression and TSP and on small-scale traveling salesman problems, OPRO performs on par with some hand-crafted heuristic algorithms. \cite{yang2024largelanguagemodelsoptimizers}

Optimization stability can be improved by generating multiple solutions when relying on random ICL samples. \cite{yang2024largelanguagemodelsoptimizers}

To balance between exploration and exploitation, LLM sampling temperature can be tuned. Lower temperature encourages exploitation in the local solution space and higher temperature allows more aggressive exploration of different solutions. \cite{yang2024largelanguagemodelsoptimizers}

Only the top instructions are kept in the meta-prompt to fit in the LLM context limit. \cite{yang2024largelanguagemodelsoptimizers}

New outstanding solution is usually found only all the prompts are of similar quality: first all the worse prompts are purged and substituted by a prompt similar to the current best. \cite{yang2024largelanguagemodelsoptimizers}

Semantically similar instructions have vastly different performance on GSM8k: “Let’s think step by step.” achieves accuracy 71.8, “Let’s solve the problem together.” has accuracy 60.5, while the accuracy of “Let’s work together to solve this problem step by step.” is only 49.4. \cite{yang2024largelanguagemodelsoptimizers}

\subsubsection{Textual gradients}
Naturally there are no gradients in the text space but some researchers try to emulate them using reflection-based operators.

APO mirrors the steps of gradient descent within a text-based Socratic dialogue substituting differentiation with LLM feedback and backpropagation with LLM editing \cite{pryzant2023automaticpromptoptimizationgradient}

Beam search is an iterative optimization process where in current prompt is expanded into many more candidates in each iteration and a selection process decides which will be used in the next iteration. \cite{pryzant2023automaticpromptoptimizationgradient}

Expansion first uses gradients to edit the current prompt and then explores the local monte-carlo search space by paraphrasing the editions \cite{pryzant2023automaticpromptoptimizationgradient}

To limit the computation used on evaluating prompts, an approach inspired by best arm identification in bandit optimization is utilized. \cite{pryzant2023automaticpromptoptimizationgradient}


Applying previous iterative prompt optimization methods, based on prompt+score pairs, to text generation tasks is challenging due to the lack of effective optimization signals. \cite{he2024crispomultiaspectcritiquesuggestionguidedautomatic}

Critiques and suggestions, written in natural language, are more helpful for prompt improvement than a single score.\cite{he2024crispomultiaspectcritiquesuggestionguidedautomatic}

CriSPO uses prompt+score+critique triples for next candidate generation. \cite{he2024crispomultiaspectcritiquesuggestionguidedautomatic}
Unlike APE \cite{pryzant2023automaticpromptoptimizationgradient} prompt generation is decoupled from suggestions and a history of critiques and suggestions as packed into the optimizer for a more stable optimization. \cite{he2024crispomultiaspectcritiquesuggestionguidedautomatic}

CoT is applied in optimization by first asking to compare high-score prompts to low-score ones and draft general ideas. \cite{he2024crispomultiaspectcritiquesuggestionguidedautomatic}

Critique-based optimization explores a larger space, which is indicated by lower similarity of the prompts in lexicons and semantics.\cite{he2024crispomultiaspectcritiquesuggestionguidedautomatic}

CriSPO outperforms OPRO \cite{yang2024largelanguagemodelsoptimizers} both on summarization and QA tasks and metaprompt allows for creating ICL and RAG template prompts. \cite{he2024crispomultiaspectcritiquesuggestionguidedautomatic}

\textbf{DSPy optimizers}

Most prompt optimizer approaches do not apply to multi-stage LLM programs where we lack gold labels or evaluation metrics for individual LLM calls. \cite{opsahlong2024optimizinginstructionsdemonstrationsmultistage}

Proposing a few high-quality instructions is essential due to the intractably large search space. \cite{opsahlong2024optimizinginstructionsdemonstrationsmultistage}

Uses a surrogate Bayesian optimization model, which is updated periodically by evaluating the program on batches, to sample instructions and demonstrations for each stage of the LLM program \cite{opsahlong2024optimizinginstructionsdemonstrationsmultistage}

Optimizing demonstrations alone usually yields better performance than just optimizing instructions, but optimizing both yield the best performance. \cite{opsahlong2024optimizinginstructionsdemonstrationsmultistage}

Optimizing instructions is most valuable for tasks with subtle conditional rules not expressible by a few examples.  \cite{opsahlong2024optimizinginstructionsdemonstrationsmultistage}

For LLM programs, it is beneficial to alternate between optimizing weights (fine-tuning) and optimizing prompts. \cite{soylu2024finetuningpromptoptimizationgreat}

\subsubsection{Evolutionary optimization}
Building upon the inherent ability of LLMs to paraphrase (mutation) and combine (crossover) text, an interesting intersection of traditional evolutionary algorithms and modern LLMs has formed. 


Sequences of phrases can be regarded as gene sequences in typical Evolutionary algorithms. \cite{guo2024connectinglargelanguagemodels}


Considers two widely used EAs: Genetic Algorithm and Differential Evolution with DE outperforming GA on most tasks \cite{guo2024connectinglargelanguagemodels}

Initial population consists of manually-written prompts to leverage human knowledge as well as some prompts generated by LLMs to reflect the fact that EAs start from random solutions to avoid local optima. \cite{guo2024connectinglargelanguagemodels}

DE-inspired approached builds on the idea that the common elements of the current best prompts need to be preserved \cite{guo2024connectinglargelanguagemodels}

Evoprompt performs best with roulette selection when compared with tournament and random selection. \cite{guo2024connectinglargelanguagemodels}

Similar results are achieved when population is initialized with the best and with random prompts, hinting that the crafted design of initial prompts is not essential. \cite{guo2024connectinglargelanguagemodels}


Previous research optimized zero-shot instructions and examples separately, overlooking their interplay and resulting in sub-optimal performance. \cite{cui2024phaseevounifiedincontextprompt}

There is a prevailing notion that prompt engineering sacrifices efficiency for performance due to the lengthening of prompts, but PhaseEvo actively shortens the prompts \cite{cui2024phaseevounifiedincontextprompt}

Current EA applications to prompt optimization suffer from extremely high computational cost and slow convergence speed due to the complexity of the high-dimensional search space. \cite{cui2024phaseevounifiedincontextprompt}

PhaseEvo alternates between two phases: exploration with evolution operators and exploitation using a feedback "gradient". \cite{cui2024phaseevounifiedincontextprompt}

TABLE 1 \todo{recreate} compares all 5 operators.  \cite{cui2024phaseevounifiedincontextprompt}

4 phases: initialization - lamarck or manual, local feedback mutation, global evolution with EDA and CR operators, local semantic mutation (paraphrasing) \cite{cui2024phaseevounifiedincontextprompt}

Candidates for evolution operators are selected based on a "performance vector", combining prompts that do not make the same mistakes.  \cite{cui2024phaseevounifiedincontextprompt}

When the performance improvement with an operator stagnates up to some operator-specific tolerance, the current phase is terminated. \cite{cui2024phaseevounifiedincontextprompt}

Evolution in phases outperforms random operator selection. \cite{cui2024phaseevounifiedincontextprompt}

PhaseEvo is the most cost-effective but still needs around 12 iterations and 4000 API calls. \cite{cui2024phaseevounifiedincontextprompt}


APE \cite{zhou2023largelanguagemodelshumanlevel} ran into problems with diminishing returns and abandoning the iterative approach entirely, Promptbreeder aims to solve this with a diversity-maintaining evolutionary algorithm for self-referential self-improvement of prompts \cite{fernando2023promptbreederselfreferentialselfimprovementprompt}

Prompt optimization techniques utilize the fact that LLMs are effective at generating mutations from examples and can encode human notions of interestingness and can be used to quantify novelty. \cite{fernando2023promptbreederselfreferentialselfimprovementprompt}

Self-referential system should improve the way it is improving, thus Promptbreeder used a "hyper-prompt" to optimize its meta-prompt \cite{fernando2023promptbreederselfreferentialselfimprovementprompt}

Uses a binary tournament genetic algorithm. \cite{fernando2023promptbreederselfreferentialselfimprovementprompt}

Uses a random uniformly sampled mutation operators out of 9 total from 5 broad categories for each replication event. \cite{fernando2023promptbreederselfreferentialselfimprovementprompt}

Zero-order mutation (creating a prompt from task description) generates new task prompts more aligned with the task description in the event the evolution diverges.  \cite{fernando2023promptbreederselfreferentialselfimprovementprompt}

LLMs tend to be biased to examples found later in EDA mutation lists. Lying to the LLM and telling it that the prompts are sorted by performance in a descending order improves diversity.  \cite{fernando2023promptbreederselfreferentialselfimprovementprompt}

Removing any self-referential operator in ablation is harmful under nearly all circumstances \cite{fernando2023promptbreederselfreferentialselfimprovementprompt}


\subsubsection{Metaprompting}
Metaprompting or "prompting to create prompts". Research shows that meta-prompting will always be superior to prompting through category theory\cite{dewynter2024metaprompting}.

\chapter{Implementation}
\section{Inference framework}
Taking inspiration from DSPy\cite{khattab2023dspycompilingdeclarativelanguage}, we first implement a simple LLM-calling framework 
capable of invoking several selected inference strategies. Motivations for this are twofold:
\begin{enumerate}
    \item DSPy is a young and ambitious project aiming at simplifying LLM pipeline design and optimization. 
    As we focus on single-stage prompt program optimization, this capability is not useful for our work. 
    Furthermore, due to the framework's infancy, it lacks proper documentation and sometimes exhibits unexpected behavior.
    \item Implementing the prompting techniques discussed in \ref{sec:inference} provides further insight into their workings and performance.
\end{enumerate}

\subsection{Structured generation}
Following current research trends\cite{zhang2025metapromptingaisystems}, we build our inference framework around a structured JSON template,
or a \texttt{Signature}. The \texttt{Signature} structure consists of input and output fields and additional instructions. 
These fields are populated by a \texttt{Field} data structure.
Of particular interest are the output fields, which hold the output name, desired type and optional description. 

Interactions with LLMs in a structured format benefit from better predictability. By implementing the \texttt{Signature} structure,
we can use LLMs as we would a function in any programming language. Functions in programming languages also have functions signatures which
specify input and output names and types.

When employing good naming practices the model can often deduce the task only by looking at output names and types.
Consider the simple \texttt{Signature} in figure \ref{box:simplesig}, which implicitly instructs the LLM to return a word with a meaning opposite to the provided input.

\begin{figurebox}{Simple Signature}{box:simplesig}
    \hlbox{ctuorange}{Word: \texttt{str}}  
    \hlbox{brown}{Antonym: \texttt{str}}
\end{figurebox}

For more complex tasks, filling the output descriptions or even adding explicit instructions is necessary.
In figure \ref{box:complexsig}, notice that it is possible to specify multiple inputs and outputs, which are then generated in the order given.

\begin{figurebox}{Complex Signature. }{box:complexsig}
    \hlbox{ctuorange}{Text: \texttt{str} (Student text)\\
    Grading guide: \texttt{str} (Steps to follow during evaluation)} 
    \hlbox{brown}{Evaluation: \texttt{str} (Textual feedback)\\
     Grade: \texttt{int} (Numerical grade 1-10)} 
    \hlbox{ctublue}{Grade the text.\\ You are an expert text evaluator. \\
    Use the grading guide to evaluate the test and give a final grade. 
    Use formal language and Markdown formatting in the evaluation\\ and output a 1-10 integer for the grade.}
\end{figurebox}

We will maintain this formatting style whenever showing a \texttt{Signature} structure in the future: \textcolor{ctuorange}{orange inputs}, \textcolor{brown}{brown outputs} and the optional \textcolor{ctublue}{blue instructions}.

Sufficiently large instruction-tuned LLMs are usually good at reliably producing JSON output.
For smaller models or more complex output structures, it might be necessary to use some form of constrained generation as discussed in \ref{sec:inference}.
A JSON schema could be constructed automatically from the \texttt{Signature} and passed into a parser-based sampler.
However, this is not necessary for our use-case and the only safeguard we implement is repeated generation in case of a parsing error.

\subsection{Predict method}
To facilitate \texttt{Signature}-powered generation, we implement a \texttt{predict} method that 
involves 1) prepending a developer prompt to the messages, and 2) parsing of \texttt{Signature} outputs.

\begin{figurebox}{Predict method developer prompt with highlighted prompt sections: directive, context, examples and format specifications.}{box:predictdev}
    \hlbox{ctulightblue}{You are an intelligent function that returns structured JSON outputs matching a given schema.
    }
    
    \hlbox{ctulightblue}{
    You will receive a JSON object containing: \\
        - `inputs`: a dictionary of named inputs \\
        - `outputs`: a dictionary specifying the expected output fields with their types and descriptions\\
        - `instructions`: a task or question to answer (optional)\\

    Your job is to:\\
        1. Understand the task from `instructions` or infer it from `inputs` and `outputs`\\
        2. Use the `inputs` to compute or generate the answer\\
        3. Respond **only** with keys from the `outputs` dictionary and values matching the described types
    }
    \hlbox{ctulightblue}{Only return a flat JSON object like:\\
    \{
    "field1": <value matching type and description>,\\
    "field2": <...>
    \}}
    
    \hlbox{ctulightblue}{Do not add metadata, explanations, or wrap outputs in additional structures.\\
    Do not include type names or field descriptions in the output.\\
    Your output must be strictly valid JSON and fill **all** requested output fields.}
\end{figurebox}

The developer prompt has to clearly explain to the LLM how to work with the JSON-based \texttt{Signature}.
In figure \ref{box:predictdev} notice the sections of the prompt following prompt engineering principles outlined in \ref{sec:preng}.

First, the directive states the task, then further context is added about the \texttt{Signature} data structure and the task.
Next, notice the example showing the proper output. Finally, few more clarifying instructions about the output format are added.
In experiments, this prompt is successful in incentivizing parsable outputs adhering to the \texttt{Signature} specifications.

Parsing the output presents some challenges as the LLM sometimes wraps the JSON output into a Markdown code block
or uses inconsistent escape sequences. We implement a simple parser based on regular expressions that is able to parse 
the majority of outputs. In case of a parsing issue or model failure, such as getting stuck in a generation loop, we add a repeated generation
feature.

\subsection{Inference techniques implementation}
Leveraging the \texttt{predict} method and the modular \texttt{Signature}-based interface, we implement a suite of inference-time prompting techniques. 
Each technique is realized through systematic modifications of the \texttt{Signature} fields, changing the developer prompt and the chaining of multiple generation steps 
and function calls. This design allows for modularity and reuse while preserving transparency.
We implement the following methods.
\begin{enumerate}
    \item \textbf{Chain-of-thought}\cite{NEURIPS2022_8bb0d291}: Prepends a reasoning field to the \texttt{Signature} outputs which forms a scratch pad for the LLM.
    \item \textbf{Chain-of-thought with Self-consistency}\cite{wang2023selfconsistencyimproveschainthought}: Multiple CoT generations with majority-voting.
    \item \textbf{ReAct}\cite{yao2023reactsynergizingreasoningacting}: Adding tools allows the LLM to interleave thoughts and action steps.
    \item \textbf{Program-of-thought}\cite{chen2023programthoughtspromptingdisentangling}: Two-stage CoT with Python-code execution
    \item \textbf{Reflexion}\cite{shinn2023reflexionlanguageagentsverbal}: After an initial generation, the model is prompted to self-critique and revise its output.
    \item \textbf{Tree of Thoughts}\cite{yao2023treethoughtsdeliberateproblem}: The problem is first decomposed and each step is expanded, forming a thought tree, which is then traversed with BFS or DFS.
\end{enumerate}

These techniques are however not the main focus of this bachelor's thesis, and we proceed without further discussion or evaluation.
In our prompt optimization method, we will utilize only the basic Chain-of-thought module.

\section{Datasets}
In this section, we discuss choosing datasets for testing our method and comparing various prompt optimization approaches. 
While searching available datasets, we focus on the following criteria:
\begin{enumerate}
    \item \textbf{Output complexity}: We focus on more complex outputs. Specifically, datasets with multiple-choice or Yes/No answers are omitted. 
    This disqualifies commonly used datasets as MMLU or BigBenchHard.
    \item \textbf{Contamination}: Recently, researchers have expressed concern\cite{white2025livebenchchallengingcontaminationlimitedllm} 
    whether benchmarks are reliable evaluations of models as they might appear in their training data. We omit most common datasets, such as GSM8k\cite{cobbe2021gsm8k}, which has been shown to have inflated scores for some models\cite{testing_language_models_on_a_held_out_high_school_national_finals_exam}.
    \item \textbf{Output verification}: We prefer to use simple automatic verification rather than using LLM-as-a-judge, which has been shown to be 
    biased in some circumstances\cite{ye2024justiceprejudicequantifyingbiases}. Neither do we use human feedback, which defeats the purpose of automatic prompt optimization.
    \item \textbf{Difficulty}: We omit tasks where models already have near-perfect score. 
    \item \textbf{Benefit from non-trivial instruction}: We focus on tasks where helpful hints and step-by-step tutorial-like instructions might increase the probability of successful solution.
\end{enumerate}
We now list the datasets that we will use for evaluation and explain why they were chosen.
\subsection{Livebench}
The Livebench\cite{white2025livebenchchallengingcontaminationlimitedllm} dataset is very recent and has been created with the issue of data contamination in mind.
It also addresses the issues of LLM-as-a-judge verification and all its categories can be verified automatically. It is also very challenging, with top models achieving $65\%$ accuracy\cite{white2025livebenchchallengingcontaminationlimitedllm}.

Out of the tasks available in Livebench, we select the \texttt{Connections} task from the \texttt{Language} subset. 
This task consists of sorting given words into non-trivial groups of four based on semantics, phonetics and other features. 
An ideal prompt would attempt to list multiple possible aspects based on which the words can be sorted and also include a helpful example.

\subsection{Code Contests}
Programming puzzles are a difficult and easily verifiable task. Although \texttt{CodeContests}\cite{li2022competition} is an older dataset, 
we anticipate this dataset presents reduced contamination risks compared to datasets with simpler outputs. With LLM-powered coding assistants on the rise, we
feel this is a relevant application area for our method. 

\subsection{Sequences}
We design a small but challenging dataset based on predicting the next number in an integer sequence.
Each sequence is created according to a formula with randomly selected coefficients. The formulas fall into several categories, for example
\begin{itemize}
    \item \textbf{Linear with modulo}: $s(i) = \operatorname{mod}_{q}(a_{1} i + b_{1})$
    \item \textbf{Sum}: $s(i) = \sum_{j=0}^{i-1}a_{1} j + b_{1}$
    \item \textbf{Alternating}: $s(i) = a_{1}i + a_{2}i(-1)^{i}$.
\end{itemize}
This tests the model's ability to 1. detect and understand patterns and 2. systematically perform simple arithmetic. 
In practice, we will optimize just for a single sequence category and observe whether the optimizer evolves a prompt with a tutorial for the specific sequence category.
Experiments showed that the \texttt{Alternating} class of sequences has a good difficulty balance, and we will use it for evaluation.

\section{Evaluation Metrics}
The evaluation metric defines the optimization goal and thus forms its central component. 
Most evaluation metrics are task-specific and divisible into two categories based on whether they are used in a supervised or self-supervised context.
\subsection{Metrics for Supervised Optimization}
Supervised optimization is supported by gold labels and its underlying metrics all perform comparisons between the results and the gold labels.
These include classification metrics like accuracy or Hamming Loss, regression metrics like Mean Squared Error, and many others.

All three main benchmarks that we will use (\texttt{Connections}, \texttt{CodeContests}, \texttt{Sequences}) 
fall into this category. For each benchmark we use a simple accuracy metric. Given a dataset $\mathcal{D}$ and questions $q$ and gold labels $g$, $(q,g) \in \mathcal{D}$:
\begin{enumerate}
    \item \textbf{Connections}:  $\mathcal{F}_{\mathcal{D}_{\text{Conn}}}(q, g) = \operatorname{Overlap}(\operatorname{Groups}(q), \operatorname{Groups}(g))$
    \item \textbf{CodeContests}: $\mathcal{F}_{\mathcal{D}_{\text{Code}}}(q, g) = \operatorname{FinishesExecution}(q) + \operatorname{PassesAllCases}(q, g)$
    \item \textbf{Sequences}: $\mathcal{F}_{\mathcal{D}_{\text{Seq}}}(q, g) = \operatorname{Equals}(q, g)$
\end{enumerate} 

\subsection{Metrics for Self-Supervised Optimization}\label{sec:ssometrics}
In self-supervised contexts, metrics are usually based on reward models pretrained on human preference or environment data.
To allow our method to be applied to gold label-free problems, we turn to LLM-based direct pairwise comparisons.
Given a dataset $\mathcal{D}$ with queries $q$, output $y$ produced by prompt $P \in \mathscr{P}$ and a set of completions $\mathcal{C}_{q}$ for each query.
\begin{equation}
    \mathcal{F}_{\mathcal{D}}^{\text{pairwise}}(q, y, \mathcal{C}_{q}) = \operatorname{WinRate}(\{\operatorname{Compare}(q,y,c)\vsep c\in \mathcal{C}_{q}\}).
\end{equation}
In practice, we combine the output comparison with comparing the output's respective prompts.
These comparisons are then used as optimization signals in the \texttt{Feedback} operator.

\section{Optimization Framework}
Although our first implementation attempt utilized an evolutionary algorithm, we will use a basic population-based hill-climber algorithm.
This design decision has several reasons.
\begin{enumerate}
    \item Most PO research uses a hill-climber architecture.
    \item EAs suffer from slow convergence compared to state-of-the-art hill-climber PO\cite{xiang2025selfsupervisedpromptoptimization}.
    \item PO is complex as it is, and more complicated architectures only introduce more hyperparameters.
\end{enumerate}


\begin{algorithm}
    \caption{Prompt Optimization Hill-Climber}
    \label{alg:promptoptimloop}
    \KwIn{Dataset $\mathcal{D}$, Population size $S$, Iteration count $I$, Batch size $B$}
    \KwOut{Optimized Prompts $\mathscr{P}^{\star}$}
    $\mathcal{D}_{\text{train}}, \mathcal{D}_{\text{dev}}, \mathcal{D}_{\text{test}} \gets \operatorname{Split}(\mathcal{D})$ \tcp{Generate training splits}
    $\mathscr{P} \gets \operatorname{InstructionInduction}(\mathcal{D}_{\text{train}})$ \tcp{Induce initial prompts}
    $i \gets 0$ \tcp{Initialize iteration count}
    $\mathcal{C} \gets \{\}$ \tcp{Initialize solutions} 
    $\mathcal{E} \gets \{\}$ \tcp{Initialize scores}
    $\mathcal{A} \gets \mathscr{P}$ \tcp{All prompts}
    \While{$i<I$}{
        $Q, G \gets \operatorname{RandomSample}(\mathcal{D}_{\text{dev}}, B)$ \\
        $\mathcal{C} \gets \{\mathcal{C}_{q}^{\mathscr{P}}\vsep q \in Q\}$ \\
        $\mathcal{E} \gets \operatorname{Evaluate}(\mathcal{C}, G)$ \\
        $\mathscr{P} \gets \operatorname{Selection}(\mathscr{P}, \mathcal{E})$ \tcp{Pruning} 
        $\mathscr{P} \gets \operatorname{Expand}(\mathscr{P}, \mathcal{C}, \mathcal{E}, \mathcal{D}_{\text{train}})$ \\
        $\mathcal{A} \gets \mathcal{A} \cup \mathscr{P}$ \\
    }
    %$Q_{\text{test}}, G_{\text{test}} \gets D_{\text{test}}$\\
    %$\mathcal{C}_{\text{test}} \gets \{\mathcal{C}_{q}^{\mathcal{A}}\vsep q \in Q_{\text{test}}\}$\\
    %$\mathcal{E}_{\text{test}} \gets \operatorname{Evaluate}(\mathcal{C}, G_{\text{test}})$\\
    $P^{\star} \gets \underset{P\in\mathcal{A}}{\operatorname{argmax}}(\mathcal{E}_{\mathcal{D}_{\text{test}}}(P))$\\
    \Return{$P^{\star}$}
\end{algorithm}

In Algorithm \ref{alg:promptoptimloop} we iterate on the general algorithm \ref{alg:genoptimloop}. 
We will discuss the design of functions used in \ref{alg:promptoptimloop} in following sections.

\begin{itemize}
    \item \textbf{Expand}: The $\operatorname{Expand}$ function can be filled with various expansion operators, of which $\operatorname{InstructionInduction}$
    is a special case. 
    \item \textbf{Evaluate}: Evaluating and identifying the most promising prompts is handled by the $\operatorname{Evaluate}$ operator, which uses task-specific automatic evaluation or LLM-feedback.
    \item \textbf{Selection}: The $\operatorname{Selection}$ operator prunes the population and should maintain only the most promising and diverse prompts for the next expansion.
\end{itemize}

\subsection{Expansion Operator Design}
Expansion operators' job is extending the optimization population with new prompts. Remember notation from \ref{eq:metaprompting}:
\begin{equation*}
    P = \mathscr{M}_{\text{optim}}(M\vsep \mathcal{R}).
\end{equation*}
Notice the use of $\mathscr{M}_{\text{optim}}$, which utilizes non-zero sampling temperature $\tau > 0$ to encourage output diversity. 
Evidently the prompt generation task can be separated into two independent problems: 1. crafting the optimal \textit{Meta-prompt} $M$ 
and 2. designing a data retrieval function $\mathcal{R} = \mathcal{R}(\mathscr{P}, \mathcal{C}, \mathcal{D}, \mathcal{E})$.
The operators' design should address the following challenges:
\begin{enumerate}
    \item \textbf{Loss of generality}: When using task samples $(q, g) \in \mathcal{D}$, the model $\mathscr{M}_{\text{optim}}$ might focus on a single query $q$ and thus fail to generate general instructions.
    \item \textbf{Loss of diversity}: Even for  $\mathscr{M}_{\text{optim}}$ with $\tau>0$, the resulting prompts can be very similar and fail to explore the prompt space $\mathcal{I}$. 
    This ties into a broader exploration vs. exploitation balance issue.
    \item \textbf{Lack of optimization signal}: Research\cite{he2024crispomultiaspectcritiquesuggestionguidedautomatic}\cite{xiang2025selfsupervisedpromptoptimization} suggests that $\mathscr{M}_{\text{optim}}$ 
    can make use of feedback on prompts' outputs and that these textual signals are more effective than numerical scores.
    \item \textbf{Out of distribution \textit{Meta-prompt}}: Prompt engineering is a novel research area and does not have a substantial support in the LLM's training corpus.
    The \textit{Meta-prompt} $M$ thus has to be carefully constructed to help the model output relevant prompts.
\end{enumerate}

We now discuss the design of each prompt generation operator and display their signatures and \textit{meta-prompts}. 
Note that all operators are ultimately used in a CoT context, where a \texttt{reasoning} field is prepended to each signature's outputs.
\subsubsection{Lamarckian}
Instruction Induction\cite{honovich2022instructioninductionexamplesnatural} is used by many PO methods and often referred to as \texttt{Lamarckian Mutation}. 
We will adopt this terminology from now on and design our \texttt{Lamarckian} operator. 
Design of its meta-prompt, shown in figure \ref{box:lamarcksig}, takes into account the design challenges mentioned earlier by 1. warning the LLM to be general and not to focus on a single example, 
2. clearly states the problem using a directive and formatting specifications. 

The problem with diversity still persists, and we consider two approaches to solving it. 
We can increase the model's creativity by increasing its sampling temperature. Another approach is to use
some kind of \textit{seed}, for example a \textit{persona}. We experiment with using personas from PersonaHub\cite{ge2024scalingsyntheticdatacreation}.
Authors of this paper argued that seeding generation with the persona helps with creating novel synthetic data. 

For the data retrieval part, \texttt{Lamarckian} utilizes only examples of the datasets. 
We randomly sample $N$ examples from a separate training split. So
\begin{equation}
    \mathcal{R}_{\text{L}}(\mathcal{D}) = \operatorname{RandomSample}(\mathcal{D}_{\text{train}}, N)
\end{equation}
\begin{figurebox}{Lamarckian Signature. Formatting specifications trimmed.}{box:lamarcksig}
    \hlbox{ctuorange}{Task examples: \texttt{str} (Samples from a problem class) \\
    Persona (Optional): \texttt{str} (Assume this persona when writing the prompt)} 
    \hlbox{brown}{Prompt proposal: \texttt{str} (Instructions for solving the problem)} 
    \hlbox{ctublue}{Craft a \textbf{general} developer prompt to help an LLM with solving a class of problems.\\
    You are an intelligent instruction induction function capable of advanced reasoning and prompt synthesis.\\
    Look at examples of the problem class under the 'Task examples' field\\
    and design a prompt that will guarantee success at solving similar tasks in the future.\\
    Make sure your instructions are \textbf{TRULY GENERAL} and apply to all given samples \textbf{simultaneously}.
}
\end{figurebox}
\newpage
\subsubsection{Iterative}
The \texttt{Iterative} operator is one of the most common and simplest operators. 
It uses a sequence of prompts and their scores in ascending order.
The hope is for the LLM to deduce the optimization direction by looking at the differences in the prompts and incite it to continue the pattern.

Although some research\cite{yang2024largelanguagemodelsoptimizers} only uses the top prompts, we opt for a roulette selection method
and sort to prompts by score in ascending order. The number $N$ is a hyperparameter dictating how many prompts to sample.
We define the retrieval function as
\begin{equation}
   \mathcal{R}_{\text{I}}(\mathscr{P}, \mathcal{E}) = \operatorname{SortByScore}(\operatorname{RouletteSampling}(\mathscr{P}, \mathcal{E}, N), \mathcal{E})
\end{equation}
Other methods\cite{tang2024unleashingpotentiallargelanguage} additionally augment the \textit{meta-prompt} with task samples, similar to the \texttt{Lamarckian} operator.

In the \textit{meta-prompt}, we instruct the LLM to try to follow the sequence. Also, we specifically say to 'craft a new prompt'
as opposed to 'improve a prompt' to incite more novelty. For formatting, we use the same instruction set as in the \texttt{Lamarckian}.
\begin{figurebox}{Iterative Signature. Formatting specifications trimmed.}{box:itersig}
    \hlbox{ctuorange}{Old prompts: \texttt{list} (List of previous prompts with scores)}
    \hlbox{brown}{Prompt proposal: \texttt{str} (Better prompt)} 
    \hlbox{ctublue}{Craft a new prompt for an LLM.
    
    You are an intelligent pattern continuation function capable of advanced reasoning and prompt synthesis.\\
    You are given a history of past prompts along with their scores.\\
    They are listed in ascending order of fitness.\\
    Follow the sequence and design an improved prompt. \\
}
\end{figurebox}

\subsubsection{Reflective}
Recent PO literature\cite{xiang2025selfsupervisedpromptoptimization} shifts to using LLM outputs as optimization signals
and argues that utilizing only numerical signals is ineffective. To address this, we design an exploitative operator, which
aims to fix faults in the prompt by analyzing its failed attempt at a task sample. 

To achieve this, a more complex \texttt{Signature} is utilized. Its outputs guide the LLM to first critique the original prompt
and then improve it. Instructions are more complete with a step-by-step guide which explains the task clearly.
Note that 1. now we use "improve" wording, 2. we stress to only alter the prompt \textit{slightly}. This is done due to 
frequent observation of the model just creating an entirely different prompt only applicable to the single example task.
For formatting, we use the same instructions as in previous \textit{meta-prompts}.

In $\mathcal{R}$, we want to select the worst possible attempt. This means we optimize "from the bottom up" and try to bootstrap the
worst prompts. The retrieval function is
\begin{equation}
    \mathcal{R}_{\text{R}}(\mathscr{P}, \mathcal{C}, \mathcal{D}, \mathcal{E}) = \operatorname{JoinAttemptWithTask}(\operatorname{FindWorstAttempt}(\mathscr{P}, \mathcal{C}, \mathcal{E}, \mathcal{D})
\end{equation}

\begin{figurebox}{Reflective Signature. Formatting specifications trimmed.}{box:reflexsig}
    \hlbox{ctuorange}{Original prompt: \texttt{str} (Improve this prompt) \\
    Task question: \texttt{str} (Task on which the prompt was used) \\
    Solution: \texttt{str} (What the original prompt produced)}
    \hlbox{brown}{Original prompt critique: \texttt{str} (Faults in the original prompt) \\
    Prompt proposal: \texttt{str} (Improved prompt)} 

    \hlbox{ctublue}{Improve a prompt for an LLM.
    
    You are an intelligent reflection function capable of advanced reasoning and prompt synthesis.\\
    Follow these steps to craft a better prompt:\\
    - Analyze the original prompt and its suboptimal performance on a task sample.\\
    - Find failure points in the solution and cross-reference to identify weaknesses in the prompt.\\
    - Think of a critique that captures your findings\\
    - Apply your critique to \textit{slightly} alter the original prompt to improve it.\\
    Your improved prompt should still be \textbf{widely applicable and generic}.}
\end{figurebox}


\subsubsection{Feedback}
As we mentioned earlier, the \texttt{Feedback} operator is suitable for use in self-supervised settings.
It leverages reasoning traces from pairwise LLM-based comparisons, discussed in \ref{sec:ssometrics}. 
Let $\mathcal{E}_{\text{comp}}$ hold textual comparisons of each prompt and their attempts and
$P_{\text{base}} = \operatorname{RandomSample}(\mathscr{P})$. Then
\begin{equation}
    \mathcal{R}_{\text{F}}(\mathscr{P}, \mathcal{E}_{\text{comp}}) = \{P_{\text{base}}, \operatorname{GetComparisons}(P_{\text{base}}, \mathcal{E}_{\text{comp}})\}
\end{equation}
In the \textit{Meta-prompt}, we frame the task as critique synthesis and use "improve" wording to guide the LLM to start from the base prompt.
We also explain that each comparison has a different verdict and the base prompt might not always be the winner. For the formatting guide, we use the same instructions
as in the previous operators.

For large populations or tasks producing long prompts, we might run into issues with LLM context window length. 
However for our purpose, modern LLMs provide more than sufficient context limits. 
\begin{figurebox}{Feedback Signature. Formatting specifications trimmed.}{box:feedbacksig}
    \hlbox{ctuorange}{
        Base prompt: \texttt{str} (Improve this prompt) \\
        Comparisons: \texttt{str} (Base prompt compared to others)
    }
    \hlbox{brown}{Prompt proposal: \texttt{str} (Improved prompt)} 
    \hlbox{ctublue}{Improve a prompt for an LLM.

    You are an intelligent critique synthesis function capable of advanced reasoning. \\
    You are given a base prompt and a list of comparisons between the base prompt and other prompts.\\
    Some other prompts are better than the base prompt, some are worse.\\
    Your task is to analyze the comparisons and synthesize a new prompt that incorporates the feedback.}
\end{figurebox}

\subsubsection{Paraphrase}
To serve as another baseline for other operators, we implement a simple \texttt{Paraphrase} operator.
This operator performs random search in the prompt space by changing the wording and structure of a prompt.
The prompt is selected via the retrieval function
\begin{equation}
    \mathcal{R}_{\text{P}}(\mathscr{P}, \mathcal{E}) = \operatorname{RouletteSampling}(\mathscr{P}, \mathcal{E}).
\end{equation}
This method uses no optimization signal or improvement instructions and relies on pure chance of finding a more potent prompt.


\begin{figurebox}{Paraphrase Signature. Formatting specifications trimmed.}{box:parasig}
    \hlbox{ctuorange}{Original prompt: \texttt{str} (Prompt to paraphrase)}
    \hlbox{brown}{Prompt proposal: \texttt{str} (Paraphrased prompt)} 
    \hlbox{ctublue}{Paraphrase a prompt for an LLM.
                    
    You are an intelligent paraphrasing function capable of advanced reasoning and prompt synthesis.\\
    You are given a prompt and your task is to paraphrase it. \\
    Use synonyms and change the structure of the prompt but keep it semantically equivalent.}
\end{figurebox}
In figure \ref{box:formatting} we show the formatting specification part, which is identical for all prompt generation \textit{meta-prompts}.

\begin{figurebox}{Formatting specifications, which are the same for all operators}{box:formatting}
    Use Markdown formatting in your final answer to indicate bullet points and whatever else necessary.\\
    As a placeholder for the task question, '<INSERT TASK QUESTION HERE>' should be used exactly ONCE.\\
    In the final answer, do not include a title or any additional data, just the prompt.
\end{figurebox}

\subsection{Selection Operator}
At the start of each optimization step, we select $n_{\text{continue}}$ prompts to continue in the process and purge the rest. 
To achieve better prompt diversity, a method based on edit distance is used. 

This method, outlined in Algorithm \ref{alg:duplicpurge},
removes the closest prompt for each prompt, starting from the best prompts. This ensures that performant prompts are kept and their worse-performing duplicates are deleted.
We opt to use edit distance instead of semantic similarity, like BERT embeddings.

\begin{algorithm}
    \caption{Purge Duplicates}
    \label{alg:duplicpurge}
    \KwIn{Population $\mathscr{P}$, Pruning factor $f_{\text{prune}}$}
    \KwOut{Pruned population $\mathscr{P}_{\text{pruned}}$}
    $\mathscr{P}_{\text{sorted}} \gets \operatorname{SortByScore}(\mathscr{P}, \mathcal{E})$ \\
    $n_{\text{continue}} \gets \vert\mathscr{P}\vert(1-f_{\text{prune}})$
    $i \gets 0$
    \While{$i<n_{\text{continue}}$} {
        $P_{\text{select}} \gets \operatorname{GetFirst}(\mathscr{P}_{\text{sorted}})$ \\ 
        $P_{\text{purge}} \gets \underset{P\in\mathscr{P}\mid P \neq P_{\text{select}}}{\operatorname{argmax}} \operatorname{LevenshteinRatio}(P, P_{\text{select}})$ \\
        $\operatorname{Remove}(\mathscr{P}, P_{\text{purge}})$
    }
    $\mathscr{P}_{\text{pruned}} \gets \mathscr{P}$\\
    \Return{$\mathscr{P}$}
\end{algorithm}


\chapter{Experiments}
%\section{Operators for Supervised Optimization}
All our experiments were conducted throught the \texttt{OpenRouter API}, which offers many LLMs from different providers.
We will differentiate between optimizer LLM $\mathcal(M)_{\text{optim}}$ and $\mathcal(M)_{\text{optim}}$.
For the former, we use a bigger model, namely \texttt{Nvidia Llama 3.1 Nemotron Ultra} with 253 billion parameters and $\tau=0.75$, and a smaller model, \texttt{Gemma3 27B} with $\tau=0$
for the latter.

We test our method against 3 datasets, \texttt{Connections}, \texttt{CodeContests} and \texttt{Sequences}, which have 30 samples each 
and will form 3 equal splits $\mathcal{D}_{\text{train}}, \mathcal{D}_{\text{dev}}, \mathcal{D}_{\text{test}}$.

To create a strong baseline, we begin by creating 50 with Instruction Induction through the \texttt{Lamarckian} operator.

\subsection{Connections}
\begin{table}[htbp]
    \centering
    \captionsetup{font=small}
    \caption{Optimization Progress Across Prompt Dimensions}
    \renewcommand{\arraystretch}{1.4} % Row padding
    
    \begin{tabular}{|c||*{16}{c|}}
    \hline
    \rowcolor{ctublue!30}
    \textbf{Step} &
    \multicolumn{4}{c|}{\cellcolor{ctublue!20}\textbf{Reflective}} &
    \multicolumn{4}{c|}{\cellcolor{ctublue!20}\textbf{Iterative}} &
    \multicolumn{4}{c|}{\cellcolor{ctublue!20}\textbf{Feedback}} &
    \multicolumn{4}{c|}{\cellcolor{ctublue!20}\textbf{Paraphrase}} \\
    \hline
    \rowcolor{ctublue!10}
    \textbf{} &
    \multicolumn{1}{c|}{\tiny Top, No-seed} & \multicolumn{1}{c|}{\tiny Mid, No-seed} & \multicolumn{1}{c|}{\tiny Top, Persona} & \multicolumn{1}{c|}{\tiny Mid, Persona} &
    \multicolumn{1}{c|}{\tiny Top, No-seed} & \multicolumn{1}{c|}{\tiny Mid, No-seed} & \multicolumn{1}{c|}{\tiny Top, Persona} & \multicolumn{1}{c|}{\tiny Mid, Persona} &
    \multicolumn{1}{c|}{\tiny Top, No-seed} & \multicolumn{1}{c|}{\tiny Mid, No-seed} & \multicolumn{1}{c|}{\tiny Top, Persona} & \multicolumn{1}{c|}{\tiny Mid, Persona} &
    \multicolumn{1}{c|}{\tiny Top, No-seed} & \multicolumn{1}{c|}{\tiny Mid, No-seed} & \multicolumn{1}{c|}{\tiny Top, Persona} & \multicolumn{1}{c|}{\tiny Mid, Persona} \\
    \hline
    
    \rowcolor{ctuorange!20}
    Baseline & -- & -- & -- & -- & -- & -- & -- & -- & -- & -- & -- & -- & -- & -- & -- & -- \\
    \hline
    
    1 &  &  &  &  &  &  &  &  &  &  &  &  &  &  &  &  \\
    \hline
    2 &  &  &  &  &  &  &  &  &  &  &  &  &  &  &  &  \\
    \hline
    3 &  &  &  &  &  &  &  &  &  &  &  &  &  &  &  &  \\
    \hline
    4 &  &  &  &  &  &  &  &  &  &  &  &  &  &  &  &  \\
    \hline
    5 &  &  &  &  &  &  &  &  &  &  &  &  &  &  &  &  \\
    \hline
    6 &  &  &  &  &  &  &  &  &  &  &  &  &  &  &  &  \\
    \hline
    7 &  &  &  &  &  &  &  &  &  &  &  &  &  &  &  &  \\
    \hline
    8 &  &  &  &  &  &  &  &  &  &  &  &  &  &  &  &  \\
    \hline
    9 &  &  &  &  &  &  &  &  &  &  &  &  &  &  &  &  \\
    \hline
    10 &  &  &  &  &  &  &  &  &  &  &  &  &  &  &  &  \\
    \hline
    
    \end{tabular}
    \end{table}
    
\subsection{Code Contests}

\subsection{Sequences}

\section{Self-Supervised Optimization for Open-Ended Tasks}

%\include{future_work}

%\chapter{Conclusion}
%\section{Conclusion}
In this bachelor's thesis, we conducted a comprehensive survey of cutting-edge literature, spanning 
inference-time scaling methods, prompt engineering and automatic prompt optimization.
We conceptually framed our research topic as a part of the compile-time paradigm, contrasting against training-time and inference-time scaling.

Drawing inspiration from contemporary research trends, we developed a structured generation framework, implementing several of the discussed inference-time techniques.
We established strong mathematical formalism and, building upon our structured generation framework, designed a modular prompt optimizer.
Our prompt optimization framework, a population-based hill-climber algorithm, features four distinct expansion operators using \textit{meta-prompts} - prompts which generate other prompts.
Additionally, we include an initialization operator utilizing an instruction induction \textit{meta-prompt} and a diversity-maintaining pruning feature.
One of our expansion approaches is built upon a LLM-based pairwise comparison technique, suitable for self-supervised contexts.

In our primary experiment, we focused on comparing our four \textit{meta-prompting} approaches against each other and an instruction induction baseline.
We evaluate all variants of our method on three tasks from various disciplines, including a custom numerical pattern prediction task.
All evaluations were repeated three times for higher statistical significance.
Although the performance of some variants of our method did not meet expectations, we demonstrated that it is possible to surpass a strong instruction induction baseline even with weak initialization prompts. 
Our results show that LLMs' sensitivity to input prompts is just as pronounced in the design of \textit{meta-prompts}, highlighting the need for their careful design.

Our secondary experiment focused on optimizing prompts for creative tasks in a self-supervised context.
We applied our method to creative text and image generation tasks with interesting results. 
These results showcase the wide applicability of prompt optimization in general.

Working on this thesis has been an excellent opportunity to gain hands-on experience with cutting-edge research in a rapidly advancing field.
Although our method did not achieve impressive results, we incorporated ideas from a plethora of recent papers, 
and we walk away with many ideas for future research.
We discuss these possible research avenues in the next section.

\section{Future work}
Prompt optimization is an exciting and exceptionally active branch of research. 
Our experiments inspire further work on this topic, and we will discuss possible research topics in this section.
\begin{enumerate}
    \item \textbf{Statistical methods for evaluation}: Evaluation uses the majority of computation resources during optimization. This was most pronounced in our naive implementation of the \texttt{Feedback} operator, where we performed pairwise comparisons for every pair of prompts. In literature\cite{pryzant2023automaticpromptoptimizationgradient}\cite{zhang2024sprigimprovinglargelanguage}, researchers utilize more sophisticated evaluation methods, like Upper Confidence Bound algorithms (UCB), which can help with distribution of computational resources to the most promising prompts.
    \item \textbf{Finding optimal \textit{meta-prompts}}: The ultimate goal of this research branch is finding the general \textit{meta-prompt}, which can create optimal prompts for any task. This is theoretically possible\cite{dewynter2024metaprompting}, but exceptionally difficult. Some research\cite{fernando2023promptbreederselfreferentialselfimprovementprompt} experimented with optimizing the \textit{meta-prompts}, but most methods hand-craft them. More rigorous experimentation and application of prompt engineering techniques, possible drawing inspiration from system prompts of foundational LLMs, might uncover much better \textit{meta-prompts}.
    \item \textbf{Improved structured generation}: Our structured generation method is in no way perfect and could represent a bottleneck, limiting the LLM's capabilities. More work is needed to find the best format for LLM usage in this context. Although modern LLM's are proficient JSON generators, recently leaked system prompts reveal that some top LLM's use XML tags instead. 
    \item \textbf{Prompt representation structures}: More sophisticated prompt representation, like using acyclic directed graphs\cite{zhang2024sprigimprovinglargelanguage}, could allow us to make more precisely defined changes to the prompts. This could help retain diversity and task relevancy as the optimized prompts grow ever longer for more complicated tasks.
    \item \textbf{Applying meta-heuristics}: The majority of research made use of basic meta-heuristics, such as the hill-climber algorithms. Application of evolutionary algorithms\cite{guo2024connectinglargelanguagemodels} and other meta-heuristics\cite{pan2024plumpromptlearningusing} presents an exciting intersection of deep learning and classical optimization.
    \item \textbf{Inference-time techniques}: For all LLM calls, we used the Chain-of-thought end-point of our inference framework. It is possible to use other inference techniques such as Tree-of-Thoughts in its place. Alternatively, dedicated reasoning models which perform Chain-of-thought automatically could be utilized.
    \item \textbf{Agentic optimization}: An interesting possibility is handing the control over the optimization process over to a \textit{manager} LLM that would control it using tools in a ReAct-like setting. We conducted preliminary experiments on this topic, but further exploration is needed. 
\end{enumerate}

These topics provide intriguing avenues for future research. Another possibility is increasing the experiment scale, either by making them more efficient
using clever optimizations or more cost-effective models, or by increasing the computation budget. 


\appendix

\printindex


\bibliographystyle{ieeetr}
\bibliography{references}

\include{appendix}
%\ctutemplate{specification.as.chapter}
\end{document}