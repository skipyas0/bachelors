% arara: pdflatex: { synctex: yes }
% arara: makeindex: { style: ctuthesis }
% arara: bibtex

% The class takes all the key=value arguments that \ctusetup does,
% and a couple more: draft and oneside
\documentclass[twoside]{ctuthesis}

\ctusetup{
	preprint = \ctuverlog,
	mainlanguage = english,
	titlelanguage = english,
	otherlanguages = {czech},
	title-czech = {Meta-prompty pro optimalizaci promptu velkého jazykového modelu},
	title-english = {Meta-prompts for LLM Prompt Optimization},
	%subtitle-czech = {abcd},
	%subtitle-english = {abcd},
	doctype = B,
	faculty = F3,
	department-english = {Department of Control Engineering},
	department-czech = {Katedra řídicí techniky},
	author = {Vojtěch Klouda},
	supervisor = {Ing. Jan Drchal PhD.},
	supervisor-address = {Resslova 307/9 Praha, E-322},
%	supervisor-specialist = {John Doe},
	fieldofstudy-english = {Cybernetics and Robotics},
	%subfieldofstudy-english = {Natural Language Processing},
	fieldofstudy-czech = {Kybernetika a robotika},
	%subfieldofstudy-czech = {Zpracování přirozeného jazyka},
	keywords-czech = {velký jazykový model, optimalizace promptů, metody inference, promptovací techniky, strukturovaná generace},
	keywords-english = {large language model, prompt optimization, inference methods, prompting techniques, structured generation},
	day = 10,
	month = 5,
	year = 2025,
	specification-file = {declaration_merged.pdf},
%	front-specification = true,
%	front-list-of-figures = false,
%	front-list-of-tables = false,
%	monochrome = true,
%	layout-short = true,
}

\ctuprocess
\usepackage[linesnumbered, ruled, vlined]{algorithm2e}

\addto\ctucaptionsczech{%
	\def\supervisorname{Vedoucí}%
	\def\subfieldofstudyname{Studijní program}%
}

\ctutemplateset{maketitle twocolumn default}{
	\begin{twocolumnfrontmatterpage}
		\ctutemplate{twocolumn.thanks}
		\ctutemplate{twocolumn.declaration}
		\ctutemplate{twocolumn.abstract.in.titlelanguage}
		\ctutemplate{twocolumn.abstract.in.secondlanguage}
		\ctutemplate{twocolumn.tableofcontents}
		\ctutemplate{twocolumn.listoffigures}
	\end{twocolumnfrontmatterpage}
}

% todo command
\newcommand{\todo}[1]{\textsuperscript{\textbf{\textcolor{red}{#1}}}}
\newcommand{\vsep}{\, \vert \,}

\newcommand{\maxmean}[2]{#1 {\color{gray}\scriptsize (#2)}}


% Theorem declarations, this is the reasonable default, anybody can do what they wish.
% If you prefer theorems in italics rather than slanted, use \theoremstyle{plainit}
\theoremstyle{plain}
\newtheorem{theorem}{Theorem}[chapter]
\newtheorem{corollary}[theorem]{Corollary}
\newtheorem{lemma}[theorem]{Lemma}
\newtheorem{proposition}[theorem]{Proposition}

\theoremstyle{definition}
\newtheorem{definition}[theorem]{Definition}
\newtheorem{example}[theorem]{Example}
\newtheorem{conjecture}[theorem]{Conjecture}

\theoremstyle{note}
\newtheorem*{remark*}{Remark}
\newtheorem{remark}[theorem]{Remark}

\setlength{\parskip}{5ex plus 0.2ex minus 0.2ex}

% Abstract in Czech
\begin{abstract-czech}
	Tvorba promptů pro velké jazykové modely (LLM) představuje významnou překážku v jejich plném využití a je obtížná jak pro odborníky, tak pro běžné uživatele.
  V této bakalářské práci navrhujeme jednoduchou metodu optimalizace promptů na základě populační varianty hill-climber algoritmu, postavenou na vlastním frameworku pro strukturovanou generaci.
  Využíváme principy promptového inženýrství k vytvoření několika \textit{meta-prompting} technik, které testujeme na různorodých úlohách oproti silnému referenčnímu řešení, vytvořenému pomocí Instruction Induction.
  Zároveň přinášíme přehled aktuální literatury v rychle se vyvíjejícím oboru optimalizace promptů a zasazujeme tuto úlohu do širšího kontextu training-time a inference-time scalingu.
  Naše jednodušší \textit{meta-prompting} techniky dosahují v experimentech nejlepších výsledků a překonávají výchozí řešení i komplikovanější varianty.
  Naše práce ukazuje možnost využití optimalizace promptů jak v analytických, tak i v kreativních kontextech, ale zdůrazňuje citlivost optimalizace s LLM na návrh \textit{meta-promptů}.
  Kód je dostupný na \url{https://github.com/skipyas0/prompt_optimizer}.
\end{abstract-czech}

% Abstract in English
\begin{abstract-english}
  Prompt design for Large Language Models (LLMs) remains a key bottleneck in leveraging their full capabilities, posing challenges for both expert practitioners and everyday users.
  In this bachelor's thesis, we develop a simple population-based hill-climber prompt optimization method built atop a custom framework for structured generation inference.
  Specifically, we apply prompt engineering principles to create several \textit{meta-prompting} approaches and evaluate them across multiple tasks against a strong Instruction Induction baseline.
  Additionally, we survey current literature in the rapidly evolving field of prompt optimization and frame it in the broader context of training-time and inference-time scaling. 
  Our simpler \textit{meta-prompting} approaches perform best in experiments, outperforming both the baseline and more complex variants.
  Our work showcases the applicability of prompt optimization to both reasoning-intensive and creative tasks, while highlighting the sensitivity to \textit{meta-prompt} design in LLM-based optimization.
  Code is available at \url{https://github.com/skipyas0/prompt_optimizer}.
\end{abstract-english}

% Acknowledgements / Podekovani
\begin{thanks}
  I would like to thank my supervisor Jan Drchal for his time and helpful notes. 
  
  Additionally, I would like to thank the RCI group for access to their computation cluster.
\end{thanks}

% Declaration / Prohlaseni
\begin{declaration}
I declare that I have completed the submitted work independently and that I have cited all sources used.

Artificial intelligence tools were used to a limited extent during the work, for example for searching relevant literature, suggesting stylistic edits, or as a coding assistant.

In Prague, \ctufield{day}.~\monthinlanguage{title}~\ctufield{year}
\end{declaration}

% Only for testing purposes
\listfiles
\usepackage[pagewise]{lineno}
\usepackage{lipsum,blindtext}
\usepackage{mathrsfs} % provides \mathscr used in the ridiculous examples
\usepackage{pdflscape}
\usepackage{tabularx}
\usepackage{booktabs}
\usepackage{array} % put in preamble if not already there
\usepackage[table]{xcolor}
\usepackage[most]{tcolorbox}
\usepackage{xparse}
\usepackage{expl3}
\usepackage{subcaption}
\usepackage{colortbl}
\usepackage{multirow}
\usepackage{geometry}
\usepackage{caption}
\usepackage{tikz}
\usepackage{hyperref}
\usetikzlibrary{positioning, arrows.meta, shapes, decorations.pathreplacing, calc}
%\geometry{margin=1in}

% \newcommand{\hlbox}[2]{%
%   \begin{tcolorbox}[colback=#1!10!white,
%                     colframe=#1!80!black,
%                     boxrule=0.8pt,
%                     arc=4pt,
%                     left=6pt,
%                     right=6pt,
%                     top=4pt,
%                     bottom=4pt,
%                     enhanced,
%                     breakable]
%   #2
%   \end{tcolorbox}
% }
\NewDocumentCommand{\hlbox}{m +m o}{%
  \begin{tcolorbox}[
    colback=#1!10!white,
    colframe=#1!80!black,
    boxrule=0.8pt,
    arc=4pt,
    left=6pt,
    right=6pt,
    top=4pt,
    bottom=4pt,
    enhanced,
    breakable,
    overlay={
      \IfValueT{#3}{%
        \node[anchor=north east, font=\scriptsize\bfseries, text=#1!80!black] 
        at (frame.north east) {#3};
      }
    }
  ]
  #2
  \end{tcolorbox}
}
\newcommand{\hlspan}[2]{%
  \tcbox[colback=#1!10!white,
         colframe=#1!80!black,
         on line,
         boxrule=0.6pt,
         arc=3pt,
         boxsep=1pt,
         left=2pt,
         right=2pt,
         enhanced]{#2}
}
\newtcolorbox[auto counter, number within=section]{promptbox}[2][]{%
  colback=gray!5!white, colframe=ctublue,
  title={\hlspan{ctublue}{\thetcbcounter} #2},
  fonttitle=\bfseries,
  enhanced,
  breakable,
  #1
}

\newtcolorbox{figureboxinner}{
  colback=gray!5!white,
  colframe=ctublue,
  fonttitle=\bfseries,
  toprule=4pt,
  leftrule=2pt,
  rightrule=2pt,
  bottomrule=1.5pt,       
  fontupper=\small,  
  rounded corners, arc=8pt,        
  enhanced
}

% Define the figurebox environment
\NewDocumentEnvironment{figurebox}{m m}{%
  \begin{figure}[htbp]
    \centering
    \begin{figureboxinner}
}{%
    \end{figureboxinner}
    \caption{#1}
    \label{#2}
  \end{figure}
}


\definecolor{lightgreen}{HTML}{E2FEEA}
\definecolor{lightred}{HTML}{FDE3E2}


\makeatletter
\patchcmd{\chapter}{\if@openright\cleardoublepage\else\clearpage\fi}{\clearpage}{}{}
\makeatother


\begin{document}

\maketitle
\ctutemplate{specification.as.chapter}
\chapter{Introduction}
\section{Background}
In recent years, Large Language Models (LLMs) have permeated the Natural Language Processing research landscape as well as into the general public. 
Already achieving human-like performance at a wide variety of tasks\cite{bubeck2023sparksartificialgeneralintelligence}, 
they are bound by scaling laws\cite{kaplan2020scalinglawsneurallanguage} which predict performance gained with adding compute, fueling
massive investments into computation capacity by industry players. 

With costs of training new state-of-the-art foundational LLMs rising rapidly, research has turned to inference-time scaling\cite{welleck2024decodingmetagenerationinferencetimealgorithms}, 
based on post-training\cite{openai2024openaio1card}\cite{deepseekai2025deepseekr1incentivizingreasoningcapability} utilizing reinforcement learning and supervised fine-tuning, and 
prompting techniques\cite{schulhoff2024promptreportsystematicsurvey}. 

Another research branch gaining substantial attention recently is compile-time scaling\cite{schnabel2024symbolicpromptprogramsearch} represented by prompt optimization\cite{ramnath2025systematicsurveyautomaticprompt}.
Optimization using LLMs\cite{meyerson2024languagemodelcrossovervariation}\cite{liu2024largelanguagemodelsevolutionary} and particularly prompt optimization\cite{yang2024largelanguagemodelsoptimizers}\cite{zhou2023largelanguagemodelshumanlevel}\cite{he2024crispomultiaspectcritiquesuggestionguidedautomatic} 
presents an exciting intersection between deep learning and traditional optimization algorithms, like evolutionary algorithms\cite{guo2024connectinglargelanguagemodels}\cite{cui2024phaseevounifiedincontextprompt}\cite{fernando2023promptbreederselfreferentialselfimprovementprompt} and other metaheuristics\cite{pan2024plumpromptlearningusing}.



\chapter{Literature}

%\section{Basics of Large Language Models}
In Natural Language Processing (NLP), language models are machine learning models that model statistical dependencies in language.
Specifically Large Language Models (LLMs) are language models with many layers and a large number of parameters, often numbered in billions, trained on enormous amounts of text.

Current state-of-the-art models use the Transformer\cite{vaswani2023attentionneed} architecture and its derivatives. 
In its essence, this architecture chains attention mechanisms with fully connected layers. 
This simple architecture, when applied on massive scale using vast computation and data resources, is at the root of the current LLM revolution in the fields of NLP and artificial intelligence.

LLMs predict the next token in a sequence of tokens. Tokens can represent a letter or a word, but current models utilize sub-word tokens, similar in length to syllables.
The meaning of these tokens can be inferred from relative position to other tokens in training text and encoded into vector form using an algorithm like Word2Vec\cite{mikolov2013efficientestimationwordrepresentations}.

When we give a sequence of tokens as an input to an LLM, it produces a probability distribution of possible next tokens. 
This happens both during training, where the probability distribution is used to update the model's weights to minimize training loss, and during \textit{inference}.
Inference is the process of using a frozen LLM to generate text. 

During inference, a sampling algorithm is employed to pick the next token from the probability distribution.
Usually a interpolation between greedy and uniform sampling is employed. This process is called temperature sampling and introduces an important hyperparameter - \textit{temperature} $\tau$.
Temperature influences the randomness of next token prediction and affects the outputs' qualities, like creativity and novelty. 
For $\tau = 0$, the sampling process is deterministic in theory, but in practice variance remains due to numerical errors.

The input sequence is generally called a \textit{prompt}. The model recursively generates next tokens, forming an output sequence, which is also referred to as a \textit{completion}.
Generation continues until a special \texttt{stop} token is generated or until a token limit is reached.

In LLM training, loss has been empirically found to scale with computation resources used\cite{kaplan2020scalinglawsneurallanguage}. This is referred to as training-time scaling.
With diminishing returns, research has turned to inference-time scaling, characterized by expending more resources during inference, for example by post-training the LLM to begin completions
with a reasoning chain. This brought impressive results particularly in reasoning-heavy domains. The post-training is done using both reinforcement learning and supervised finetuning.
\section{Inference-time scaling}
Inference-time scaling or test-time scaling is a paradigm that has gained traction in the recent years
with the advent of dedicated reasoning models \todo{cite some model cards / deepseek}. 
As opposed to training-time scaling, where the performance of models scales with 
training times, model parameter counts and dataset sizes \todo{cite smth about training scaling},
inference-time scaling aims to improve performance by dedicating more resources to each inference call.

At their heart, LLMs are probabilistic models over sequences and to generate a sequence, they employ generation algorithms. 
Welleck et al.\cite{welleck2024decodingmetagenerationinferencetimealgorithms} provide an overview of these generation algorithms
and then frame more advanced inference-time techniques as meta-generations, or strategies that employ sub-generators.
Most generation algorithms attempt to find either highly probable sequences (MAP algorithms) or sample from the model's distribution.
The simplest MAP algorithm is greedy decoding, which recursively finds the next token with the highest probability in the distribution.

A generalization of greedy decoding is the beam search algorithm which maintains a structure of possible prefixes and each step expands them and scores them.
An example of a beam search algorithm\cite{wang2024chainofthoughtreasoningprompting} can identify decoding branches where the model 
employs a reasoning chain to solve a given task. Authors of this algorithm found that answer tokens found in the decoding paths with a reasoning chains 
have greater token probabilities, meaning the model shows greater confidence in its answer having reasoned about it beforehand.
In general beam search improves on simple greedy decoding but at a high computational cost.

An interpolation between greedy decoding and uniform sampling is temperature sampling, which 
outperforms other adapters in input-output tasks like code generation and translation. 
An example of algorithms that sample from the model's distribution is the ancestral sampling algorithm.
Interpolating between ancestral sampling and simple greedy sampling gave rise to decding algorithms such as
nucleus, top-k and $\eta$- and $\epsilon$-sampling. When we require a structured output, for example a JSON data 
structure following a JSON schema, we can utilize parser-based decoding, which enforce a structural requirement.
This can however come at worsened performance when using inflexible templates.

These strategies can be divided into the categories of chained, parallel, step-level, and refinement-based meta-generators\cite{welleck2024decodingmetagenerationinferencetimealgorithms}.

\subsection{Chained meta-generation}

Chained meta-generation is the composition of several subgenerators in sequence. 
These can be LLM calls or other functions that use previous inputs, such as code execution function \todo{cite program of thought}.
The subgenerators can be implemented as several LLM calls or with a single call given sufficient instructions in the prompt. \cite{khattab2023dspycompilingdeclarativelanguage}
Some examples include Program-of-thought, Plan-and-Solve and Chain-of-Thought techniques.

\subsubsection{Chain-of-thought (CoT)}
Chain-of-Thought (CoT) is a LLM prompting technique that works by inducing a coherent series of intermediate 
reasoning steps that lead to the final answer for a problem\cite{wei2023chainofthoughtpromptingelicitsreasoning}.
Existing work suggest LLMs falter in a direct-QA scenarios (without inducing CoT), where the greedy decoding path mostly does not contain a reasoning chain\cite{wang2024chainofthoughtreasoningprompting}. 
In its essence, the model is a left-to-right text completion engine. We can make the analogy with human thinking 
modes, where it is said that humans have a fast automatic "System 1" mode and a slow and deliberate "System 2" mode\cite{yao2023treethoughtsdeliberateproblem}. 
In direct-QA mode, the LLM can underestimate the difficulty of the task\cite{wang2024chainofthoughtreasoningprompting} and stay in the "System 1" thinking mode.
By crafting a good prompt that instructs the model to reason we can shift the model from "System 1" to "System 2" thinking.
Furthermore, CoT allows models to allocate additional computation to problems with more reasoning steps\cite{wei2023chainofthoughtpromptingelicitsreasoning}
Prystawski et al.\cite{prystawski2023thinkstepstepreasoning} also speculate that direct prediction fails for tasks where
the relevant variables are rarely seen together in training, whereas CoT reasoning can incrementally chain known dependencies. 


CoT can been elicited by prompting techniques - few-shot with steps demonstrations or 
zero-shot with specific instructions\cite{wang2024chainofthoughtreasoningprompting}
First CoT methods\cite{wei2023chainofthoughtpromptingelicitsreasoning} involved one/few-shot prompting, 
where the prompt included examples of CoT reasoning in the prompt in facilitate a reasoning chain response.
Although effective, this requires human engineering of multi-step reasoning prompts.
This method is also highly sensitive to prompt design with performance deteriorating 
for mismatched prompt example and task question types\cite{NEURIPS2022_8bb0d291}.
For this method, authors found that CoT is an emergent capability of model scale 
and did not observe benefits for small models\cite{wei2023chainofthoughtpromptingelicitsreasoning}.

On the other hand, zero-shot prompting can induce a reasoning chain with a simple prompt like "Let's think step-by-step",
making it versatile and task-agnostic\cite{NEURIPS2022_8bb0d291}. Similar prompts also improve reasoning performance and 
some research\todo{tady OPRO? nebo kde hledali cot prefixy} has been been done on finding the optimal CoT prefix prompt.

Apart from prompting, CoT can been elicited by model training or tuning. 
This method, requiring a significant amount of reasoning data\cite{wang2024chainofthoughtreasoningprompting},
has gained traction with the development of dedicated reasoning models like OpenAI's o1 or Deepseek-R1\todo{cite o1 deepseek}.
Using methods such as supervised fine-tuning (SFT) or reinforcement learning (RL), the model is trained to
automatically produce longer reasoning chains, often bound in dedicated "thought" tags or tokens. These models have shown significant performance boosts on reasoning benchmarks \todo{cite}.
Models similar to o1 all primarily extend solution length by self-revision\cite{zeng2025revisitingtesttimescalingo1like}
After finishing a thought process, the model tries to self-revise, which is marked by words such as "Wait" or "Alternatively". 
The model then tries to spot mistakes or inconsistencies in its reasoning or propose an alternative solution. 
Self-revision ability is thus a key factor in the effectiveness of sequential scaling for reasoning models. \cite{zeng2025revisitingtesttimescalingo1like}

Prompting techniques like chain-of-thought can increase answer quality 
at the cost of longer and more computationally expensive outputs. \cite{brown2024largelanguagemonkeysscaling}
Performance gains are observed mainly on arithmetic and coding tasks with more performance gains 
being observed for more complicated problems\cite{wei2023chainofthoughtpromptingelicitsreasoning}.
Further research by Liu et al.\cite{liu2024mindstepbystep} suggests that for some tasks CoT can be detrimental.
Their experiments proved their hypothesis that CoT hurts performance on tasks where humans do better without deliberation
and where the nature of LLM, like the much greater context memory, does not provide an advantage over human thinking. 
This phenomenon was observed on tasks like facial recognition, implicit statistical learning or pattern recognition.
Limited performance gains were noticed on commonsense reasoning tasks\cite{NEURIPS2022_8bb0d291}.

Longer reasoning chains mean more computing power spent at inference. How far can we take this sequential scaling?
In their study, Zeng at al.\cite{zeng2025revisitingtesttimescalingo1like} argue that longer CoTs do not consistently improve accuracy of reasoning models.
Furthermore, they find that the average length of correct solutions is shorter than that of incorrect ones. 
Because self-revision accounts for most of the CoT length, the effectiveness of the method relies on the model's ability to self-revise.
Authors of this paper argue that the self-revision ability of models is insufficient as they demonstrate limited capacity to correct their answers
during self-revision. Some models on some tasks are even more likely to change a correct answer to an incorrect one than vice-versa.

\subsection{Parallel meta-generation}

Parallel meta-generation involves multiple generations concurrently. 
The final answer can then be chosen - with a reward model or with 
voting - or constructed from the ensemble of generations \cite{welleck2024decodingmetagenerationinferencetimealgorithms}.

One of the simplest such techniques is self-consistency\cite{wang2023selfconsistencyimproveschainthought} (SC),
a method which builds upon CoT to aggregate answers from diverse reasoning 
chains and selects the best one based on majority voting. 
It significantly improves accuracy in a range of arithmetic and commonsense reasoning tasks \cite{wang2023selfconsistencyimproveschainthought}.
The effectiveness of SC comes from the fact that, for tasks with objective answers, there are more ways to be right than wrong.
For our next discussion of SC and related methods we will compare the 
terms \textit{coverage} $\mathrm{C}_{\mathbb{D}}$ and \textit{accuracy} $\mathrm{A}_{\mathbb{D}}$ for a dataset ${\mathbb{D}}$.
Given a language model $\mathcal{M}$, a task query $q_k \in {\mathbb{D}}$ and a task 
instruction $\mathbf{i}$, we can define the generation collection of length $n$ as
\begin{equation}
    Y_k = \{y_{jk}\mid j \in 1, ..., n\},
\end{equation}
\begin{equation}
    y_{jk} \sim \mathcal{M}(\mathbf{i}(q_k)).
\end{equation}
For objective tasks we can check the correctness with a metric $\mathcal{G}$
\begin{equation}
    \mathcal{G}_{k}(y_{jk}, q_k) = 
    \begin{cases}
        1.0 & y_{jk} \text{ is the correct answer for } q_k\\
        0.0 & y_{jk} \text{ is an incorrect answer for } q_k.
    \end{cases}
\end{equation}
To choose the final answer, we will define a answer selection function $\mathcal{S}(Y)$. 
This can be a majority vote selection function or some reward-based method.
We can now define \textit{coverage} $\mathrm{C}_{\mathbb{D}}$ and \textit{accuracy} $\mathrm{A}_{\mathbb{D}}$ as
\begin{align}
    \mathrm{C}_{\mathbb{D}} &= \frac{1}{|\mathbb{D}|} \sum_{q_k \in \mathbb{D}} \max_{j=1,...,n} \mathcal{G}_k(y_{jk}, q_k) \\
    \mathrm{A}_{\mathbb{D}} &= \frac{1}{|\mathbb{D}|} \sum_{q_k \in \mathbb{D}} \mathcal{G}_k\left( \mathcal{S}(Y_k), q_k \right).
\end{align}



Works with both few-shot and zero-shot CoT. \cite{wang2023selfconsistencyimproveschainthought}
One limitation of self-consistency is that it incurs more computation cost. 
Small number of paths is enough, as in most cases the performance saturates quickly. \cite{wang2023selfconsistencyimproveschainthought}




Generating large sample collections is only useful if the correct samples in a collection can be identified. \cite{brown2024largelanguagemonkeysscaling}

It is possible and sometimes cost-effective, to amplify weaker models with many samples and outperform 
single samples from more capable models. \cite{brown2024largelanguagemonkeysscaling}

Relationship between coverage and the number of samples can often be modeled using an exponentiated power law, suggesting a form of scaling laws for inference-time compute although not as exact as training scaling laws. \cite{brown2024largelanguagemonkeysscaling}

Repeated sampling can make use of high batch sizes and specialized optimizations that improve 
system throughput relative to single-attempt inference workloads. \cite{brown2024largelanguagemonkeysscaling}

Coverage and precision diverge as number of samples increases on tasks without automatic verifiers, 
highlighting the need for improving sample verification methods. \cite{brown2024largelanguagemonkeysscaling}


All parallel scaling methods rely on guidance signals to select the optimal token, step, or solution from a set of candidates. \cite{zeng2025revisitingtesttimescalingo1like}

For the same number of generated tokens, parallel scaling provides a significantly larger improvement in coverage compared to sequential scaling. \cite{zeng2025revisitingtesttimescalingo1like}

Practical parallel scaling method must select a final answer from a set of candidate answers. \cite{zeng2025revisitingtesttimescalingo1like}

Shortest majority vote takes into account the fact that correct solution chains are shorter on average and extends Majority vote by weighing solution counts with the log of average solution length for given solution category, outperforming Majority voting on AIME. \cite{zeng2025revisitingtesttimescalingo1like}

\subsection{Step-level meta-generation}
Step-level meta-generation implements search algorithms on the generation state-space, which can be made up of tokens or longer sequences. Many search algorithms and state evaluation functions are possible. \cite{welleck2024decodingmetagenerationinferencetimealgorithms}


Multi-turn inference-time methods with a fixed width exhibit diminishing gains when computational budget is increased, failing to leverage the vast output space of LLMs.\cite{misaki2025widerdeeperscalingllm} 

Unlike the standard tasks typically tackled by tree search algorithms where the number of possible actions at each node is finite, each call to LLM can yield a new output even for the same input, making each node’s branching factor theoretically infinite.\cite{misaki2025widerdeeperscalingllm} 


\subsubsection{Tree-of-thought}
Similar to CoT with self-consistency but the reasoning chain is split up into steps creating a tree. This reasoning tree can then be explored using a graph search algorithm such as DFS.


\subsection{Refinement meta-generation}
Refinement meta-generation generates a revised version of the output based on past versions and additional information such as intrinsic or extrinsic feedback or environment observations. \cite{welleck2024decodingmetagenerationinferencetimealgorithms}
For extrinsic refinement, it is plausible that there are information sources which add new information, and hence lead to a potential gain with refinement but the efficacy of intrinsic refinement has been mixed. \cite{welleck2024decodingmetagenerationinferencetimealgorithms}


Inference-Time Scaling by training models to think before responding is insufficient because these methods include Reinforcement learning with verifiers, making them unsuitable for open-ended tasks. \cite{wang2025dedicatedfeedbackeditmodels}

Authors train dedicated Feedback and Edit models that can be used at inference time to improve model responses to open-ended general domain tasks. \cite{wang2025dedicatedfeedbackeditmodels}

Efficacy of LLMs in providing feedback and making edits to their own responses is unclear as using LLMs that were not specifically trained to provide feedback is ineffective compared to using high quality feedback. \cite{wang2025dedicatedfeedbackeditmodels}


\subsubsection{Reflexion}
Reflexion converts binary or scalar feedback from the environment into verbal feedback in the form 
of a textual summary, which is then added as additional context for the LLM agent, e.g. CoT or ReAct module, in the next episode. \cite{shinn2023reflexionlanguageagentsverbal}

The model can go through several steps of using the tools, which generates "observation". The model uses these observations to generate a final answer and leaves the ReAct chain when ready using a "finish" function.
Reflections go into a long-term memory context limited to a sliding window with maximum capacity. \cite{shinn2023reflexionlanguageagentsverbal}

Improves performance over strong baselines on sequential decision making, reasoning and programming tasks. \cite{shinn2023reflexionlanguageagentsverbal}
\subsubsection{ReAct}
Multi-turn prompting technique that forms the basis of agentic LLMs. The model is given a set of tools, such as a Wikipedia search function or a math expression evaluator.


\section{Prompting techniques}

Prompting techniques encode human priors, making it difficult to assess a language model's intrinsic reasoning abilities \cite{wang2024chainofthoughtreasoningprompting}

\subsection{Prompt engineering}
In many modern LLM applications, prompts have become programs themselves. \cite{schnabel2024symbolicpromptprogramsearch}
Motivation for prompt engineering is to improve the model's capabilities not by changing the underlying weights with training on data but by crafting an optimal instruction string, or a prompt.
This can be done by providing examples of the task as a part of the prompt or by instructing the model how to solve the task.

\subsubsection{In-context learning}
Prompts are distinguished based on the number of included examples.
\begin{table}[h!]
    \centering
    \begin{tabular}{|c|p{12cm}|}
    \hline
    \textbf{Prompting Type} & \textbf{Description} \\
    \hline
    Zero-shot Prompting & Prompt has no examples, model relies on its pre-trained knowledge. \\
    \hline
    One-shot Prompting & Prompt has one example to guide the model. \\
    \hline
    Few-shot Prompting & Prompt includes a few examples (typically 2 to 5). \\
    \hline
    \end{tabular}
    \caption{Comparison of Zero-shot, One-shot, and Few-shot Prompting}
\end{table}        
Research\cite{brown2020languagemodelsfewshotlearners} has shown that with growing model size the knowledge-generalizing ability of the model increases. Instead of expensive fine-tuning
models can reuse knowledge from pre-training and solve many tasks when provided just by a few examples.

Few-shot prompting highlights that LLMs can be seen as powerful pattern-completion engines. \cite{meyerson2024languagemodelcrossovervariation}

Providing a prompt of examples from a distribution can condition the LLM to generate further high-probability examples from that distribution \cite{meyerson2024languagemodelcrossovervariation}

\subsection{Prompting techniques}
\todo{talk about how we can achieve meta-generation just by updating the prompt}

\section{Prompt engineering}\label{sec:preng}
By prompt engineering we mean crafting a suitable instruction which, when combined with a query, incites a LLM response that matches our requirements.
Our task requirements can for example be: a) obtaining the correct answer for a mathematical problem, b) fixing a bug in a code base, or c) explaining the contents of an image.
Each of these tasks needs a separate instructional prompt $P$ which can then be used with multiple queries, representing specific task instances. 

\begin{figure}[htbp]
    \centering
    % Subfigure 1
    \begin{subfigure}[b]{0.45\textwidth}
        \centering
        \begin{tikzpicture}[
        node distance=0.3cm and 0.3cm,
        every node/.style={font=\sffamily\scriptsize},
        textnode/.style={draw, rectangle, rounded corners=5pt, fill=ctulightblue, text=ctublue, minimum width=1cm, minimum height=1cm, align=center},
        funcnode/.style={draw, circle, fill=ctuorange, text=white, minimum size=0.7cm, align=center},
        arrow/.style={-Stealth, thick}
    ]
    \node[textnode] (taska) {Task\\A};
    \node[textnode,below=of taska] (taskb) {Task\\B};
    \node[textnode,below=of taskb] (taskc) {Task\\C};

    \node[funcnode, right=of taska] (llma) {LLM\\A};
    \node[funcnode, right=of taskb] (llmb) {LLM\\B};
    \node[funcnode, right=of taskc] (llmc) {LLM\\C};

    \node[textnode, right=of llma] (outa) {Output\\A};
    \node[textnode, right=of llmb] (outb) {Output\\B};
    \node[textnode, right=of llmc] (outc) {Output\\C};

    \draw[arrow] (taska) -- (llma);
    \draw[arrow] (taskb) -- (llmb);
    \draw[arrow] (taskc) -- (llmc);

    \draw[arrow] (llma) -- (outa);
    \draw[arrow] (llmb) -- (outb);
    \draw[arrow] (llmc) -- (outc);
    \end{tikzpicture}

        \caption{Pre-train + Fine-tine}
        \label{fig:finetune}
    \end{subfigure}
    \hfill
    % Subfigure 2
    \begin{subfigure}[b]{0.45\textwidth}
        \centering
        \begin{tikzpicture}[
        node distance=0.3cm and 0.3cm,
        every node/.style={font=\sffamily\scriptsize},
        textnode/.style={draw, rectangle, rounded corners=5pt, fill=ctulightblue, text=ctublue, minimum width=1cm, minimum height=1cm, align=center},
        funcnode/.style={draw, circle, fill=ctuorange, text=white, minimum size=1cm, align=center},
        arrow/.style={-Stealth, thick}
    ]
    \node[textnode] (taska) {Task\\A};
    \node[textnode,below=of taska] (taskb) {Task\\B};
    \node[textnode,below=of taskb] (taskc) {Task\\C};
    
    \node[textnode, right=of taska] (prompta) {Prompt\\A};
    \node[textnode, right=of taskb] (promptb) {Prompt\\B};
    \node[textnode, right=of taskc] (promptc) {Prompt\\C};

    \node[funcnode, right=1.8cm of taskb] (llm) {LLM};

    \node[textnode, right=3.1cm of taska] (outa) {Output\\A};
    \node[textnode, right=3.1cm of taskb] (outb) {Output\\B};
    \node[textnode, right=3.1cm of taskc] (outc) {Output\\C};

    \node[font=\large] at ($(taska)!0.5!(prompta)+(-0.05,0)$) {$+$};
    \node[font=\large] at ($(taskb)!0.5!(promptb)+(-0.05,0)$) {$+$};
    \node[font=\large] at ($(taskc)!0.5!(promptc)+(-0.05,0)$) {$+$};

    \draw[arrow] (prompta) -- (llm);
    \draw[arrow] (promptb) -- (llm);
    \draw[arrow] (promptc) -- (llm);

    \draw[arrow] (llm) -- (outa);
    \draw[arrow] (llm) -- (outb);
    \draw[arrow] (llm) -- (outc);
    \end{tikzpicture}
        \caption{Train + Prompt}
        \label{fig:trainprompt}
    \end{subfigure}
    \caption{Comparison between pre-train+fine-tune and train+prompt paradigms.}
    \label{fig:prompttraincomp}
\end{figure}
This signifies a shift from the training and fine-tuning paradigm, where 
a base model is first trained on a large corpus of data and then adapted for a specific task with supervised fine-tuning. 
This process requires a substantial amount of training data and computation power, making specialized LLMs unsuitable
for many users and for applications, where extensive data collection is infeasible.

We illustrate this shift in figure \ref{fig:prompttraincomp}. Although prompt design is difficult, it is much less resource-intensive than fine-tuning.

Since the inception of modern LLMs, prompt engineering has evolved into a field of its own. Current LLM systems, often containing
multiple chained and interlinked models, require robust and well thought-out prompts at each step. 
Indeed, in many modern LLM applications, prompts have become programs themselves\cite{schnabel2024symbolicpromptprogramsearch}.

In this section, we will briefly cover the most notable prompt engineering techniques, which we will then
be able to utilize in our study of automatic prompt optimization.
\subsection{Components of a prompt}
We can roughly dissect a prompt into 4 different components\cite{schulhoff2024promptreportsystematicsurvey}. 
\begin{itemize}
    \item \textbf{Directive:} The main task of the prompt, e.g., \textit{"Write an email to a coworker."}
    \item \textbf{Context:} Everything necessary or beneficial to completing the directive, e.g., \textit{"I was supposed to send a report to my boss, but I forgot."}
    \item \textbf{Examples:} How you would have solved a similar task, e.g. a past email on a similar topic.
    \item \textbf{Output specifications:} Style and format instructions, e.g., \textit{"Respond with three paragraphs in formal style with tasteful emojis."}
\end{itemize}

Although flexible and sometimes blended together, prompts often follow this structure and order. 
In more technical applications where prompts might become more convoluted, it might be beneficial to use tags or delimiters to explicitly separate components.
Models often reward prompts with a more code-like structure\cite{10.1145/3544548.3581388}. We now discuss each component in detail.

\subsubsection{Directive}
The directive should be a clear and objective description of the task, assuming that the model
already knows how to solve it\cite{reynolds2021promptprogramminglargelanguage}. 
Specific requirements that narrow the scope should be avoided and left for other components.
\begin{figurebox}{Directive with a placeholder}{box:limerickexample}
    Write a limerick about \texttt{$\{$topic$\}$}.
\end{figurebox}

In cases where the prompt serves as a prompt template, meaning it can be reused with various data points,
the directive can include a placeholder. For example consider the directive in figure \ref{box:limerickexample}, 
which could be reused for multiple values of \texttt{topic}.

\subsubsection{Context}
Context should provide all the background information relevant to the task at hand.
The user can include more information about barriers which prevent them from solving the problem on their own,
define the target audience or attach relevant documents. 

In figure \ref{box:contextprompt}, we show a 
prompt which asks the model to summarize an article, with the context component providing further specifications.

\begin{figurebox}{Using context to personalize output}{box:contextprompt}
    \textbf{Directive:}
    
    \hlbox{ctublue}{
    Summarize this article on climate change for a high-school debate. \\
    \texttt{$\{$article$\}$}
    }
    
    \textbf{Context:}
    
    \hlbox{ctublue}{
    I already understand the basic causes of climate change, but I struggle with the economic side. Focus on how it affects economies and give arguments I could use in a policy debate.
    }
\end{figurebox}

In this example, the initial instruction serves as the directive, while the second component provides contextual information about the user's prior knowledge and objectives.
Without this context, the model may produce a generic summary, overlooking the user's interest in economic impacts and failing to discuss arguments relevant for a policy-oriented debate.

The context can become the endpoint for retrieval pipelines, which search a data source for 
relevant documents, or memory mechanisms, which gather personal information about the user from other conversations. 

Another possible feature often used in the context component is a memetic proxy\cite{reynolds2021promptprogramminglargelanguage},
like role-assignment. Instead of writing a long instruction covering all the requirements and assumptions 
behind someone being an "experienced business analyst", we can just say "You are an experienced business analyst".
In this way, we can instruct the model to adopt an identity or an expertise level.
This primes the model to consistently use more technical language in its response.

\subsubsection{Examples}\label{sec:icl}
By providing examples of solutions to similar tasks, we can condition the LLM to generate
further examples from that distribution\cite{meyerson2024languagemodelcrossovervariation}, increasing the 
chance of a suitable completion. Furthermore, examples make the decoding more robust and decrease prompt sensitivity\cite{zhuo2024prosaassessingunderstandingprompt}.

\begin{table}[ht!]
    \centering
    
    \begin{tabular}{ c p{8cm} }
        \hline
        \textbf{Prompting Type} & \textbf{Description} \\
        \hline
        Zero-shot Prompting & Prompt has no examples. Model relies on instructions and pre-trained knowledge. \\
        
        One-shot Prompting & Prompt has one example to guide the model. \\
        
        Few-shot Prompting & Prompt includes a few examples. \\
        \hline
    \end{tabular}
    \caption{Comparison of Zero-shot, One-shot, and Few-shot Prompting}\label{tab:nshot}
\end{table}        

Table \ref{tab:nshot} shows the agreed-upon terminology for prompts with examples. 
The concept of adding examples to the prompt is also called In-Context Learning (ICL).
In some cases, Few-shot prompting can be effective even without the use of other instructions\cite{brown2020languagemodelsfewshotlearners}.
When adding examples to a prompt, we have to pay attention to several aspects\cite{schulhoff2024promptreportsystematicsurvey}.
\begin{itemize}
    \item \textbf{Exemplar quantity}: More is better with diminishing returns.
    \item \textbf{Exemplar ordering}: Models tend to pay more attention to the last examples.
    \item \textbf{Exemplar label distribution}: Unbalanced labels in examples skew model generation.
    \item \textbf{Exemplar label quality}: It is unclear whether incorrect examples hurt performance.
    \item \textbf{Exemplar format}: Optimal format may vary across tasks.
    \item \textbf{Exemplar similarity}: Effect of exemplar similarity depends on the situation.
\end{itemize}

Contrary to Brown et al.\cite{brown2020languagemodelsfewshotlearners} who interpret the effectiveness of
few-shot prompting as the model learning the task by observing examples, later research\cite{reynolds2021promptprogramminglargelanguage}
suggests that examples merely allow the LLM to more precisely locate the task in its learned task space.
Still, the authors argued for the use of examples for redundancy enforcing the desired behavior\cite{reynolds2021promptprogramminglargelanguage}.

\subsubsection{Output specifications}
Using this component, we guide the structure, tone, style and formatting of the LLM's answer. 
A common technique, discussed in \ref{sec:cot} is inducing reasoning
with a CoT prompt, like "Let's think step-by-step", or instructing the LLM to 
first plan the solution and then execute it. 

We can influence the length of the answer, ask the model to be formal or humorous,
request specific formatting, like the use of \LaTeX{} equations, or have it answer in a JSON format
for a machine-readable response. 

With innumerable options for output specification, users should test multiple configurations as changing the output format
can heavily influence the final prediction accuracy\cite{salinas2024butterflyeffectalteringprompts}.

\subsection{\textit{Meta-prompting}}
Before we proceed, we first need to overview the terminology discrepencies in contemporary research. 
\textit{Meta-prompts} (also meta prompts or metaprompts) were first coined as a term by Reynolds and McDonell\cite{reynolds2021promptprogramminglargelanguage}. 
Since then, this term was used in multiple contexts.
\begin{enumerate}
    \item \textbf{Task-agnostic zero-shot prompt:} In contrast to task-sample specific few-shot prompting, 
    \textit{meta-prompts} were used to mean a ``task-agnostic zero-shot prompt'', or ``seeds encapsulating a more general intention that will unfold into
    a specific prompt when combined with the task question''\cite{reynolds2021promptprogramminglargelanguage}. 
    \item \textbf{Natural language metaprocedure:} Extending 1, Zhang et al.\cite{zhang2025metapromptingaisystems} define \textit{meta-prompts} as example-agnostic prompts
    designed to capture the reasoning structure of a specific category of tasks. They do this by employing a typed and structured prompt, 
    resembling control flow templates and often expressed in JSON-like structures.
    This is similar to how DSPy\cite{khattab2023dspycompilingdeclarativelanguage} implements LLM calls with function signatures.
    \item \textbf{Soft prompt optimization method:} In a parallel branch of research, MetaPrompting\cite{hou2023metapromptinglearninglearnbetter} 
    was used as a name of a soft token prompt optimization method. Soft token prompts are, in contrast to human-readable textual discrete token prompts, 
    raw floating-point vector inputs to the first layer of a language model. This makes soft token prompt optimization amenable to gradient optimization.
    \item \textbf{Prompt generator:} In automatic prompt engineering and prompt optimization literature, \textit{meta-prompt} is a prompt that generates prompts\cite{dewynter2024metaprompting}.
    This can be seen as an application of 2 to the task of prompt generation\cite{zhang2025metapromptingaisystems}.
\end{enumerate}

While both 1 and 2 are of interest to us, in this thesis, we will treat \textit{meta-prompts} as prompt generators. 

\subsubsection{\textit{Meta-prompt} as a prompt generator}
Prompt optimization literature\cite{ramnath2025systematicsurveyautomaticprompt} treats \textit{meta-prompts} simply as a prompt which generates other prompts.
As prompt generation is a complex reasoning-intensive language generation task\cite{ye2024promptengineeringpromptengineer}, 
all the principles regarding prompt design for other difficult tasks apply to \textit{meta-prompts} as well.

\textit{Meta-prompts} are usually templates for other data, such as examples of the task for Instruction Induction\cite{honovich2022instructioninductionexamplesnatural},
past prompt generations along with their scores\cite{yang2024largelanguagemodelsoptimizers} and critiques of prompt outputs\cite{he2024crispomultiaspectcritiquesuggestionguidedautomatic}
or text descriptions of the task\cite{ye2024promptengineeringpromptengineer}. Some reseach also utilizes professional task advice\cite{ramnath2025systematicsurveyautomaticprompt}
or a prompt engineering tutorial\cite{ye2024promptengineeringpromptengineer}. Other methods used seed phrases, like a thinking 
style\cite{fernando2023promptbreederselfreferentialselfimprovementprompt}, to steer the generation.

In one of the few theoretically rigorous works on the subject, de Wynter et al.\cite{dewynter2024metaprompting} formalize meta-prompts
using category theory. Using this formalism, they show the possibility of creating a general purpose meta-prompt as well as suggesting that such meta-prompt
will perform better than task-specific prompts in a wide range of applications.

\section{Prompt optimization}
\subsection{Soft prompt tuning}
Prompts for models which allow access to gradients, which is not the case for proprietary models accessed via APIs, can be optimized in the high-dimensional embedding space.
This makes the optimization problem continous. Soft prompts however pose the problem of interpretability and are non-transferable across different LLMs \cite{deng2022rlpromptoptimizingdiscretetext}.
\subsection{Discrete prompt tuning}
The area of optimizing prompts discretely while utilizing language models as optimization operators has attracted significant research interest in recent years.
\subsubsection{Textual gradients}
Naturally there are no gradients in the text space but some researchers try to emulate them using reflection-based operators.
\subsubsection{Evolutionary optimization}
Building upon the inherent ability of LLMs to paraphrase (mutation) and combine (crossover) text, an interesting intersection of traditional evolutionary algorithms and modern LLMs has formed. 
\subsubsection{Metaprompting}
Metaprompting or "prompting to create prompts". Research shows that meta-prompting will always be superior to prompting through category theory\cite{dewynter2024metaprompting}.

\chapter{Implementation}
\section{Problem Formulation}\label{sec:notation}
Before we describe our approach and discuss implementation in detail, we will first establish mathematical notation and use it to 
formulate the prompt optimization task.

Let $\mathcal{T}$ be the space of character sequences. Then we will define an LLM as a stochastic mapping
\begin{equation}
    \mathscr{M}: \mathcal{T} \rightarrow \mathcal{L}(\mathcal{T}),
\end{equation}
where $\mathcal{L}(\mathcal{T})$ is a probabilistic language distribution learned during the LLM's training.
This distribution is governed by the LLM's hyperparameters $H$, which affect its behavior. 

Of particular interest is the sampling temperature $\tau \in H$ which interpolates between greedy decoding and uniform sampling.
In theory, $\mathscr{M}$ is deterministic for $\tau=0$, but in practice numerical errors still introduce variance.

We use the lower index to specify the purpose of the LLM instance. This also highlights
the fact that each instance $\mathscr{M}$ can use different hyperparameters and underlying models. 
We will use this to differentiate between $\mathscr{M}_{\text{solve}}$ and $\mathscr{M}_{\text{optim}}$.
The optimizer model $\mathscr{M}_{\text{optim}}$ will be utilized to generate and improve prompts which will be evaluated on $\mathscr{M}_{\text{solve}}$.


We consider a prompt $P \in \mathcal{I}$, where $\mathcal{I} \subseteq \mathcal{T}$ is the prompt space.
When such prompt is used as an input into the LLM, it produces a completion
\begin{equation}
    y \sim \mathscr{M}(P)
\end{equation}
and $y \in \mathcal{U}$, where
$\mathcal{U} \subseteq \mathcal{T}$ is the completion space.
In contexts where $P$ serves as a template for an additional query $q$, we will write
\begin{equation}
    y \sim \mathscr{M}(P\vsep q),
\end{equation}
where $P \vsep q$ denotes the result of inserting query $q$ into a designated placeholder in $P$.

Let $\mathscr{P} \subseteq \mathcal{I}$ be a population of prompts. Then for a query $q$ we define the set of completions 
\begin{equation}
    \mathcal{C}_{q}^{\mathscr{P}} = \{\mathscr{M}_{\text{solve}}(P\vsep q)\vsep P \in \mathscr{P}\}.
\end{equation}
For convenience, we might omit the explicit population superscript and write just $ \mathcal{C}_{q}$.

We can use LLMs to solve a general task
\begin{equation}
    t = (q, g) \in \mathcal{D},
\end{equation}
where $\mathcal{D} \subseteq \mathcal{Q} \times \mathcal{G}$ is a dataset of query-answer pairs, $\mathcal{Q}\subseteq\mathcal{T}$ is the set of queries $q$
and $\mathcal{G}\subseteq\mathcal{U}$ is the set of gold labels $g$.

We consider each dataset to have one or more assigned evaluation metrics $\mathscr{F}_{\mathcal{D}}^{\text{supervised}}: \mathcal{U} \times \mathcal{G} \rightarrow \mathbb{R}$,
which scores the LLM output using the corresponding gold label. 
Extending the set of completions for queries from the entire dataset, we define
\begin{equation}
    \mathcal{C}_{\mathcal{D}} = \{\mathcal{C}_{q}\vsep q \in \mathcal{D}\}
\end{equation}

For open-ended tasks, the gold label does not exist, $G = \varnothing$. To achieve effective evaluation even for such tasks, 
we formulate a metric based on pairwise comparisons and define
\begin{equation}
    \mathscr{F}_{\mathcal{D}}^{\text{pairwise}}: \mathcal{Q} \times 2^\mathcal{U} \rightarrow \mathbb{R},
\end{equation}
which maps a query and a set of outputs to a real number. Here we use the power set notation $2^\mathcal{U}$ to signify the set of all possible pairs of completions $y \in \mathcal{U}$.

To generalize, we define $\mathscr{F}_{\mathcal{D}}$ which combines both $\mathscr{F}_{\mathcal{D}}^{\text{pairwise}}$ and $\mathscr{F}_{\mathcal{D}}^{\text{supervised}}$
Using this, we define the mean performance $\mathcal{E}$ of a prompt $P$ on a dataset $\mathcal{D}$ as
\begin{equation}
    \mathcal{E}_{\mathcal{D}}(P) = \frac{1}{\Vert \mathcal{D} \Vert}\underset{t=(q,g)\in \mathcal{D}}{\sum}\mathscr{F}(\mathscr{M}(P\vsep q), t, \mathcal{C}_{q}).
\end{equation}

For convenience, we will define the scores of the population $\mathscr{P}$ on dataset $\mathcal{D}$ as
\begin{equation}
    \mathcal{E}_{\mathcal{D}}(\mathscr{P}) = \{\mathcal{E}_{\mathcal{D}}(P)\vsep P\in \mathscr{P}\} = \{\mathscr{F}(y, g, \mathcal{C}_{\mathcal{D}})\vsep y \in \mathcal{C}_{\mathcal{D}}\}.
\end{equation}

We can then formally define the problem of prompt optimization for a task dataset $\mathcal{D}$ as finding the optimal prompt 
\begin{equation}
    \label{eq:optimdef}
    P^{\star} = \underset{P\in\mathcal{I}}{\operatorname{argmax}}\,\mathbb{E}_{(q, g) \sim \mathcal{D}}\left[\mathscr{F}(\mathscr{M}_{\text{solve}}(P \vsep q), g, \mathcal{C}_{q}^{\mathscr{P}})\right].
\end{equation}

In Algorithm \ref{alg:genoptimloop} we can see the general outline of a population-based optimization method.
The initialization operator $\mathscr{O}_I$ creates an initial population of individuals $\mathscr{P}$. 
In each iteration, a selection operator $\mathscr{O}_S$ first selects a portion of the population according to some criteria. 
These selected individuals are then used by the expansion operator $\mathscr{O}_E$ to create new individuals.
This process continues until a termination condition $\Phi_{stop}$ is reached.

\begin{algorithm}
    \caption{General optimization loop}
    \label{alg:genoptimloop}
    \KwIn{Initialization Operator $\mathscr{O}_I$, Selection Operator $\mathscr{O}_S$, Expansion Operator $\mathscr{O}_E$, Termination Condition $\Phi_{stop}$}
    \KwOut{Optimized Population $\mathscr{P}$}
    \KwData{$\mathscr{P} \gets \mathscr{O}_I$} \tcp{Initialize the population}
    \While{$\neg \Phi_{stop}(\mathscr{P})$}{
        \tcp{Selection and Expansion Steps}
        $\mathscr{P}_{\text{selected}} \gets \mathscr{O}_S(\mathscr{P})$ \\ 
        $\mathscr{P}_{\text{expanded}} \gets \mathscr{O}_E(\mathscr{P}_{\text{selected}})$ \\
        $\mathscr{P} \gets \mathscr{P}_{\text{expanded}}$ \tcp{Update the population} 
        }
        \Return{$\mathscr{P}$} \tcp{Return the optimized population}
    \end{algorithm}
    
We apply this technique to the problem of prompt optimization by defining the aforementioned operators.
Of particular interest are the initialization operator $\mathscr{O}_I$ and the expansion operator $\mathscr{O}_E$, which
both need to produce new prompts. For this purpose, we utilize an LLM instance $\mathscr{M}_{\text{optim}}$ and leverage its 
text generation and reasoning capability.

Let $M\in\mathcal{I}$ be a \textit{Meta-prompt}, or a prompt-generation prompt. We can now formulate generating a prompt 
\begin{equation}
    \label{eq:metaprompting}
    P = \mathscr{M}_{\text{optim}}(M\vsep \mathcal{R})),
\end{equation}
where $\mathcal{R} = \mathcal{R}(\mathscr{P}, \mathcal{C}, \mathcal{D}, \mathcal{E}_{\mathcal{D}}(\mathscr{P}))$ is a retrieval function that selects data
from the current population, past generations, dataset samples and population scores. 
By changing the \textit{Meta-prompt} and the retrieval function $\mathcal{R}$, a variety of possible operators $\mathscr{O}_I$ and $\mathscr{O}_E$
can be defined, thus shaping the prompt optimization process.  

\section{Inference framework}
Taking inspiration from DSPy\cite{khattab2023dspycompilingdeclarativelanguage}, we first implement a simple LLM-calling framework 
capable of invoking several selected inference strategies. Motivations for this are twofold:
\begin{enumerate}
    \item DSPy is a young and ambitious project aiming at simplifying LLM pipeline design and optimization. 
    As we focus on single-stage prompt program optimization, this capability is not useful for our work. 
    Furthermore, due to the framework's infancy, it lacks proper documentation and sometimes exhibits unexpected behavior.
    \item Implementing the prompting techniques discussed in \ref{sec:inference} provides further insight into their workings and performance.
\end{enumerate}

\subsection{Structured generation}
Following current research trends\cite{zhang2025metapromptingaisystems}, we build our inference framework around a structured JSON template,
or a \texttt{Signature}. The \texttt{Signature} structure consists of input and output fields and additional instructions. 
These fields are populated by a \texttt{Field} data structure.
Of particular interest are the output fields, which hold the output name, desired type and optional description. 

Interactions with LLMs in a structured format benefit from better predictability. By implementing the \texttt{Signature} structure,
we can use LLMs as we would a function in any programming language. Functions in programming languages also have functions signatures which
specify input and output names and types.

When employing good naming practices the model can often deduce the task only by looking at output names and types.
Consider the simple \texttt{Signature} in figure \ref{box:simplesig}, which implicitly instructs the LLM to return a word with a meaning opposite to the provided input.

\begin{figurebox}{Simple Signature}{box:simplesig}
    \hlbox{ctuorange}{Word: \texttt{str}}  
    \hlbox{brown}{Antonym: \texttt{str}}
\end{figurebox}

For more complex tasks, filling the output descriptions or even adding explicit instructions is necessary.
In figure \ref{box:complexsig}, notice that it is possible to specify multiple inputs and outputs, which are then generated in the order given.

\begin{figurebox}{Complex Signature. }{box:complexsig}
    \hlbox{ctuorange}{Text: \texttt{str} (Student text)\\
    Grading guide: \texttt{str} (Steps to follow during evaluation)} 
    \hlbox{brown}{Evaluation: \texttt{str} (Textual feedback)\\
     Grade: \texttt{int} (Numerical grade 1-10)} 
    \hlbox{ctublue}{Grade the text.\\ You are an expert text evaluator. \\
    Use the grading guide to evaluate the test and give a final grade. 
    Use formal language and Markdown formatting in the evaluation\\ and output a 1-10 integer for the grade.}
\end{figurebox}

We will maintain this formatting style whenever showing a \texttt{Signature} structure in the future: \textcolor{ctuorange}{orange inputs}, \textcolor{brown}{brown outputs} and the optional \textcolor{ctublue}{blue instructions}.

Sufficiently large instruction-tuned LLMs are usually good at reliably producing JSON output.
For smaller models or more complex output structures, it might be necessary to use some form of constrained generation as discussed in \ref{sec:inference}.
A JSON schema could be constructed automatically from the \texttt{Signature} and passed into a parser-based sampler.
However, this is not necessary for our use-case and the only safeguard we implement is repeated generation in case of a parsing error.

\subsection{Predict method}
To facilitate \texttt{Signature}-powered generation, we implement a \texttt{predict} method that 
involves 1) prepending a developer prompt to the messages, and 2) parsing of \texttt{Signature} outputs.

\begin{figurebox}{Predict method developer prompt with highlighted prompt sections: directive, context, examples and format specifications.}{box:predictdev}
    \hlbox{ctulightblue}{You are an intelligent function that returns structured JSON outputs matching a given schema.
    }
    
    \hlbox{ctulightblue}{
    You will receive a JSON object containing: \\
        - `inputs`: a dictionary of named inputs \\
        - `outputs`: a dictionary specifying the expected output fields with their types and descriptions\\
        - `instructions`: a task or question to answer (optional)\\

    Your job is to:\\
        1. Understand the task from `instructions` or infer it from `inputs` and `outputs`\\
        2. Use the `inputs` to compute or generate the answer\\
        3. Respond **only** with keys from the `outputs` dictionary and values matching the described types
    }
    \hlbox{ctulightblue}{Only return a flat JSON object like:\\
    \{
    "field1": <value matching type and description>,\\
    "field2": <...>
    \}}
    
    \hlbox{ctulightblue}{Do not add metadata, explanations, or wrap outputs in additional structures.\\
    Do not include type names or field descriptions in the output.\\
    Your output must be strictly valid JSON and fill **all** requested output fields.}
\end{figurebox}

The developer prompt has to clearly explain to the LLM how to work with the JSON-based \texttt{Signature}.
In figure \ref{box:predictdev} notice the sections of the prompt following prompt engineering principles outlined in \ref{sec:preng}.

First, the directive states the task, then further context is added about the \texttt{Signature} data structure and the task.
Next, notice the example showing the proper output. Finally, few more clarifying instructions about the output format are added.
In experiments, this prompt is successful in incentivizing parsable outputs adhering to the \texttt{Signature} specifications.

Parsing the output presents some challenges as the LLM sometimes wraps the JSON output into a Markdown code block
or uses inconsistent escape sequences. We implement a simple parser based on regular expressions that is able to parse 
the majority of outputs. In case of a parsing issue or model failure, such as getting stuck in a generation loop, we add a repeated generation
feature.

\subsection{Inference techniques implementation}
Leveraging the \texttt{predict} method and the modular \texttt{Signature}-based interface, we implement a suite of inference-time prompting techniques. 
Each technique is realized through systematic modifications of the \texttt{Signature} fields, changing the developer prompt and the chaining of multiple generation steps 
and function calls. This design allows for modularity and reuse while preserving transparency.
We implement the following methods.
\begin{enumerate}
    \item \textbf{Chain-of-thought}\cite{NEURIPS2022_8bb0d291}: Prepends a reasoning field to the \texttt{Signature} outputs which forms a scratch pad for the LLM.
    \item \textbf{Chain-of-thought with Self-consistency}\cite{wang2023selfconsistencyimproveschainthought}: Multiple CoT generations with majority-voting.
    \item \textbf{ReAct}\cite{yao2023reactsynergizingreasoningacting}: Adding tools allows the LLM to interleave thoughts and action steps.
    \item \textbf{Program-of-thought}\cite{chen2023programthoughtspromptingdisentangling}: Two-stage CoT with Python-code execution
    \item \textbf{Reflexion}\cite{shinn2023reflexionlanguageagentsverbal}: After an initial generation, the model is prompted to self-critique and revise its output.
    \item \textbf{Tree of Thoughts}\cite{yao2023treethoughtsdeliberateproblem}: The problem is first decomposed and each step is expanded, forming a thought tree, which is then traversed with BFS or DFS.
\end{enumerate}

These techniques are however not the main focus of this bachelor's thesis, and we proceed without further discussion or evaluation.
In our prompt optimization method, we will utilize only the basic Chain-of-thought module.

\section{Datasets}
In this section, we discuss choosing datasets for testing our method and comparing various prompt optimization approaches. 
While searching available datasets, we focus on the following criteria:
\begin{enumerate}
    \item \textbf{Output complexity}: We focus on more complex outputs. Specifically, datasets with multiple-choice or Yes/No answers are omitted. 
    This disqualifies commonly used datasets as MMLU or BigBenchHard.
    \item \textbf{Contamination}: Recently, researchers have expressed concern\cite{white2025livebenchchallengingcontaminationlimitedllm} 
    whether benchmarks are reliable evaluations of models as they might appear in their training data. We omit most common datasets, such as GSM8k\cite{cobbe2021gsm8k}, which has been shown to have inflated scores for some models\cite{testing_language_models_on_a_held_out_high_school_national_finals_exam}.
    \item \textbf{Output verification}: We prefer to use simple automatic verification rather than using LLM-as-a-judge, which has been shown to be 
    biased in some circumstances\cite{ye2024justiceprejudicequantifyingbiases}. Neither do we use human feedback, which defeats the purpose of automatic prompt optimization.
    \item \textbf{Difficulty}: We omit tasks where models already have near-perfect score. 
    \item \textbf{Benefit from non-trivial instruction}: We focus on tasks where helpful hints and step-by-step tutorial-like instructions might increase the probability of successful solution.
\end{enumerate}
We now list the datasets that we will use for evaluation and explain why they were chosen.
\subsection{Livebench}
The Livebench\cite{white2025livebenchchallengingcontaminationlimitedllm} dataset is very recent and has been created with the issue of data contamination in mind.
It also addresses the issues of LLM-as-a-judge verification and all its categories can be verified automatically. It is also very challenging, with top models achieving $65\%$ accuracy\cite{white2025livebenchchallengingcontaminationlimitedllm}.

Out of the tasks available in Livebench, we select the \texttt{Connections} task from the \texttt{Language} subset. 
This task consists of sorting given words into non-trivial groups of four based on semantics, phonetics and other features. 
An ideal prompt would attempt to list multiple possible aspects based on which the words can be sorted and also include a helpful example.

\subsection{Code Contests}
Programming puzzles are a difficult and easily verifiable task. Although \texttt{CodeContests}\cite{li2022competition} is an older dataset, 
we anticipate this dataset presents reduced contamination risks compared to datasets with simpler outputs. With LLM-powered coding assistants on the rise, we
feel this is a relevant application area for our method. 

\subsection{Sequences}
We design a small but challenging dataset based on predicting the next number in an integer sequence.
Each sequence is created according to a formula with randomly selected coefficients. The formulas fall into several categories, for example
\begin{itemize}
    \item \textbf{Linear with modulo}: $s(i) = \operatorname{mod}_{q}(a_{1} i + b_{1})$
    \item \textbf{Sum}: $s(i) = \sum_{j=0}^{i-1}a_{1} j + b_{1}$
    \item \textbf{Alternating}: $s(i) = a_{1}i + a_{2}i(-1)^{i}$.
\end{itemize}
This tests the model's ability to 1. detect and understand patterns and 2. systematically perform simple arithmetic. 
In practice, we will optimize just for a single sequence category and observe whether the optimizer evolves a prompt with a tutorial for the specific sequence category.
Experiments showed that the \texttt{Alternating} class of sequences has a good difficulty balance, and we will use it for evaluation.

\section{Evaluation Metrics}
The evaluation metric defines the optimization goal and thus forms its central component. 
Most evaluation metrics are task-specific and divisible into two categories based on whether they are used in a supervised or self-supervised context.
\subsection{Metrics for Supervised Optimization}
Supervised optimization is supported by gold labels and its underlying metrics all perform comparisons between the results and the gold labels.
These include classification metrics like accuracy or Hamming Loss, regression metrics like Mean Squared Error, and many others.

All three main benchmarks that we will use (\texttt{Connections}, \texttt{CodeContests}, \texttt{Sequences}) 
fall into this category. For each benchmark we use a simple accuracy metric. Given a dataset $\mathcal{D}$ and questions $q$ and gold labels $g$, $(q,g) \in \mathcal{D}$:
\begin{enumerate}
    \item \textbf{Connections}:  $\mathcal{F}_{\mathcal{D}_{\text{Conn}}}(q, g) = \operatorname{Overlap}(\operatorname{Groups}(q), \operatorname{Groups}(g))$
    \item \textbf{CodeContests}: $\mathcal{F}_{\mathcal{D}_{\text{Code}}}(q, g) = \operatorname{FinishesExecution}(q) + \operatorname{PassesAllCases}(q, g)$
    \item \textbf{Sequences}: $\mathcal{F}_{\mathcal{D}_{\text{Seq}}}(q, g) = \operatorname{Equals}(q, g)$
\end{enumerate} 

\subsection{Metrics for Self-Supervised Optimization}\label{sec:ssometrics}
In self-supervised contexts, metrics are usually based on reward models pretrained on human preference or environment data.
To allow our method to be applied to gold label-free problems, we turn to LLM-based direct pairwise comparisons.
Given a dataset $\mathcal{D}$ with queries $q$, output $y$ produced by prompt $P \in \mathscr{P}$ and a set of completions $\mathcal{C}_{q}$ for each query.
\begin{equation}
    \mathcal{F}_{\mathcal{D}}^{\text{pairwise}}(q, y, \mathcal{C}_{q}) = \operatorname{WinRate}(\{\operatorname{Compare}(q,y,c)\vsep c\in \mathcal{C}_{q}\}).
\end{equation}
In practice, we combine the output comparison with comparing the output's respective prompts.
These comparisons are then used as optimization signals in the \texttt{Feedback} operator.

\section{Optimization Framework}
Although our first implementation attempt utilized an evolutionary algorithm, we will use a basic population-based hill-climber algorithm.
This design decision has several reasons.
\begin{enumerate}
    \item Most PO research uses a hill-climber architecture.
    \item EAs suffer from slow convergence compared to state-of-the-art hill-climber PO\cite{xiang2025selfsupervisedpromptoptimization}.
    \item PO is complex as it is, and more complicated architectures only introduce more hyperparameters.
\end{enumerate}


\begin{algorithm}
    \caption{Prompt Optimization Hill-Climber}
    \label{alg:promptoptimloop}
    \KwIn{Dataset $\mathcal{D}$, Population size $S$, Iteration count $I$, Batch size $B$}
    \KwOut{Optimized Prompts $\mathscr{P}^{\star}$}
    $\mathcal{D}_{\text{train}}, \mathcal{D}_{\text{dev}}, \mathcal{D}_{\text{test}} \gets \operatorname{Split}(\mathcal{D})$ \tcp{Generate training splits}
    $\mathscr{P} \gets \operatorname{InstructionInduction}(\mathcal{D}_{\text{train}})$ \tcp{Induce initial prompts}
    $i \gets 0$ \tcp{Initialize iteration count}
    $\mathcal{C} \gets \{\}$ \tcp{Initialize solutions} 
    $\mathcal{E} \gets \{\}$ \tcp{Initialize scores}
    $\mathcal{A} \gets \mathscr{P}$ \tcp{All prompts}
    \While{$i<I$}{
        $Q, G \gets \operatorname{RandomSample}(\mathcal{D}_{\text{dev}}, B)$ \\
        $\mathcal{C} \gets \{\mathcal{C}_{q}^{\mathscr{P}}\vsep q \in Q\}$ \\
        $\mathcal{E} \gets \operatorname{Evaluate}(\mathcal{C}, G)$ \\
        $\mathscr{P} \gets \operatorname{Selection}(\mathscr{P}, \mathcal{E})$ \tcp{Pruning} 
        $\mathscr{P} \gets \operatorname{Expand}(\mathscr{P}, \mathcal{C}, \mathcal{E}, \mathcal{D}_{\text{train}})$ \\
        $\mathcal{A} \gets \mathcal{A} \cup \mathscr{P}$ \\
    }
    %$Q_{\text{test}}, G_{\text{test}} \gets D_{\text{test}}$\\
    %$\mathcal{C}_{\text{test}} \gets \{\mathcal{C}_{q}^{\mathcal{A}}\vsep q \in Q_{\text{test}}\}$\\
    %$\mathcal{E}_{\text{test}} \gets \operatorname{Evaluate}(\mathcal{C}, G_{\text{test}})$\\
    $P^{\star} \gets \underset{P\in\mathcal{A}}{\operatorname{argmax}}(\mathcal{E}_{\mathcal{D}_{\text{test}}}(P))$\\
    \Return{$P^{\star}$}
\end{algorithm}

In Algorithm \ref{alg:promptoptimloop} we iterate on the general algorithm \ref{alg:genoptimloop}. 
We will discuss the design of functions used in \ref{alg:promptoptimloop} in following sections.

\begin{itemize}
    \item \textbf{Expand}: The $\operatorname{Expand}$ function can be filled with various expansion operators, of which $\operatorname{InstructionInduction}$
    is a special case. 
    \item \textbf{Evaluate}: Evaluating and identifying the most promising prompts is handled by the $\operatorname{Evaluate}$ operator, which uses task-specific automatic evaluation or LLM-feedback.
    \item \textbf{Selection}: The $\operatorname{Selection}$ operator prunes the population and should maintain only the most promising and diverse prompts for the next expansion.
\end{itemize}

\subsection{Expansion Operator Design}
Expansion operators' job is extending the optimization population with new prompts. Remember notation from \ref{eq:metaprompting}:
\begin{equation*}
    P = \mathscr{M}_{\text{optim}}(M\vsep \mathcal{R}).
\end{equation*}
Notice the use of $\mathscr{M}_{\text{optim}}$, which utilizes non-zero sampling temperature $\tau > 0$ to encourage output diversity. 
Evidently the prompt generation task can be separated into two independent problems: 1. crafting the optimal \textit{Meta-prompt} $M$ 
and 2. designing a data retrieval function $\mathcal{R} = \mathcal{R}(\mathscr{P}, \mathcal{C}, \mathcal{D}, \mathcal{E})$.
The operators' design should address the following challenges:
\begin{enumerate}
    \item \textbf{Loss of generality}: When using task samples $(q, g) \in \mathcal{D}$, the model $\mathscr{M}_{\text{optim}}$ might focus on a single query $q$ and thus fail to generate general instructions.
    \item \textbf{Loss of diversity}: Even for  $\mathscr{M}_{\text{optim}}$ with $\tau>0$, the resulting prompts can be very similar and fail to explore the prompt space $\mathcal{I}$. 
    This ties into a broader exploration vs. exploitation balance issue.
    \item \textbf{Lack of optimization signal}: Research\cite{he2024crispomultiaspectcritiquesuggestionguidedautomatic}\cite{xiang2025selfsupervisedpromptoptimization} suggests that $\mathscr{M}_{\text{optim}}$ 
    can make use of feedback on prompts' outputs and that these textual signals are more effective than numerical scores.
    \item \textbf{Out of distribution \textit{Meta-prompt}}: Prompt engineering is a novel research area and does not have a substantial support in the LLM's training corpus.
    The \textit{Meta-prompt} $M$ thus has to be carefully constructed to help the model output relevant prompts.
\end{enumerate}

We now discuss the design of each prompt generation operator and display their signatures and \textit{meta-prompts}. 
Note that all operators are ultimately used in a CoT context, where a \texttt{reasoning} field is prepended to each signature's outputs.
\subsubsection{Lamarckian}
Instruction Induction\cite{honovich2022instructioninductionexamplesnatural} is used by many PO methods and often referred to as \texttt{Lamarckian Mutation}. 
We will adopt this terminology from now on and design our \texttt{Lamarckian} operator. 
Design of its meta-prompt, shown in figure \ref{box:lamarcksig}, takes into account the design challenges mentioned earlier by 1. warning the LLM to be general and not to focus on a single example, 
2. clearly states the problem using a directive and formatting specifications. 

The problem with diversity still persists, and we consider two approaches to solving it. 
We can increase the model's creativity by increasing its sampling temperature. Another approach is to use
some kind of \textit{seed}, for example a \textit{persona}. We experiment with using personas from PersonaHub\cite{ge2024scalingsyntheticdatacreation}.
Authors of this paper argued that seeding generation with the persona helps with creating novel synthetic data. 

For the data retrieval part, \texttt{Lamarckian} utilizes only examples of the datasets. 
We randomly sample $N$ examples from a separate training split. So
\begin{equation}
    \mathcal{R}_{\text{L}}(\mathcal{D}) = \operatorname{RandomSample}(\mathcal{D}_{\text{train}}, N)
\end{equation}
\begin{figurebox}{Lamarckian Signature. Formatting specifications trimmed.}{box:lamarcksig}
    \hlbox{ctuorange}{Task examples: \texttt{str} (Samples from a problem class) \\
    Persona (Optional): \texttt{str} (Assume this persona when writing the prompt)} 
    \hlbox{brown}{Prompt proposal: \texttt{str} (Instructions for solving the problem)} 
    \hlbox{ctublue}{Craft a \textbf{general} developer prompt to help an LLM with solving a class of problems.\\
    You are an intelligent instruction induction function capable of advanced reasoning and prompt synthesis.\\
    Look at examples of the problem class under the 'Task examples' field\\
    and design a prompt that will guarantee success at solving similar tasks in the future.\\
    Make sure your instructions are \textbf{TRULY GENERAL} and apply to all given samples \textbf{simultaneously}.
}
\end{figurebox}
\newpage
\subsubsection{Iterative}
The \texttt{Iterative} operator is one of the most common and simplest operators. 
It uses a sequence of prompts and their scores in ascending order.
The hope is for the LLM to deduce the optimization direction by looking at the differences in the prompts and incite it to continue the pattern.

Although some research\cite{yang2024largelanguagemodelsoptimizers} only uses the top prompts, we opt for a roulette selection method
and sort to prompts by score in ascending order. The number $N$ is a hyperparameter dictating how many prompts to sample.
We define the retrieval function as
\begin{equation}
   \mathcal{R}_{\text{I}}(\mathscr{P}, \mathcal{E}) = \operatorname{SortByScore}(\operatorname{RouletteSampling}(\mathscr{P}, \mathcal{E}, N), \mathcal{E})
\end{equation}
Other methods\cite{tang2024unleashingpotentiallargelanguage} additionally augment the \textit{meta-prompt} with task samples, similar to the \texttt{Lamarckian} operator.

In the \textit{meta-prompt}, we instruct the LLM to try to follow the sequence. Also, we specifically say to 'craft a new prompt'
as opposed to 'improve a prompt' to incite more novelty. For formatting, we use the same instruction set as in the \texttt{Lamarckian}.
\begin{figurebox}{Iterative Signature. Formatting specifications trimmed.}{box:itersig}
    \hlbox{ctuorange}{Old prompts: \texttt{list} (List of previous prompts with scores)}
    \hlbox{brown}{Prompt proposal: \texttt{str} (Better prompt)} 
    \hlbox{ctublue}{Craft a new prompt for an LLM.
    
    You are an intelligent pattern continuation function capable of advanced reasoning and prompt synthesis.\\
    You are given a history of past prompts along with their scores.\\
    They are listed in ascending order of fitness.\\
    Follow the sequence and design an improved prompt. \\
}
\end{figurebox}

\subsubsection{Reflective}
Recent PO literature\cite{xiang2025selfsupervisedpromptoptimization} shifts to using LLM outputs as optimization signals
and argues that utilizing only numerical signals is ineffective. To address this, we design an exploitative operator, which
aims to fix faults in the prompt by analyzing its failed attempt at a task sample. 

To achieve this, a more complex \texttt{Signature} is utilized. Its outputs guide the LLM to first critique the original prompt
and then improve it. Instructions are more complete with a step-by-step guide which explains the task clearly.
Note that 1. now we use "improve" wording, 2. we stress to only alter the prompt \textit{slightly}. This is done due to 
frequent observation of the model just creating an entirely different prompt only applicable to the single example task.
For formatting, we use the same instructions as in previous \textit{meta-prompts}.

In $\mathcal{R}$, we want to select the worst possible attempt. This means we optimize "from the bottom up" and try to bootstrap the
worst prompts. The retrieval function is
\begin{equation}
    \mathcal{R}_{\text{R}}(\mathscr{P}, \mathcal{C}, \mathcal{D}, \mathcal{E}) = \operatorname{JoinAttemptWithTask}(\operatorname{FindWorstAttempt}(\mathscr{P}, \mathcal{C}, \mathcal{E}, \mathcal{D})
\end{equation}

\begin{figurebox}{Reflective Signature. Formatting specifications trimmed.}{box:reflexsig}
    \hlbox{ctuorange}{Original prompt: \texttt{str} (Improve this prompt) \\
    Task question: \texttt{str} (Task on which the prompt was used) \\
    Solution: \texttt{str} (What the original prompt produced)}
    \hlbox{brown}{Original prompt critique: \texttt{str} (Faults in the original prompt) \\
    Prompt proposal: \texttt{str} (Improved prompt)} 

    \hlbox{ctublue}{Improve a prompt for an LLM.
    
    You are an intelligent reflection function capable of advanced reasoning and prompt synthesis.\\
    Follow these steps to craft a better prompt:\\
    - Analyze the original prompt and its suboptimal performance on a task sample.\\
    - Find failure points in the solution and cross-reference to identify weaknesses in the prompt.\\
    - Think of a critique that captures your findings\\
    - Apply your critique to \textit{slightly} alter the original prompt to improve it.\\
    Your improved prompt should still be \textbf{widely applicable and generic}.}
\end{figurebox}


\subsubsection{Feedback}
As we mentioned earlier, the \texttt{Feedback} operator is suitable for use in self-supervised settings.
It leverages reasoning traces from pairwise LLM-based comparisons, discussed in \ref{sec:ssometrics}. 
Let $\mathcal{E}_{\text{comp}}$ hold textual comparisons of each prompt and their attempts and
$P_{\text{base}} = \operatorname{RandomSample}(\mathscr{P})$. Then
\begin{equation}
    \mathcal{R}_{\text{F}}(\mathscr{P}, \mathcal{E}_{\text{comp}}) = \{P_{\text{base}}, \operatorname{GetComparisons}(P_{\text{base}}, \mathcal{E}_{\text{comp}})\}
\end{equation}
In the \textit{Meta-prompt}, we frame the task as critique synthesis and use "improve" wording to guide the LLM to start from the base prompt.
We also explain that each comparison has a different verdict and the base prompt might not always be the winner. For the formatting guide, we use the same instructions
as in the previous operators.

For large populations or tasks producing long prompts, we might run into issues with LLM context window length. 
However for our purpose, modern LLMs provide more than sufficient context limits. 
\begin{figurebox}{Feedback Signature. Formatting specifications trimmed.}{box:feedbacksig}
    \hlbox{ctuorange}{
        Base prompt: \texttt{str} (Improve this prompt) \\
        Comparisons: \texttt{str} (Base prompt compared to others)
    }
    \hlbox{brown}{Prompt proposal: \texttt{str} (Improved prompt)} 
    \hlbox{ctublue}{Improve a prompt for an LLM.

    You are an intelligent critique synthesis function capable of advanced reasoning. \\
    You are given a base prompt and a list of comparisons between the base prompt and other prompts.\\
    Some other prompts are better than the base prompt, some are worse.\\
    Your task is to analyze the comparisons and synthesize a new prompt that incorporates the feedback.}
\end{figurebox}

\subsubsection{Paraphrase}
To serve as another baseline for other operators, we implement a simple \texttt{Paraphrase} operator.
This operator performs random search in the prompt space by changing the wording and structure of a prompt.
The prompt is selected via the retrieval function
\begin{equation}
    \mathcal{R}_{\text{P}}(\mathscr{P}, \mathcal{E}) = \operatorname{RouletteSampling}(\mathscr{P}, \mathcal{E}).
\end{equation}
This method uses no optimization signal or improvement instructions and relies on pure chance of finding a more potent prompt.


\begin{figurebox}{Paraphrase Signature. Formatting specifications trimmed.}{box:parasig}
    \hlbox{ctuorange}{Original prompt: \texttt{str} (Prompt to paraphrase)}
    \hlbox{brown}{Prompt proposal: \texttt{str} (Paraphrased prompt)} 
    \hlbox{ctublue}{Paraphrase a prompt for an LLM.
                    
    You are an intelligent paraphrasing function capable of advanced reasoning and prompt synthesis.\\
    You are given a prompt and your task is to paraphrase it. \\
    Use synonyms and change the structure of the prompt but keep it semantically equivalent.}
\end{figurebox}
In figure \ref{box:formatting} we show the formatting specification part, which is identical for all prompt generation \textit{meta-prompts}.

\begin{figurebox}{Formatting specifications, which are the same for all operators}{box:formatting}
    Use Markdown formatting in your final answer to indicate bullet points and whatever else necessary.\\
    As a placeholder for the task question, '<INSERT TASK QUESTION HERE>' should be used exactly ONCE.\\
    In the final answer, do not include a title or any additional data, just the prompt.
\end{figurebox}

\subsection{Selection Operator}
At the start of each optimization step, we select $n_{\text{continue}}$ prompts to continue in the process and purge the rest. 
To achieve better prompt diversity, a method based on edit distance is used. 

This method, outlined in Algorithm \ref{alg:duplicpurge},
removes the closest prompt for each prompt, starting from the best prompts. This ensures that performant prompts are kept and their worse-performing duplicates are deleted.
We opt to use edit distance instead of semantic similarity, like BERT embeddings.

\begin{algorithm}
    \caption{Purge Duplicates}
    \label{alg:duplicpurge}
    \KwIn{Population $\mathscr{P}$, Pruning factor $f_{\text{prune}}$}
    \KwOut{Pruned population $\mathscr{P}_{\text{pruned}}$}
    $\mathscr{P}_{\text{sorted}} \gets \operatorname{SortByScore}(\mathscr{P}, \mathcal{E})$ \\
    $n_{\text{continue}} \gets \vert\mathscr{P}\vert(1-f_{\text{prune}})$
    $i \gets 0$
    \While{$i<n_{\text{continue}}$} {
        $P_{\text{select}} \gets \operatorname{GetFirst}(\mathscr{P}_{\text{sorted}})$ \\ 
        $P_{\text{purge}} \gets \underset{P\in\mathscr{P}\mid P \neq P_{\text{select}}}{\operatorname{argmax}} \operatorname{LevenshteinRatio}(P, P_{\text{select}})$ \\
        $\operatorname{Remove}(\mathscr{P}, P_{\text{purge}})$
    }
    $\mathscr{P}_{\text{pruned}} \gets \mathscr{P}$\\
    \Return{$\mathscr{P}$}
\end{algorithm}


\chapter{Experiments}
\section{Supervised Optimization Operator Evaluation}
In our main experiment, we compare the effectivity of 4 optimization operators on 3 diverse gold-labeled datasets.
We will compare the results and costs of each method against each other and a strong Instruction Induction baseline.

All our experiments were conducted throught the \texttt{OpenRouter API}, which offers many LLMs from different providers.
We will differentiate between optimizer LLM $\mathcal{M}_{\text{optim}}$ and $\mathcal{M}_{\text{solve}}$.
Both use the medium model \texttt{Google Gemma 3 27B}\cite{gemmateam2025gemma3technicalreport}. 
To encourage diversity, $\mathcal{M}_{\text{optim}}$ works with sampling temperature $\tau = 0.75$. 
Bigger sampling temperature was observed to diminish the model's ability to follow structured output specifications.
To keep prompt testing as deterministic as possible, we initialize $\mathcal(M)_{\text{solve}}$ with $\tau = 0.0$.


\subsection{Experiment setup}
We test our method against 3 datasets, \texttt{CodeContests}, \texttt{Connections}  and \texttt{Sequences}, which have 30 samples each 
and form 3 equal splits $\mathcal{D}_{\text{train}}$, $\mathcal{D}_{\text{dev}}$ and $\mathcal{D}_{\text{test}}$. Each out of the 4 operators - \texttt{Reflective}, \texttt{Iterative}, \texttt{Feedback} and \texttt{Paraphrase} - is tested on all three datasets for 3 repetitions
to alleviate randomness in the results.

To create a strong baseline, we begin by creating $50$ prompts $\mathcal{P}_{\text{baseline}}$ with Instruction Induction through the \texttt{Lamarckian} operator 
using \texttt{Persona}\cite{ge2024scalingsyntheticdatacreation} seeding. These prompts form $\mathcal{P}_{\text{baseline}}$. Examples are sampled from $\mathcal{D}_{\text{train}}$ with the whole 10 sample split being used for \texttt{Connections} and \texttt{Sequences}
and 3 samples for \texttt{CodeContests}. This is because code input/output pairs are a lot longer and we observed a greater likelihood of failure to follow output structure for more samples.

These $50$ prompts are evaluated on $\mathcal{D}_{\text{test}}$ and the second quintile (prompts 11-20 when ranked by test score) is used as the initial population $\mathcal{P}_{\text{init}}$ for the optimizer.
We use population size $S=10$, iteration count $I=10$, batch size $B=3$ and pruning factor $f_{\text{prune}} = 0.5$. This means that each step produces $5$ new prompts and $50$ prompts overall.
We select these parameters to make the optimization process resources \textit{roughly} comparable to the Instruction Induction baseline generation.


\subsection{Experiment results}
In tables \ref{tab:rescodecontests}, \ref{tab:resconnections} and \ref{tab:ressequences} we results of our 
main experiment on \texttt{CodeContests}, \texttt{Connections} and \texttt{Sequences} respectively.
In the row marked HB, we present the maximum score achieved by prompts $P\in\mathcal{P}_{\text{baseline}}$, forming a hard baseline (HB). 
This forms the hard baseline. The next row, marked SB, shows the best score out of the initializing prompts $P\in\mathcal{P}_{\text{init}}$, which forms a soft baseline (SB).
The following rows show the best score of each step $s$, which is calculated as an average over all three experiment repetitions $e\in\{1,2,3\}$:
\begin{equation}
    \frac{1}{3}\sum_{e=1}^{3}\operatorname{Max}(\mathcal{E}_{\mathcal{D}_{\text{test}}}(\mathcal{P}_{s}^{e})),
\end{equation}
where $\mathcal{P}_{s}^{e}$ signifies the population at step $s$ of experiment repetition $e$.
For each operator, the best step score is marked in \textbf{bold} and the best score overall is also \underline{underlined}.
Cells with values which surpass the soft baseline are shaded in \textcolor{green}{green} and those which fall short are in \textcolor{red}{red}.
We also include the average step score in bracketed smaller script, defined by 
\begin{equation}
    \frac{1}{3}\sum_{e=1}^{3}\operatorname{Mean}(\mathcal{E}_{\mathcal{D}_{\text{test}}}(\mathcal{P}_{s}^{e})).
\end{equation}
Figures \ref{fig:codecontests}, \ref{fig:connections} and \ref{fig:sequences} show the corresponding graphs with a gray dotted line marking 
the soft baseline and a orange solid line marking the hard baseline.

\begin{table}[htbp]
    \centering
    \captionsetup{font=small}
    \caption{Results for codecontests}  
    \label{tab:rescodecontests}
    \renewcommand{\arraystretch}{1.4} % row spacing

    \begin{tabular}{|c||c|c|c|c|}
    \hline
    \rowcolor{ctulightblue}
    \textsc{Step} &
    \cellcolor{ctulightblue}\textsc{Reflective} &
    \cellcolor{ctulightblue}\textsc{Iterative} &
    \cellcolor{ctulightblue}\textsc{Feedback} &
    \cellcolor{ctulightblue}\textsc{Paraphrase} \\
    \hline

    \rowcolor{ctuorange!15}
    HB & \textbf{0.26} & 0.26 & \textbf{0.26} & \textbf{0.26} \\ \hline
SB & 0.16 & 0.16 & 0.16 & 0.16 \\ \hline
$1$ & \cellcolor{lightgreen}\maxmean{0.23}{0.15} & \cellcolor{lightgreen}\maxmean{0.17}{0.12} & \cellcolor{lightgreen}\maxmean{0.26}{0.15} & \cellcolor{lightgreen}\maxmean{0.22}{0.14} \\ \hline
$2$ & \cellcolor{lightgreen}\maxmean{0.22}{0.12} & \cellcolor{lightgreen}\maxmean{0.25}{0.16} & \cellcolor{lightred}\maxmean{0.11}{0.06} & \cellcolor{lightgreen}\maxmean{0.22}{0.14} \\ \hline
$3$ & \cellcolor{lightgreen}\maxmean{0.23}{0.15} & \cellcolor{lightgreen}\maxmean{0.19}{0.12} & \cellcolor{lightred}\maxmean{0.16}{0.07} & \cellcolor{lightgreen}\maxmean{0.19}{0.14} \\ \hline
$4$ & \cellcolor{lightgreen}\maxmean{0.18}{0.14} & \cellcolor{lightgreen}\maxmean{\underline{\textbf{0.27}}}{0.15} & \cellcolor{lightred}\maxmean{0.09}{0.05} & \cellcolor{lightgreen}\maxmean{0.19}{0.14} \\ \hline
$5$ & \cellcolor{lightgreen}\maxmean{0.23}{0.17} & \cellcolor{lightred}\maxmean{0.16}{0.11} & \cellcolor{lightred}\maxmean{0.09}{0.05} & \cellcolor{lightgreen}\maxmean{0.22}{0.14} \\ \hline
$6$ & \cellcolor{lightgreen}\maxmean{0.23}{0.17} & \cellcolor{lightgreen}\maxmean{0.18}{0.11} & \cellcolor{lightred}\maxmean{0.08}{0.04} & \cellcolor{lightgreen}\maxmean{0.23}{0.14} \\ \hline
$7$ & \cellcolor{lightgreen}\maxmean{0.25}{0.16} & \cellcolor{lightgreen}\maxmean{0.25}{0.13} & \cellcolor{lightred}\maxmean{0.05}{0.04} & \cellcolor{lightgreen}\maxmean{0.19}{0.12} \\ \hline
$8$ & \cellcolor{lightred}\maxmean{0.15}{0.12} & \cellcolor{lightgreen}\maxmean{0.23}{0.16} & \cellcolor{lightred}\maxmean{0.08}{0.05} & \cellcolor{lightgreen}\maxmean{0.17}{0.11} \\ \hline
$9$ & \cellcolor{lightgreen}\maxmean{0.23}{0.14} & \cellcolor{lightgreen}\maxmean{0.19}{0.11} & \cellcolor{lightred}\maxmean{0.15}{0.07} & \cellcolor{lightgreen}\maxmean{0.17}{0.12} \\ \hline
$10$ & \cellcolor{lightgreen}\maxmean{0.24}{0.13} & \cellcolor{lightgreen}\maxmean{0.25}{0.13} & \cellcolor{lightred}\maxmean{0.08}{0.04} & \cellcolor{lightgreen}\maxmean{0.23}{0.15} \\ \hline


\end{tabular}
\end{table}
\begin{figure}
    \includegraphics[width=\linewidth]{codecontests.pdf}
    \caption{Optimization progression for the \texttt{CodeContests} dataset.}
    \label{fig:codecontests}
\end{figure}

\begin{table}[htbp]
    \centering
    \captionsetup{font=small}
    \caption{Results for connections}  
    \label{tab:resconnections}
    \renewcommand{\arraystretch}{1.4} % row spacing

    \begin{tabular}{|c||c|c|c|c|}
    \hline
    \rowcolor{ctulightblue}
    \textsc{Step} &
    \cellcolor{ctulightblue}\textsc{Reflective} &
    \cellcolor{ctulightblue}\textsc{Iterative} &
    \cellcolor{ctulightblue}\textsc{Feedback} &
    \cellcolor{ctulightblue}\textsc{Paraphrase} \\
    \hline

    \rowcolor{ctuorange!15}
    HB & \textbf{0.33} & \textbf{0.33} & \textbf{0.33} & 0.33 \\ \hline
SB & 0.20 & 0.20 & 0.20 & 0.20 \\ \hline
$1$ & \cellcolor{lightgreen}\maxmean{0.29}{0.15} & \cellcolor{lightgreen}\maxmean{0.27}{0.16} & \cellcolor{lightgreen}\maxmean{0.20}{0.15} & \cellcolor{lightgreen}\maxmean{0.31}{0.21} \\ \hline
$2$ & \cellcolor{lightgreen}\maxmean{0.20}{0.13} & \cellcolor{lightgreen}\maxmean{0.29}{0.18} & \cellcolor{lightgreen}\maxmean{0.24}{0.14} & \cellcolor{lightgreen}\maxmean{0.32}{0.20} \\ \hline
$3$ & \cellcolor{lightgreen}\maxmean{0.20}{0.14} & \cellcolor{lightgreen}\maxmean{0.27}{0.17} & \cellcolor{lightred}\maxmean{0.19}{0.14} & \cellcolor{lightgreen}\maxmean{0.29}{0.21} \\ \hline
$4$ & \cellcolor{lightred}\maxmean{0.18}{0.04} & \cellcolor{lightgreen}\maxmean{0.26}{0.17} & \cellcolor{lightred}\maxmean{0.09}{0.06} & \cellcolor{lightgreen}\maxmean{0.28}{0.19} \\ \hline
$5$ & \cellcolor{lightred}\maxmean{0.16}{0.08} & \cellcolor{lightgreen}\maxmean{0.27}{0.18} & \cellcolor{lightred}\maxmean{0.14}{0.09} & \cellcolor{lightgreen}\maxmean{0.31}{0.20} \\ \hline
$6$ & \cellcolor{lightred}\maxmean{0.17}{0.09} & \cellcolor{lightgreen}\maxmean{0.27}{0.19} & \cellcolor{lightred}\maxmean{0.13}{0.08} & \cellcolor{lightgreen}\maxmean{0.32}{0.21} \\ \hline
$7$ & \cellcolor{lightred}\maxmean{0.16}{0.08} & \cellcolor{lightgreen}\maxmean{0.29}{0.19} & \cellcolor{lightred}\maxmean{0.10}{0.04} & \cellcolor{lightgreen}\maxmean{0.30}{0.19} \\ \hline
$8$ & \cellcolor{lightred}\maxmean{0.16}{0.08} & \cellcolor{lightgreen}\maxmean{0.28}{0.19} & \cellcolor{lightred}\maxmean{0.07}{0.04} & \cellcolor{lightgreen}\maxmean{0.33}{0.23} \\ \hline
$9$ & \cellcolor{lightred}\maxmean{0.16}{0.07} & \cellcolor{lightgreen}\maxmean{0.27}{0.20} & \cellcolor{lightred}\maxmean{0.05}{0.03} & \cellcolor{lightgreen}\maxmean{0.28}{0.22} \\ \hline
$10$ & \cellcolor{lightred}\maxmean{0.16}{0.07} & \cellcolor{lightgreen}\maxmean{0.26}{0.16} & \cellcolor{lightred}\maxmean{0.07}{0.03} & \cellcolor{lightgreen}\maxmean{\underline{\textbf{0.35}}}{0.25} \\ \hline


    \end{tabular}
\end{table}

\begin{figure}
    \includegraphics[width=\linewidth]{connections.pdf}
    \caption{Optimization progression for the \texttt{Connections} dataset.}
    \label{fig:connections}
\end{figure}

\begin{table}[htbp]
    \centering
    \captionsetup{font=small}
    \caption{Results for sequences}  
    \label{tab:ressequences}
    \renewcommand{\arraystretch}{1.4} % row spacing

    \begin{tabular}{|c||c|c|c|c|}
    \hline
    \rowcolor{ctulightblue}
    \textsc{Step} &
    \cellcolor{ctulightblue}\textsc{Reflective} &
    \cellcolor{ctulightblue}\textsc{Iterative} &
    \cellcolor{ctulightblue}\textsc{Feedback} &
    \cellcolor{ctulightblue}\textsc{Paraphrase} \\
    \hline

    \rowcolor{ctuorange!15}
    HB & \textbf{0.70} & 0.70 & \textbf{0.70} & 0.70 \\ \hline
SB & 0.40 & 0.40 & 0.40 & 0.40 \\ \hline
$1$ & \cellcolor{lightgreen}\maxmean{0.47}{0.23} & \cellcolor{lightgreen}\maxmean{0.63}{0.36} & \cellcolor{lightgreen}\maxmean{0.43}{0.29} & \cellcolor{lightgreen}\maxmean{0.73}{0.38} \\ \hline
$2$ & \cellcolor{lightgreen}\maxmean{0.53}{0.39} & \cellcolor{lightgreen}\maxmean{0.53}{0.32} & \cellcolor{lightgreen}\maxmean{0.47}{0.32} & \cellcolor{lightgreen}\maxmean{0.60}{0.33} \\ \hline
$3$ & \cellcolor{lightred}\maxmean{0.27}{0.15} & \cellcolor{lightgreen}\maxmean{0.67}{0.43} & \cellcolor{lightgreen}\maxmean{0.47}{0.33} & \cellcolor{lightgreen}\maxmean{0.63}{0.37} \\ \hline
$4$ & \cellcolor{lightgreen}\maxmean{0.47}{0.21} & \cellcolor{lightgreen}\maxmean{0.63}{0.42} & \cellcolor{lightgreen}\maxmean{0.47}{0.32} & \cellcolor{lightgreen}\maxmean{0.60}{0.42} \\ \hline
$5$ & \cellcolor{lightgreen}\maxmean{0.57}{0.33} & \cellcolor{lightgreen}\maxmean{0.63}{0.40} & \cellcolor{lightred}\maxmean{0.40}{0.28} & \cellcolor{lightgreen}\maxmean{0.60}{0.38} \\ \hline
$6$ & \cellcolor{lightgreen}\maxmean{0.53}{0.35} & \cellcolor{lightgreen}\maxmean{0.57}{0.34} & \cellcolor{lightred}\maxmean{0.27}{0.19} & \cellcolor{lightgreen}\maxmean{\underline{\textbf{0.80}}}{0.45} \\ \hline
$7$ & \cellcolor{lightgreen}\maxmean{0.57}{0.44} & \cellcolor{lightgreen}\maxmean{0.60}{0.31} & \cellcolor{lightred}\maxmean{0.37}{0.23} & \cellcolor{lightgreen}\maxmean{0.73}{0.37} \\ \hline
$8$ & \cellcolor{lightgreen}\maxmean{0.63}{0.46} & \cellcolor{lightgreen}\maxmean{0.70}{0.50} & \cellcolor{lightred}\maxmean{0.37}{0.25} & \cellcolor{lightgreen}\maxmean{0.57}{0.29} \\ \hline
$9$ & \cellcolor{lightgreen}\maxmean{0.50}{0.40} & \cellcolor{lightgreen}\maxmean{\underline{\textbf{0.80}}}{0.37} & \cellcolor{lightgreen}\maxmean{0.43}{0.27} & \cellcolor{lightgreen}\maxmean{0.63}{0.44} \\ \hline
$10$ & \cellcolor{lightgreen}\maxmean{0.57}{0.41} & \cellcolor{lightgreen}\maxmean{0.57}{0.39} & \cellcolor{lightred}\maxmean{0.33}{0.21} & \cellcolor{lightgreen}\maxmean{0.60}{0.35} \\ \hline


    \end{tabular}
\end{table}


\begin{figure}
    \includegraphics[width=\linewidth]{sequences.pdf}
    \caption{Optimization progression for the \texttt{Sequences} dataset.}
    \label{fig:sequences}
\end{figure}

\begin{figure}
    \includegraphics[width=\linewidth]{relative.pdf}
    \caption{Average relative improvement of operators.}
    \label{fig:relative}
\end{figure}

\begin{figure}
    \includegraphics[width=\linewidth]{usage.pdf}
    \caption{Token optimization cost per operator. Logarithmic scale.}
    \label{fig:usage}
\end{figure}

Somewhat counter-intuitively, we observe that the simpler operators, \texttt{Iterative} and \texttt{Paraphrase},
consistently outperform the others over all datasets and are the only ones which outperform the hard baseline at any point.

The most complicated operator, \texttt{Feedback}, which draws on binary output and prompt comparisons exhibits the
worst performance overall and falls under the soft baseline for all tasks later in the optimization process. This also happens to the
\texttt{Reflective} operator on the \texttt{Connections} task.

Improvements over hard baseline are marginal and we have to conclude that the optimization did not produce 
prompts that would help the model solve the task consistently. However we observe significant gains 
over the soft baseline, which was used to initialize the experiment.

In figure \ref{fig:relative} we show the progressive score change relative to the soft baseline
over all experiments. This graphic again illustrates that the \texttt{Feedback} operator consistently downgrades performance.

We also graph token usage during optimization in \ref{fig:usage}. Due to the immense requirements of
the \texttt{Feedback} operator, a logarithmic scale is used. While other operators need around 200 thousand tokens for a complete optimizer routine,
\texttt{Feedback} needs almost 4 million.


\subsection{Discussion}

Contrary to expectations, operators utilizing textual feedback signals were outperformed by simpler operators
both on raw performance and on the ratio of performance to computation resources.
The most striking is the failure of the \texttt{Feedback} operator. This operator, which needs almost 20x the tokens as the other operators,
proposes new prompts using a history of comparisons between a base prompt and other prompts and their outputs. 

We now discuss several possible explanations for the failure of the \texttt{Feedback} and \texttt{Reflective} operators.
\begin{enumerate}
    \item \textbf{Weak model}: Our optimizer model \texttt{Google Gemma 3 27B} is smaller than most models used in the literature, due to our limited computation budget. It could be the case that generating effective feedback and using it to improve prompts is an emergent capability of larger models.
    \item \textbf{Structured generation}: We used our custom structured generation framework which allowed us to easily define \textit{Meta-prompts} and use inference techniques such as CoT, but it is possible that it made the already challenging prompt optimization task even harder and more convoluted, favouring simpler operators.
    \item \textbf{Unclear comparison formulation}: In \texttt{Feedback}, the comparisons are made between \texttt{prompt\_a} and \texttt{prompt\_b}. Although we include information about whether the base prompt won the comparison, we did not state whether it was \texttt{prompt\_a} or \texttt{prompt\_b}. This was done under the assumption that the model could infer this.
    \item \textbf{Overfitting}: In comparisons using \texttt{Reflective} and \texttt{Feedback}, the model looks at outputs on a single task. For example, in \texttt{Sequences}, many prompts include instructions like ``Output the last negative number''. It would probably be beneficial to compare or score attempts at multiple tasks simultaneously.
    \item \textbf{Diversity and chance}: Both \texttt{Reflective} and \texttt{Feedback} use the ``improve'' wording as opposed to \texttt{Paraphrase} and \texttt{Iterative}, which use the ``create new'' wording. Although not substantiated in data, the latter operators seem to be generating more diverse prompts. This plays well with our edit-distance-based pruning method.
    \item \textbf{Comparisons vs. performance}: It begs the question whether our LLM-generated comparisons reflect the actual quality of outputs and prompts, or if they promote the model's biases — which favour length and sounding smart~\cite{ye2024justiceprejudicequantifyingbiases}.
\end{enumerate}

By trying to establish a strong baseline, we did not utilize the best prompts for initialization. 
As initial prompts play a crucial role\cite{yang2024dualphaseacceleratedpromptoptimization} in exploring the vast search space,
this might have hindered our method's performance. We hoped to achieve better initial prompt diversity by seeding the \texttt{Lamarckian} operator using \texttt{Personas}~\cite{ge2024scalingsyntheticdatacreation}.
However this diversity seems to come at the cost of diminished initial scores. We cannot answer for sure whether this trade-off is worth it. 

Our optimizer hyperparameters were selected arbitrarily and focused on depth rather than breadth by prioritizing iteration count over population size.
On all tasks except \texttt{Sequences}, the optimization process saturated quite early. This suggests that focusing on depth is misguided 
and larger population sizes might be beneficial.

On \texttt{Sequences}, most of the best performing prompts include the same two few-shot examples, which have been present since initialization.
Although examples are no doubt beneficial, for this particular task, they might be giving unfair advantage as some 
samples in $\mathcal{D}_{\text{test}}$ had the same answers although the sequences were different.  

Contrary to expectations, operators with access to the optimized prompts' outputs on $\mathcal{D}_{\text{dev}}$ did not evolve any new notable examples 
and did not attempt to augment existing examples. This can be explained by the fact that we do not emphasize few-shot prompting in the \textit{Meta-prompts}.
\newpage

\section{Self-Supervised Optimization for Open-Ended Tasks}
To showcase the applicability of our method on tasks without available gold labels, we conduct two experiments
on creative applications. The only designed operator applicable in self-supervised contexts is the \texttt{Feedback} 
operator which performs direct output and prompt comparisons.

\subsection{Limericks}
Limerick is a short humorous poem. We design a simple task, where the LLM has to write an alliterative limerick about an animal. 
All words should start with the starting letter of the animal given, for example:

\begin{promptbox}[label={box:limerick}]{Limerick task example}
    \textbf{Input}:
    \hlbox{ctuorange}{snake}
    \textbf{Output}:
    \hlbox{ctublue}{silly snakes swiftly slide,\\sneaky, slippery, side-to-side,\\slurping soda,\\singing sonata,\\simply sparkling, super-sized!}
\end{promptbox}

We provide three such examples without further context and observe the quality of the results. These examples are used to construct the initial population through
the \texttt{Lamarckian} operator.
We run the optimizer with population size $S=4$, iteration count $I=3$, batch size $B=1$ and pruning factor $f_{\text{prune}} = 0.5$.
The LLM of choice is again the \texttt{Google Gemma 3 27B} with $\tau = 0.75$ for $\mathcal{M}_{\text{optim}}$ and $\tau = 0.0$ for $\mathcal{M}_{\text{solve}}$.

We present the best results from the initial population (generation $0$) and the last generation ($3$), as ranked by $\mathcal{M}_{\text{optim}}$.
The initial prompts already produce satisfactory results, with the winning result making sense and having only one error in alliteration.
However during the optimization, it seems that the comparison function started favoring longer results, consistent with findings of Ye et al.\cite{ye2024justiceprejudicequantifyingbiases}.
This problem originated in the first optimization step, where the \texttt{Feedback} operator created a prompt that instructed to "provide 3 different poem options".
Comparisons show that multiple options are preferred by the comparison function as they "showcase more variety and exploration of the theme".

\begin{promptbox}[label={box:limerickresult}]{Limerick optimization result}
\textbf{Initial}:
    \hlbox{ctulightblue}{
        daring dogs delightfully dash,\\
        dreaming deeply, a dazzling flash,\\
        digging daily,\\
        dancing daintily,\\
        devoted dreamers, a delightful dash!
    }
\textbf{Step 3}:
    \hlbox{ctulightblue}{
        daring dogs delightedly dash, \\
        dancing daily, a delightful flash, \\
        dreaming deeply, \\ 
        digging deftly, \\ 
        delicious dreams in a doggy splash. \\
\\
        dazzling dogs dreamily doze, \\ 
        delivering devotion, as everyone knows,\\ 
        dancing daily, \\
        delighting dually, \\
        deliciously drifting where the dandelion grows. \\
\\ 
        dynamic dogs diligently dive,\\ 
        discovering delights, truly alive,\\ 
        dreaming deeply,\\
        dancing steeply,\\
        daringly doing, to thrive and strive.
    }
\end{promptbox}

We conclude that our method failed to improve poem-generating prompts in this experiment, probably due to the 
ambiguous task definition, small population and batch sizes. Performance could be improved by designing 
a more specific operator set for creative tasks, as opposed our general-purpose operator set.

\subsection{Images}
We employ OpenAI's \texttt{Dall-E 3}\cite{BetkerImprovingIG} to demonstrate that our method can create prompts even for diffusion text-to-image models.
Learning from the failure from the previous experiment, we specialize the operator set for the task of image generation.
Utilizing the vision capabilities of \texttt{Google Gemma 3 27B}, the \texttt{Feedback} operator is modified to allow direct image comparisons. 

We choose the same hyperparameters as in the Limerick experiment. For the optimizer LLM, the vision-enabled \texttt{Google Gemma 3 27B} is used with $\tau = 0.75$ and 
$\mathcal{M}_{\text{solve}}$ is a diffusion text-to-image model \texttt{Dall-E 3} with image size $1024\times1024$ and default settings.

We select two triples of emotionally expressive adjectives and frame the optimization as a search for a prompt that creates an image best fitting this description. 
The first triple is "wistful", "hopeful", "lonely" and the second is "graceful", "furious", "vulnerable". In figures \ref{fig:wistful} and \ref{fig:graceful} the reader can find
highest win rate images from each optimization step.


\begin{figure}[htbp]
    \centering

    \begin{subfigure}{0.24\linewidth}
        \includegraphics[width=\linewidth]{image-gen0-id725f8.jpg}
        \caption{Initial}
    \end{subfigure}
    \hfill
    \begin{subfigure}{0.24\linewidth}
        \includegraphics[width=\linewidth]{image-gen1-id534fc.jpg}
        \caption{Step 1}
    \end{subfigure}
    \hfill
    \begin{subfigure}{0.24\linewidth}
        \includegraphics[width=\linewidth]{image-gen2-id62971.jpg}
        \caption{Step 2}
    \end{subfigure}
    \hfill
    \begin{subfigure}{0.24\linewidth}
        \includegraphics[width=\linewidth]{image-gen3-id4cf7b.jpg}
        \caption{Step 3}
    \end{subfigure}

    \caption{Best images from each optimization step for the words "wistful", "hopeful" and "lonely".}
    \label{fig:wistful}
\end{figure}

\begin{figure}[htbp]
    \centering

    \begin{subfigure}{0.24\linewidth}
        \includegraphics[width=\linewidth]{image-gen0-id1de72.jpg}
        \caption{Initial}
    \end{subfigure}
    \hfill
    \begin{subfigure}{0.24\linewidth}
        \includegraphics[width=\linewidth]{image-gen1-id4edb5.jpg}
        \caption{Step 1}
    \end{subfigure}
    \hfill
    \begin{subfigure}{0.24\linewidth}
        \includegraphics[width=\linewidth]{image-gen2-id6a3db.jpg}
        \caption{Step 2}
    \end{subfigure}
    \hfill
    \begin{subfigure}{0.24\linewidth}
        \includegraphics[width=\linewidth]{image-gen3-id622c1.jpg}
        \caption{Step 3}
    \end{subfigure}

    \caption{Best images from each optimization step for the words "graceful", "furious" and "vulnerable".}
    \label{fig:graceful}
\end{figure}

All generated images reflect the target description and, as artistic taste is subjective, we will leave it to the reader 
to decide if the optimized prompts produce better images. We observe, that in \ref{fig:wistful}, the image seems to get more personal as it zooms in on the subject.
The winning images are all in a very similar style. This is in contrast to \ref{fig:graceful}, where the focus seems to have shifted to 
photorealistic images in the first optimization step.

Note that in this case, the optimization is framed in a different manner from previous experiments.
We are conducting "run-time" optimization for a particular task instance unlike previous "compile-time" experiments, where we optimized for a whole task class.

\chapter{Conclusion}
\section{Future work}
Prompt optimization is an exciting and exceptionally active branch of research. 
Our experiments inspire further work on this topic and we will discuss possible research topics in this section.
\begin{enumerate}
    \item \textbf{Statistical methods for evaluation}
    \item \textbf{Finding optimal \textit{Meta-prompts}}
    \item \textbf{Improved structured generation}
    \item \textbf{Prompt representation structures}
\end{enumerate}
\section{Conclusion}

\appendix

\printindex


\bibliographystyle{ieeetr}
\bibliography{references}


\end{document}