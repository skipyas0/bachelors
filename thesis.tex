% arara: pdflatex: { synctex: yes }
% arara: makeindex: { style: ctuthesis }
% arara: bibtex

% The class takes all the key=value arguments that \ctusetup does,
% and a couple more: draft and oneside
\documentclass[twoside]{ctuthesis}

\ctusetup{
	preprint = \ctuverlog,
	mainlanguage = english,
	titlelanguage = english,
	otherlanguages = {czech},
	title-czech = {Metody evoluční optimalizace vstupních řetězců pro velké jazykové modely},
	title-english = {Methods of Evolutionary Optimization of Prompts for Large Language Models},
	subtitle-czech = {},
	subtitle-english = {},
	doctype = B,
	faculty = F3,
	department-english = {Artificial Intelligence Center},
	department-czech = {Centrum pro umělou inteligenci},
	author = {Vojtěch Klouda},
	supervisor = {Ing. Jan Drchal PhD.},
	supervisor-address = {Resslova 307/9 Praha, E-322},
%	supervisor-specialist = {John Doe},
	fieldofstudy-english = {Artificial Intelligence},
	subfieldofstudy-english = {Natural Language Processing},
	fieldofstudy-czech = {Umělá inteligence},
	subfieldofstudy-czech = {Zpracování přirozeného jazyka},
	keywords-czech = {jazykový model, evoluční algoritmus},
	keywords-english = {language model, evolutionary algorithm},
	day = 10,
	month = 5,
	year = 2025,
%	specification-file = {ctutest-zadani.pdf},
%	front-specification = true,
%	front-list-of-figures = false,
%	front-list-of-tables = false,
%	monochrome = true,
%	layout-short = true,
}

\ctuprocess

\addto\ctucaptionsczech{%
	\def\supervisorname{Vedoucí}%
	\def\subfieldofstudyname{Studijní program}%
}

\ctutemplateset{maketitle twocolumn default}{
	\begin{twocolumnfrontmatterpage}
		\ctutemplate{twocolumn.thanks}
		\ctutemplate{twocolumn.declaration}
		\ctutemplate{twocolumn.abstract.in.titlelanguage}
		\ctutemplate{twocolumn.abstract.in.secondlanguage}
		\ctutemplate{twocolumn.tableofcontents}
		\ctutemplate{twocolumn.listoffigures}
	\end{twocolumnfrontmatterpage}
}

% todo command
\newcommand{\todo}[1]{\textsuperscript{\textbf{\textcolor{red}{#1}}}}
\newcommand{\vsep}{\, \vert \,}

\newcommand{\maxmean}[2]{#1 {\color{gray}\scriptsize (#2)}}


% Theorem declarations, this is the reasonable default, anybody can do what they wish.
% If you prefer theorems in italics rather than slanted, use \theoremstyle{plainit}
\theoremstyle{plain}
\newtheorem{theorem}{Theorem}[chapter]
\newtheorem{corollary}[theorem]{Corollary}
\newtheorem{lemma}[theorem]{Lemma}
\newtheorem{proposition}[theorem]{Proposition}

\theoremstyle{definition}
\newtheorem{definition}[theorem]{Definition}
\newtheorem{example}[theorem]{Example}
\newtheorem{conjecture}[theorem]{Conjecture}

\theoremstyle{note}
\newtheorem*{remark*}{Remark}
\newtheorem{remark}[theorem]{Remark}

\setlength{\parskip}{5ex plus 0.2ex minus 0.2ex}

% Abstract in Czech
\begin{abstract-czech}
	TODO
\end{abstract-czech}

% Abstract in English
\begin{abstract-english}
	TODO
\end{abstract-english}

% Acknowledgements / Podekovani
\begin{thanks}
	= )))
\end{thanks}

% Declaration / Prohlaseni
\begin{declaration}
Prohlašuji, že jsem předloženou práci vypracoval samostatně, a že jsem uvedl veškerou použitou literaturu.

V Praze, \ctufield{day}.~\monthinlanguage{title}~\ctufield{year}
\end{declaration}

% Only for testing purposes
\listfiles
\usepackage[pagewise]{lineno}
\usepackage{lipsum,blindtext}
\usepackage{mathrsfs} % provides \mathscr used in the ridiculous examples

\begin{document}

\maketitle

\chapter{Introduction}
This is all very important
\chapter{Literature Review}
\section{Large Language Models}
Since the Transformer architecture \todo{Cite attention is all you need} brought us the first GPT and BERT models, LLMs have changed the NLP field.
Now there are many powerful proprietary (OpenAI, Anthropic) and open-source (Llama, Qwen) models. \todo{cite}

\section{Prompts}
Prompts are instructions we give to the LLM to get the output we want.  
It has been shown experimentally that small changes in prompt design, like simple paraphrasing, can influence the accuracy of the model's output. \todo{cite} 
Naturally this inspired research into optimizing LLM performance by changing its prompts.
\subsection{Prompting techniques}
In solving challenging reasoning tasks, it is beneficial to get the model to show its reasoning steps. 
As LLMs just predict the next token, building on its previous thoughts improves performance on tasks like the GSM8k benchmark.
This technique, called Chain-of-Thought has been built upon which resulted in methods such as Tree-of-Thought, ReAct and Reflexion. \todo{cite prompt survey} 
\subsection{Prompt optimization methods}
There has been work of optimizing soft and discrete prompts. 
As proprietary models usually do not allow access to its internal states, soft prompt optimization is not and option there.
Several search and optimization methods have been employed in this task, like beam search and evolutionary algorithms. \todo{sota rewrite here}


\appendix

\printindex

\appendix

\bibliographystyle{ieeetr}
\bibliography{references}

%\ctutemplate{specification.as.chapter}

\end{document}