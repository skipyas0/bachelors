% arara: pdflatex: { synctex: yes }
% arara: makeindex: { style: ctuthesis }
% arara: bibtex

% The class takes all the key=value arguments that \ctusetup does,
% and a couple more: draft and oneside
\documentclass[twoside]{ctuthesis}

\ctusetup{
	preprint = \ctuverlog,
	mainlanguage = english,
	titlelanguage = english,
	otherlanguages = {czech},
	title-czech = {Metody evoluční optimalizace vstupních řetězců pro velké jazykové modely},
	title-english = {Methods of Evolutionary Optimization of Prompts for Large Language Models},
	subtitle-czech = {},
	subtitle-english = {},
	doctype = B,
	faculty = F3,
	department-english = {Artificial Intelligence Center},
	department-czech = {Centrum pro umělou inteligenci},
	author = {Vojtěch Klouda},
	supervisor = {Ing. Jan Drchal PhD.},
	supervisor-address = {Resslova 307/9 Praha, E-322},
%	supervisor-specialist = {John Doe},
	fieldofstudy-english = {Artificial Intelligence},
	subfieldofstudy-english = {Natural Language Processing},
	fieldofstudy-czech = {Umělá inteligence},
	subfieldofstudy-czech = {Zpracování přirozeného jazyka},
	keywords-czech = {jazykový model, evoluční algoritmus},
	keywords-english = {language model, evolutionary algorithm},
	day = 10,
	month = 5,
	year = 2025,
%	specification-file = {ctutest-zadani.pdf},
%	front-specification = true,
%	front-list-of-figures = false,
%	front-list-of-tables = false,
%	monochrome = true,
%	layout-short = true,
}

\ctuprocess

\addto\ctucaptionsczech{%
	\def\supervisorname{Vedoucí}%
	\def\subfieldofstudyname{Studijní program}%
}

\ctutemplateset{maketitle twocolumn default}{
	\begin{twocolumnfrontmatterpage}
		\ctutemplate{twocolumn.thanks}
		\ctutemplate{twocolumn.declaration}
		\ctutemplate{twocolumn.abstract.in.titlelanguage}
		\ctutemplate{twocolumn.abstract.in.secondlanguage}
		\ctutemplate{twocolumn.tableofcontents}
		\ctutemplate{twocolumn.listoffigures}
	\end{twocolumnfrontmatterpage}
}

% todo command
\newcommand{\todo}[1]{\textsuperscript{\textbf{\textcolor{red}{#1}}}}
\newcommand{\vsep}{\, \vert \,}

\newcommand{\maxmean}[2]{#1 {\color{gray}\scriptsize (#2)}}


% Theorem declarations, this is the reasonable default, anybody can do what they wish.
% If you prefer theorems in italics rather than slanted, use \theoremstyle{plainit}
\theoremstyle{plain}
\newtheorem{theorem}{Theorem}[chapter]
\newtheorem{corollary}[theorem]{Corollary}
\newtheorem{lemma}[theorem]{Lemma}
\newtheorem{proposition}[theorem]{Proposition}

\theoremstyle{definition}
\newtheorem{definition}[theorem]{Definition}
\newtheorem{example}[theorem]{Example}
\newtheorem{conjecture}[theorem]{Conjecture}

\theoremstyle{note}
\newtheorem*{remark*}{Remark}
\newtheorem{remark}[theorem]{Remark}

\setlength{\parskip}{5ex plus 0.2ex minus 0.2ex}

% Abstract in Czech
\begin{abstract-czech}
	TODO
\end{abstract-czech}

% Abstract in English
\begin{abstract-english}
	TODO
\end{abstract-english}

% Acknowledgements / Podekovani
\begin{thanks}
	= )))
\end{thanks}

% Declaration / Prohlaseni
\begin{declaration}
Prohlašuji, že jsem předloženou práci vypracoval samostatně, a že jsem uvedl veškerou použitou literaturu.

V Praze, \ctufield{day}.~\monthinlanguage{title}~\ctufield{year}
\end{declaration}

% Only for testing purposes
\listfiles
\usepackage[pagewise]{lineno}
\usepackage{lipsum,blindtext}
\usepackage{mathrsfs} % provides \mathscr used in the ridiculous examples

\begin{document}

\maketitle

\chapter{Introduction}
\section{Background}
The central focus of this work is an instance of an LLM, denoted $\mathcal{M}$. 
When appropriate, we can differentiate between instances with a lower index, specifying its purpose. 
For example, when using separate LLM instances for optimizing and task-solving, we will denote them $\mathcal{M}_{optim}$ and $\mathcal{M}_{solve}$ respectively.
This way, we put emphasis on the fact we can choose a different LLM provider and hyperparameters for each instance.


In general, $\mathcal{M}: \mathbb{T} \times \mathbb{T}$ is a stochastic (for a positive sampling temperature) mapping on the space of text sequences $\mathbb{T}$.
A prompt $p \in \mathbb{T}$ is a text sequence that, when inputted into an LLM, produces an output
\begin{equation}
    y \sim \mathcal{M}(p).
\end{equation} 

We can use LLMs to solve a general task
\begin{equation}
    t \in \mathcal{D} \vsep \mathcal{D} = \{ (q_1, g_1), (q_2, g_2), ... , (q_n, g_n), \},
\end{equation}
where $\mathbb{D}$ is a dataset consisting of $n$ pairs of queries $q$ and gold-labels $g$. 
For open-ended tasks, the gold-label does not exist.


We can further define a prompt as
\begin{equation}
    p = \mathbf{i}(q),
\end{equation}
where $\mathbf{i} \in \mathbf{T}$ is a set of text instructions into which a task query $q$ is inserted. 

\begin{algorithm}
    \caption{General optimization loop}
    \label{alg:genoptimloop}
    \KwIn{Initialization Operator $\mathcal{O}_I$, Selection Operator $\mathcal{O}_S$, Expansion Operator $\mathcal{O}_E$, Termination Condition $\Phi_{stop}$}
    \KwOut{Optimized Population $\mathcal{P}$}
    \KwData{$\mathcal{P} \gets \mathcal{O}_I$} \tcp{Initialize the population}
    \While{$\neg \Phi_{stop}(\mathcal{P})$}{
        \tcp{Selection and Expansion Steps}
        $\mathcal{P}_{\text{selected}} \gets \mathcal{O}_S(\mathcal{P})$ \\ 
        $\mathcal{P}_{\text{expanded}} \gets \mathcal{O}_E(\mathcal{P}_{\text{selected}})$ \\
        $\mathcal{P} \gets \mathcal{P}_{\text{expanded}}$ \tcp{Update the population} 
        }
        \Return{$\mathcal{P}$} \tcp{Return the optimized population}
    \end{algorithm}
    
Next, we move onto the optimization notation. In Algorithm \ref{alg:genoptimloop} we can see the general outline of a population-based optimization method.
The initialization operator $\mathcal{O}_I$ creates an initial population of individuals $\mathcal{P}$. 
Then, in each step, a selection operator $\mathcal{O}_S$ selects a portion of the population according to some criteria. 
These selected individuals are then used by the expansion operator $\mathcal{O}_E$ to create new individuals.
This process continues until a termination condition $\Phi_{stop}$ is reached.

\chapter{Literature}


\section{Large Language Models}
\subsection{Brief history of NLP approaches}
The goal of this section is to familiarize the reader with the progress in the Natural Language Processing (NLP) field in the recent decade.

\subsubsection{Pre-transformer era}
\paragraph{Statistical NLP}
Data-driven methods such as Hidden Markov models, Conditional Random Fields and Max Entropy models are 
being used for part-of-speech tagging, named entity recognition, machine translation and speech recognition.
\paragraph{Word embeddings}
Algorithms that encode meaning of words in high-dimensional vectors allow models to understand words and relationships between them.
\paragraph{lstm, seq2seq, attention}
Attention allows models to connect key parts of input.

\subsubsection{Transformer era}
\paragraph{Attention is all you need}
Discovers that simplifying the architecture and focusing on the attention mechanisms allows for much better efficiency in training and paves way for a new era in NLP.
\paragraph{Pre-training+fine-tuning}
New paradigm where the language model is first pre-trained on an enormous corpus of data and later fine-tuned for specific tasks, like instruction-tuning for assistants.
\paragraph{multimodality}
Visual embedders allow LLMs to understand images. Embeddings are conserved under different modalities ("dog" and a picture of a dog have the same embedding).
\paragraph{Mixture-of-experts (MoE)}
More efficient architecture that allows for delegating of work to several "expert" submodels, resulting in sparse computation as only a fraction of the model parameters is activated.
\paragraph{Reinforcement learning}
Supervised Fine-tuning (SFT) has been combined or sometimes replaced altogether by various forms of Reinforcement learning (RL).
Proximal policy optimization based on human or AI feedback is the basis for assistant chat bots such as ChatGPT.
Novel RL approaches (GRPO) used to promote reasoning.
\paragraph{Inference-time compute}
Development of reasoning models is converging on the idea that letting the model spend more compute time on each answer leads to better results.
Using SFT and/or RL the model is taught to "think" or show its "inner monologue" as a part of the answer inside a designated "<think>" tag.
When leaving the thought chain, the "</think>" tag can be substituted by an introspective question like "Wait, did I forget something?" resulting in a prolonged thinking chain. 
potentially catching errors.
\paragraph{Overview of best current models}
Current models, with sizes around 1 trillion parameters, match or surpass average human performance on many benchmarks including math and coding.



\section{Prompting techniques}
\subsection{Prompt engineering}
Motivation for prompt engineering is to improve the model's capabilities not by changing the underlying weights with training on data but by crafting an optimal instruction string, or a prompt.
This can be done by providing examples of the task as a part of the prompt or by instructing the model how to solve the task.
In its essence, the model is a left-to-right text completion engine. We can make the analogy with human thinking modes, where it is said that humans
have a fast automatic "System 1" mode and a slow and deliberate "System 2" mode \cite{yao2023treethoughtsdeliberateproblem}. 
With a good prompt we can shift the model from "System 1" to "System 2".
\subsubsection{In-context learning}
Prompts are distinguished based on the number of included examples.
\begin{table}[h!]
    \centering
    \begin{tabular}{|c|p{12cm}|}
    \hline
    \textbf{Prompting Type} & \textbf{Description} \\
    \hline
    Zero-shot Prompting & Prompt has no examples, model relies on its pre-trained knowledge. \\
    \hline
    One-shot Prompting & Prompt has one example to guide the model. \\
    \hline
    Few-shot Prompting & Prompt includes a few examples (typically 2 to 5). \\
    \hline
    \end{tabular}
    \caption{Comparison of Zero-shot, One-shot, and Few-shot Prompting}
\end{table}        
Research\cite{brown2020languagemodelsfewshotlearners} has shown that with growing model size the knowledge-generalizing ability of the model increases. Instead of expensive fine-tuning
models can reuse knowledge from pre-training and solve many tasks when provided just by a few examples.
\subsection{Prompting techniques}
\subsubsection{Chain-of-thought (CoT)}
Instructing the model to "think step-by-step" results significant improvements on many benchmarks. Some research has shown that this approach hurts performance on some tasks where humans perform better without thinking.
Apart from more compute time being allocated at inference, the model also benefits from having its whole reasoning chain as a part of the context when writing the final answer to the task.
Several variants exist, such as CoT with self-consistency, where the final answer is acquired by majority-voting from several thinking chains.
\subsubsection{Reflexion}
Model reflects on its response and improves it.
\subsubsection{ReAct}
Multi-turn prompting technique that forms the basis of agentic LLMs. The model is given a set of tools, such as a Wikipedia search function or a math expression evaluator.
The model can go through several steps of using the tools, which generates "observation". The model uses these observations to generate a final answer and leaves the ReAct chain when ready using a "finish" function.
\subsubsection{Tree-of-thought}
Similar to CoT with self-consistency but the reasoning chain is split up into steps creating a tree. This reasoning tree can then be explored using a graph search algorithm such as DFS.



\section{Optimization methods}
\subsection{Basics}
Optimization is the search for the optimum (maximum or minimum) of an arbitrary function. 
We can divide optimization into two categories based on the nature of the decision variables:
\begin{enumerate}
    \item continuous
    \item discrete.
\end{enumerate}
\subsection{Continuous optimization methods}
Gradient descent, Newton's method, EAs
\subsection{Discrete optimization methods}
Hill-climber, search methods, GAs


\section{Prompt optimization}
\subsection{Soft prompt tuning}
Prompts for models which allow access to gradients, which is not the case for proprietary models accessed via APIs, can be optimized in the high-dimensional embedding space.

This makes the optimization problem continous. Soft prompts however pose the problem of interpretability and are non-transferable across different LLMs \cite{deng2022rlpromptoptimizingdiscretetext}.

Continuous prompt-optimization techniques, although effective, require parameters of LLMs inaccessible to black-box APIs and often fall short of interpretability. \cite{guo2024connectinglargelanguagemodels}

\subsection{Discrete prompt tuning}
The area of optimizing prompts discretely while utilizing language models as optimization operators has attracted significant research interest in recent years.

Natural language prompt engineering is particularly interesting because it is a natural interface for humans to communicate with machines, but plain language prompts do not always produce the desired result. \cite{zhou2023largelanguagemodelshumanlevel}

Natural language program synthesis search space is infinitely large. \cite{zhou2023largelanguagemodelshumanlevel}


Meta-prompts are flexible but studies lack principled guidelines about their design. \cite{tang2024unleashingpotentiallargelanguage}

Reproduces key model parameter learning factors - update direction and update method - in LLMs to seek theoretical foundations. \cite{tang2024unleashingpotentiallargelanguage}

OPRO\cite{yang2024largelanguagemodelsoptimizers} and APO\cite{pryzant2023automaticpromptoptimizationgradient} introduced analogical "gradient" forms. \cite{tang2024unleashingpotentiallargelanguage}

Analogical momentum forms inspired by the momentum method involve including the optimization trajectory in the meta-prompt. To fit into the context limit and reduce noise, trajectory can be summarized or k most recent/relevant/important gradients can be retrieved. \cite{tang2024unleashingpotentiallargelanguage}

To mimic effects or learning rate, prompt variation can be limited by edit distance (maximum words to be changed). Warm-up and decay strategies can be applied to this constraint. \cite{tang2024unleashingpotentiallargelanguage}

New prompt can be created by editing a previous prompt or generate a new one by following a demonstration. \cite{tang2024unleashingpotentiallargelanguage}

In an experiment on BBH, authors found that optimization without reflection performs better and the best momentum method being relevance. For prompt variation control, the best combination was cosine decay and no warm-up.  \cite{tang2024unleashingpotentiallargelanguage}

Summarization-based trajectory is less helpful because it tends to only capture common elements. \cite{tang2024unleashingpotentiallargelanguage}

Task input-output examples are beneficial in the meta-prompt to provide additional context to the LLM to understand the task. \cite{tang2024unleashingpotentiallargelanguage}

GPT-4 can consistently find better task prompts than GPT-3.5-turbo, which suggests the need for a capable model as the prompt optimizer \cite{tang2024unleashingpotentiallargelanguage}

Trajectory-based methods perform very well possible because trajectory helps the prompt optimizer pay more attention to the important information instead of the noise in the current step. \cite{tang2024unleashingpotentiallargelanguage}

\textbf{APE}
LLMs are used to construct a good set of candidate solutions by inferring the most likely instructions from input/output demonstrations. \cite{zhou2023largelanguagemodelshumanlevel}

Local search around the best candidates by resampling - asking the LLM to paraphrase the candidate prompt - this however only provides marginal improvements over just choosing the best-performing prompt from instruction induction. \cite{zhou2023largelanguagemodelshumanlevel}

APE was used to improve on Zero-Shot-CoT \cite{NEURIPS2022_8bb0d291} universal "Let's think by step" prompt"on GSM8k.\cite{zhou2023largelanguagemodelshumanlevel}


Prompt to the LLM optimizer is called the meta-prompt and includes previous prompts with their training accuracies sorted in ascending order along with the task description and training set samples. \cite{yang2024largelanguagemodelsoptimizers}

The main advantage of LLMs for optimization is their ability of understanding natural language, which allows people to describe their optimization tasks without formal specifications. \cite{yang2024largelanguagemodelsoptimizers}

Motivated by linear regression and TSP and on small-scale traveling salesman problems, OPRO performs on par with some hand-crafted heuristic algorithms. \cite{yang2024largelanguagemodelsoptimizers}

Optimization stability can be improved by generating multiple solutions when relying on random ICL samples. \cite{yang2024largelanguagemodelsoptimizers}

To balance between exploration and exploitation, LLM sampling temperature can be tuned. Lower temperature encourages exploitation in the local solution space and higher temperature allows more aggressive exploration of different solutions. \cite{yang2024largelanguagemodelsoptimizers}

Only the top instructions are kept in the meta-prompt to fit in the LLM context limit. \cite{yang2024largelanguagemodelsoptimizers}

New outstanding solution is usually found only all the prompts are of similar quality: first all the worse prompts are purged and substituted by a prompt similar to the current best. \cite{yang2024largelanguagemodelsoptimizers}

Semantically similar instructions have vastly different performance on GSM8k: “Let’s think step by step.” achieves accuracy 71.8, “Let’s solve the problem together.” has accuracy 60.5, while the accuracy of “Let’s work together to solve this problem step by step.” is only 49.4. \cite{yang2024largelanguagemodelsoptimizers}

\subsubsection{Textual gradients}
Naturally there are no gradients in the text space but some researchers try to emulate them using reflection-based operators.

APO mirrors the steps of gradient descent within a text-based Socratic dialogue substituting differentiation with LLM feedback and backpropagation with LLM editing \cite{pryzant2023automaticpromptoptimizationgradient}

Beam search is an iterative optimization process where in current prompt is expanded into many more candidates in each iteration and a selection process decides which will be used in the next iteration. \cite{pryzant2023automaticpromptoptimizationgradient}

Expansion first uses gradients to edit the current prompt and then explores the local monte-carlo search space by paraphrasing the editions \cite{pryzant2023automaticpromptoptimizationgradient}

To limit the computation used on evaluating prompts, an approach inspired by best arm identification in bandit optimization is utilized. \cite{pryzant2023automaticpromptoptimizationgradient}


Applying previous iterative prompt optimization methods, based on prompt+score pairs, to text generation tasks is challenging due to the lack of effective optimization signals. \cite{he2024crispomultiaspectcritiquesuggestionguidedautomatic}

Critiques and suggestions, written in natural language, are more helpful for prompt improvement than a single score.\cite{he2024crispomultiaspectcritiquesuggestionguidedautomatic}

CriSPO uses prompt+score+critique triples for next candidate generation. \cite{he2024crispomultiaspectcritiquesuggestionguidedautomatic}
Unlike APE \cite{pryzant2023automaticpromptoptimizationgradient} prompt generation is decoupled from suggestions and a history of critiques and suggestions as packed into the optimizer for a more stable optimization. \cite{he2024crispomultiaspectcritiquesuggestionguidedautomatic}

CoT is applied in optimization by first asking to compare high-score prompts to low-score ones and draft general ideas. \cite{he2024crispomultiaspectcritiquesuggestionguidedautomatic}

Critique-based optimization explores a larger space, which is indicated by lower similarity of the prompts in lexicons and semantics.\cite{he2024crispomultiaspectcritiquesuggestionguidedautomatic}

CriSPO outperforms OPRO \cite{yang2024largelanguagemodelsoptimizers} both on summarization and QA tasks and metaprompt allows for creating ICL and RAG template prompts. \cite{he2024crispomultiaspectcritiquesuggestionguidedautomatic}

\textbf{DSPy optimizers}

Most prompt optimizer approaches do not apply to multi-stage LLM programs where we lack gold labels or evaluation metrics for individual LLM calls. \cite{opsahlong2024optimizinginstructionsdemonstrationsmultistage}

Proposing a few high-quality instructions is essential due to the intractably large search space. \cite{opsahlong2024optimizinginstructionsdemonstrationsmultistage}

Uses a surrogate Bayesian optimization model, which is updated periodically by evaluating the program on batches, to sample instructions and demonstrations for each stage of the LLM program \cite{opsahlong2024optimizinginstructionsdemonstrationsmultistage}

Optimizing demonstrations alone usually yields better performance than just optimizing instructions, but optimizing both yield the best performance. \cite{opsahlong2024optimizinginstructionsdemonstrationsmultistage}

Optimizing instructions is most valuable for tasks with subtle conditional rules not expressible by a few examples.  \cite{opsahlong2024optimizinginstructionsdemonstrationsmultistage}

For LLM programs, it is beneficial to alternate between optimizing weights (fine-tuning) and optimizing prompts. \cite{soylu2024finetuningpromptoptimizationgreat}

\subsubsection{Evolutionary optimization}
Building upon the inherent ability of LLMs to paraphrase (mutation) and combine (crossover) text, an interesting intersection of traditional evolutionary algorithms and modern LLMs has formed. 


Sequences of phrases can be regarded as gene sequences in typical Evolutionary algorithms. \cite{guo2024connectinglargelanguagemodels}


Considers two widely used EAs: Genetic Algorithm and Differential Evolution with DE outperforming GA on most tasks \cite{guo2024connectinglargelanguagemodels}

Initial population consists of manually-written prompts to leverage human knowledge as well as some prompts generated by LLMs to reflect the fact that EAs start from random solutions to avoid local optima. \cite{guo2024connectinglargelanguagemodels}

DE-inspired approached builds on the idea that the common elements of the current best prompts need to be preserved \cite{guo2024connectinglargelanguagemodels}

Evoprompt performs best with roulette selection when compared with tournament and random selection. \cite{guo2024connectinglargelanguagemodels}

Similar results are achieved when population is initialized with the best and with random prompts, hinting that the crafted design of initial prompts is not essential. \cite{guo2024connectinglargelanguagemodels}


Previous research optimized zero-shot instructions and examples separately, overlooking their interplay and resulting in sub-optimal performance. \cite{cui2024phaseevounifiedincontextprompt}

There is a prevailing notion that prompt engineering sacrifices efficiency for performance due to the lengthening of prompts, but PhaseEvo actively shortens the prompts \cite{cui2024phaseevounifiedincontextprompt}

Current EA applications to prompt optimization suffer from extremely high computational cost and slow convergence speed due to the complexity of the high-dimensional search space. \cite{cui2024phaseevounifiedincontextprompt}

PhaseEvo alternates between two phases: exploration with evolution operators and exploitation using a feedback "gradient". \cite{cui2024phaseevounifiedincontextprompt}

TABLE 1 \todo{recreate} compares all 5 operators.  \cite{cui2024phaseevounifiedincontextprompt}

4 phases: initialization - lamarck or manual, local feedback mutation, global evolution with EDA and CR operators, local semantic mutation (paraphrasing) \cite{cui2024phaseevounifiedincontextprompt}

Candidates for evolution operators are selected based on a "performance vector", combining prompts that do not make the same mistakes.  \cite{cui2024phaseevounifiedincontextprompt}

When the performance improvement with an operator stagnates up to some operator-specific tolerance, the current phase is terminated. \cite{cui2024phaseevounifiedincontextprompt}

Evolution in phases outperforms random operator selection. \cite{cui2024phaseevounifiedincontextprompt}

PhaseEvo is the most cost-effective but still needs around 12 iterations and 4000 API calls. \cite{cui2024phaseevounifiedincontextprompt}


APE \cite{zhou2023largelanguagemodelshumanlevel} ran into problems with diminishing returns and abandoning the iterative approach entirely, Promptbreeder aims to solve this with a diversity-maintaining evolutionary algorithm for self-referential self-improvement of prompts \cite{fernando2023promptbreederselfreferentialselfimprovementprompt}

Prompt optimization techniques utilize the fact that LLMs are effective at generating mutations from examples and can encode human notions of interestingness and can be used to quantify novelty. \cite{fernando2023promptbreederselfreferentialselfimprovementprompt}

Self-referential system should improve the way it is improving, thus Promptbreeder used a "hyper-prompt" to optimize its meta-prompt \cite{fernando2023promptbreederselfreferentialselfimprovementprompt}

Uses a binary tournament genetic algorithm. \cite{fernando2023promptbreederselfreferentialselfimprovementprompt}

Uses a random uniformly sampled mutation operators out of 9 total from 5 broad categories for each replication event. \cite{fernando2023promptbreederselfreferentialselfimprovementprompt}

Zero-order mutation (creating a prompt from task description) generates new task prompts more aligned with the task description in the event the evolution diverges.  \cite{fernando2023promptbreederselfreferentialselfimprovementprompt}

LLMs tend to be biased to examples found later in EDA mutation lists. Lying to the LLM and telling it that the prompts are sorted by performance in a descending order improves diversity.  \cite{fernando2023promptbreederselfreferentialselfimprovementprompt}

Removing any self-referential operator in ablation is harmful under nearly all circumstances \cite{fernando2023promptbreederselfreferentialselfimprovementprompt}


\subsubsection{Metaprompting}
Metaprompting or "prompting to create prompts". Research shows that meta-prompting will always be superior to prompting through category theory\cite{dewynter2024metaprompting}.

\chapter{Methodology}


\chapter{Experiments}

\appendix

\printindex


\bibliographystyle{ieeetr}
\bibliography{references}

%\ctutemplate{specification.as.chapter}
\end{document}