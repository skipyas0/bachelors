% arara: pdflatex: { synctex: yes }
% arara: makeindex: { style: ctuthesis }
% arara: bibtex

% The class takes all the key=value arguments that \ctusetup does,
% and a couple more: draft and oneside
\documentclass[twoside]{ctuthesis}

\ctusetup{
	preprint = \ctuverlog,
	mainlanguage = english,
	titlelanguage = english,
	otherlanguages = {czech},
	title-czech = {abcd},
	title-english = {Meta-prompts for LLM Prompt Optimization},
	subtitle-czech = {abcd},
	subtitle-english = {abcd},
	doctype = B,
	faculty = F3,
	department-english = {Artificial Intelligence Center},
	department-czech = {Centrum pro umělou inteligenci},
	author = {Vojtěch Klouda},
	supervisor = {Ing. Jan Drchal PhD.},
	supervisor-address = {Resslova 307/9 Praha, E-322},
%	supervisor-specialist = {John Doe},
	fieldofstudy-english = {Artificial Intelligence},
	subfieldofstudy-english = {Natural Language Processing},
	fieldofstudy-czech = {Umělá inteligence},
	subfieldofstudy-czech = {Zpracování přirozeného jazyka},
	keywords-czech = {jazykový model, optimalizace},
	keywords-english = {language model, optimization},
	day = 10,
	month = 5,
	year = 2025,
%	specification-file = {ctutest-zadani.pdf},
%	front-specification = true,
%	front-list-of-figures = false,
%	front-list-of-tables = false,
%	monochrome = true,
%	layout-short = true,
}

\ctuprocess

\addto\ctucaptionsczech{%
	\def\supervisorname{Vedoucí}%
	\def\subfieldofstudyname{Studijní program}%
}

\ctutemplateset{maketitle twocolumn default}{
	\begin{twocolumnfrontmatterpage}
		\ctutemplate{twocolumn.thanks}
		\ctutemplate{twocolumn.declaration}
		\ctutemplate{twocolumn.abstract.in.titlelanguage}
		\ctutemplate{twocolumn.abstract.in.secondlanguage}
		\ctutemplate{twocolumn.tableofcontents}
		\ctutemplate{twocolumn.listoffigures}
	\end{twocolumnfrontmatterpage}
}

% todo command
\newcommand{\todo}[1]{\textsuperscript{\textbf{\textcolor{red}{#1}}}}
\newcommand{\vsep}{\, \vert \,}

\newcommand{\maxmean}[2]{#1 {\color{gray}\scriptsize (#2)}}


% Theorem declarations, this is the reasonable default, anybody can do what they wish.
% If you prefer theorems in italics rather than slanted, use \theoremstyle{plainit}
\theoremstyle{plain}
\newtheorem{theorem}{Theorem}[chapter]
\newtheorem{corollary}[theorem]{Corollary}
\newtheorem{lemma}[theorem]{Lemma}
\newtheorem{proposition}[theorem]{Proposition}

\theoremstyle{definition}
\newtheorem{definition}[theorem]{Definition}
\newtheorem{example}[theorem]{Example}
\newtheorem{conjecture}[theorem]{Conjecture}

\theoremstyle{note}
\newtheorem*{remark*}{Remark}
\newtheorem{remark}[theorem]{Remark}

\setlength{\parskip}{5ex plus 0.2ex minus 0.2ex}

% Abstract in Czech
\begin{abstract-czech}
	TODO
\end{abstract-czech}

% Abstract in English
\begin{abstract-english}
	TODO
\end{abstract-english}

% Acknowledgements / Podekovani
\begin{thanks}
	= )))
\end{thanks}

% Declaration / Prohlaseni
\begin{declaration}
Prohlašuji, že jsem předloženou práci vypracoval samostatně, a že jsem uvedl veškerou použitou literaturu.

V Praze, \ctufield{day}.~\monthinlanguage{title}~\ctufield{year}
\end{declaration}

% Only for testing purposes
\listfiles
\usepackage[pagewise]{lineno}
\usepackage{lipsum,blindtext}
\usepackage{mathrsfs} % provides \mathscr used in the ridiculous examples

\begin{document}

\maketitle

\chapter{Introduction}
\section{Background}
In recent years, Large Language Models (LLMs) have permeated the Natural Language Processing research landscape as well as into the general public. 
Already achieving human-like performance at a wide variety of tasks\cite{bubeck2023sparksartificialgeneralintelligence}, 
they are bound by scaling laws\cite{kaplan2020scalinglawsneurallanguage} which predict performance gained with adding compute, fueling
massive investments into computation capacity by industry players. 

With costs of training new state-of-the-art foundational LLMs rising rapidly, research has turned to inference-time scaling\cite{welleck2024decodingmetagenerationinferencetimealgorithms}, 
based on post-training\cite{openai2024openaio1card}\cite{deepseekai2025deepseekr1incentivizingreasoningcapability} utilizing reinforcement learning and supervised fine-tuning, and 
prompting techniques\cite{schulhoff2024promptreportsystematicsurvey}. 

Another research branch gaining substantial attention recently is compile-time scaling\cite{schnabel2024symbolicpromptprogramsearch} represented by prompt optimization\cite{ramnath2025systematicsurveyautomaticprompt}.
Optimization using LLMs\cite{meyerson2024languagemodelcrossovervariation}\cite{liu2024largelanguagemodelsevolutionary} and particularly prompt optimization\cite{yang2024largelanguagemodelsoptimizers}\cite{zhou2023largelanguagemodelshumanlevel}\cite{he2024crispomultiaspectcritiquesuggestionguidedautomatic} 
presents an exciting intersection between deep learning and traditional optimization algorithms, like evolutionary algorithms\cite{guo2024connectinglargelanguagemodels}\cite{cui2024phaseevounifiedincontextprompt}\cite{fernando2023promptbreederselfreferentialselfimprovementprompt} and other metaheuristics\cite{pan2024plumpromptlearningusing}.



\chapter{Literature}


\section{Large Language Models}
\subsection{Brief history of NLP approaches}
The goal of this section is to familiarize the reader with the progress in the Natural Language Processing (NLP) field in the recent decade.

\subsubsection{Pre-transformer era}
\paragraph{Statistical NLP}
Data-driven methods such as Hidden Markov models, Conditional Random Fields and Max Entropy models are 
being used for part-of-speech tagging, named entity recognition, machine translation and speech recognition.
\paragraph{Word embeddings}
Algorithms that encode meaning of words in high-dimensional vectors allow models to understand words and relationships between them.
\paragraph{lstm, seq2seq, attention}
Attention allows models to connect key parts of input.

\subsubsection{Transformer era}
\paragraph{Attention is all you need}
Discovers that simplifying the architecture and focusing on the attention mechanisms allows for much better efficiency in training and paves way for a new era in NLP.
\paragraph{Pre-training+fine-tuning}
New paradigm where the language model is first pre-trained on an enormous corpus of data and later fine-tuned for specific tasks, like instruction-tuning for assistants.
\paragraph{multimodality}
Visual embedders allow LLMs to understand images. Embeddings are conserved under different modalities ("dog" and a picture of a dog have the same embedding).
\paragraph{Mixture-of-experts (MoE)}
More efficient architecture that allows for delegating of work to several "expert" submodels, resulting in sparse computation as only a fraction of the model parameters is activated.
\paragraph{Reinforcement learning}
Supervised Fine-tuning (SFT) has been combined or sometimes replaced altogether by various forms of Reinforcement learning (RL).
Proximal policy optimization based on human or AI feedback is the basis for assistant chat bots such as ChatGPT.
Novel RL approaches (GRPO) used to promote reasoning.
\paragraph{Inference-time compute}
Development of reasoning models is converging on the idea that letting the model spend more compute time on each answer leads to better results.
Using SFT and/or RL the model is taught to "think" or show its "inner monologue" as a part of the answer inside a designated "<think>" tag.
When leaving the thought chain, the "</think>" tag can be substituted by an introspective question like "Wait, did I forget something?" resulting in a prolonged thinking chain. 
potentially catching errors.
\paragraph{Overview of best current models}
Current models, with sizes around 1 trillion parameters, match or surpass average human performance on many benchmarks including math and coding.



\section{Inference-time scaling}

Language models are probabilistic models over sequences and most generation algorithms attempt to either find highly probable sequences or sample from the model’s distribution. \cite{welleck2024decodingmetagenerationinferencetimealgorithms}

MAP decoding algorithms' objective is to choose the most likely sequence. \cite{welleck2024decodingmetagenerationinferencetimealgorithms}

The simplest but still widely used MAP algorithm is Greedy decoding that recursively selects the highest probability token from the next-token distribution. \cite{welleck2024decodingmetagenerationinferencetimealgorithms}

Beam search improves on greedy decoding in many settings but incurs a high computational cost. \cite{welleck2024decodingmetagenerationinferencetimealgorithms}

MAP decoding generations often fall outside of the typical set of sequences in the language model’s distribution. \cite{welleck2024decodingmetagenerationinferencetimealgorithms}

Alternative to MAP is to sample directly from the language model's distribution. \cite{welleck2024decodingmetagenerationinferencetimealgorithms}

Decoding strategies like nucleus, top-k and η- and ϵ-sampling interpolate between greedy and ancestral sampling while temperature sampling interpolates between greedy and uniform sampling. \cite{welleck2024decodingmetagenerationinferencetimealgorithms}

Next-token distributions are usually not provided by APIs but instead token-level algorithms are implemented by the API provider and used by setting hyper-parameters. \cite{welleck2024decodingmetagenerationinferencetimealgorithms}

Temperature sampling usually outperformed the other adapters in input-output tasks such as code generation and translation but in general, which adapter to use remains an open question. \cite{welleck2024decodingmetagenerationinferencetimealgorithms}

Parser-based decoding can enforce a structural requirement, such as following a JSON schema, sometimes hindering performance with inflexible templates.  \cite{welleck2024decodingmetagenerationinferencetimealgorithms}

Meta-generators, or strategies that utilize sub-generators, can be divided into the categories of chained, parallel, step-level, and refinement-based meta-generators. \cite{welleck2024decodingmetagenerationinferencetimealgorithms}



\subsection{Chained meta-generation}
Chain-of-thought can be seen as a motivating example of chained meta-generation. \cite{welleck2024decodingmetagenerationinferencetimealgorithms}

\subsubsection{Chain-of-thought (CoT)}

Longer CoTs do not consistently improve accuracy of o1-like models especially for weaker models like QwQ ans R1-Distill-1.5b. \cite{zeng2025revisitingtesttimescalingo1like}

Average length of correct solutions is shorter than that of incorrect ones for the same questions. \cite{zeng2025revisitingtesttimescalingo1like}

Models demonstrate limited ability to correct their answers during revision, with most revisions retaining the original answers or sometimes even changing correct answers for incorrect ones. \cite{zeng2025revisitingtesttimescalingo1like}

Self-revision ability is a key factor in the effectiveness of sequential scaling for o1-like models. \cite{zeng2025revisitingtesttimescalingo1like}

Models similar to o1 all primarily extend solution length by self-revision, which is characterized by use of markers such as "Wait" or "Alternatively". \cite{zeng2025revisitingtesttimescalingo1like}

\textbf{Principles}

Chain-of-Thought (CoT) is a LLM prompting technique that works by inducing a coherent series of intermediate 
reasoning steps that lead to the final answer for a problem\cite{wei2023chainofthoughtpromptingelicitsreasoning}.
We differentiate between Zero-Shot CoT\cite{NEURIPS2022_8bb0d291} and Few-Shot CoT\cite{wei2023chainofthoughtpromptingelicitsreasoning}.

CoT allows models to allocate additional computation to problems with more reasoning steps (inference-time scaling) \cite{wei2023chainofthoughtpromptingelicitsreasoning}


Examples of CoT reasoning in the prompt in a one/few-shot setting to facilitate a reasoning chain response. \cite{wei2023chainofthoughtpromptingelicitsreasoning}

CoT prompting is an emergent ability of model scale, does not positively impact performance for small models \cite{wei2023chainofthoughtpromptingelicitsreasoning}

CoT can been elicited by prompting techniques - few-shot with steps demonstrations or zero-shot with specific instructions \cite{wang2024chainofthoughtreasoningprompting}

Few-shot-CoT requires human engineering of multi-step reasoning prompts and their performance deteriorates if prompt example and task question types are unmatched, suggesting high sensitivity to prompt design. \cite{NEURIPS2022_8bb0d291}

LLMs are decent zero-shot reasoners by adding a simple "Let's think step-by-step" prompt, which is versatile and task-agnostic. Similar prompts that encourage reasoning also improve performance, which leaves the question of how to design better Zero-shot-CoT templates. \cite{NEURIPS2022_8bb0d291}

Existing work suggest LLMs falter in direct-QA scenarios.  \cite{wang2024chainofthoughtreasoningprompting}
Greedy decoding in a direct-QA scenario often does not contain a CoT path, which may stem from the model's skewed perception of problem difficulty as it might have been trained on simpler questions \cite{wang2024chainofthoughtreasoningprompting}

Direct prediction is inaccurate for some inferences because the relevant variables are rarely seen together in training. \cite{prystawski2023thinkstepstepreasoning}

Chain-of-thought reasoning improves estimation by incrementally chaining local statistical dependencies that are observed frequently in training. \cite{prystawski2023thinkstepstepreasoning}

\textbf{CoT without prompting}

CoT can been elicited by model training or tuning with significant amount of reasoning data \cite{wang2024chainofthoughtreasoningprompting}

Exploring top-k, k>0 tokens at the first decoding step and continuing with greedy reveals natural CoT reasoning in many cases 
with the resulting answer having much higher confidence and the decoding path with the highest answer confidence 
among the top-10 decoding paths contains a CoT path in 88\% of the GSM8k samples. CoT-decoding can be easily combined with 
CoT-prompting or weighted aggregation similar to self-consistency, yielding even larger reasoning gains over multiple language models \cite{wang2024chainofthoughtreasoningprompting}


\textbf{Applicability}

More performance gains for more complicated problems. \cite{wei2023chainofthoughtpromptingelicitsreasoning}

Zero-shot-CoT outperforms simple Zero-shot on arithmetic tasks but does not provide performance gains on commonsense reasoning tasks. \cite{NEURIPS2022_8bb0d291}

CoT can impact performance on tasks where verbal thinking and deliberation hurts performance in humans but for some
it is less clear if they should generalize to LLMs \cite{liu2024mindstepbystep}

Tasks where human limitations generalize to LLMs:
Implicit statistical learining - 36\% performance decrease from gpt-4o to o1-preview
Facial recognition, classifying data with patterns and exceptions - performance drops \cite{liu2024mindstepbystep}

Tasks where human limitations do not generalize a significant performance decrease is not seen or performance increases:
explaining a logical inconsistency:  assumes a reasonable baserate in recognizing logical inconsistencies
spatial intuition: consequence of information encoded in visual and motor representations that are likely not available to models 
aggregating features for a decision, complex, multi-dimensional tasks that exceed human working memory capacity but can easily fit in the context window of an LLM \cite{liu2024mindstepbystep}

More advanced inference-time reasoning techniques like ToT alleviate performance a little but still underperform zero-shot. \cite{liu2024mindstepbystep}

\textbf{Self-consistency}

Self-consistency aggregates answers from diverse reasoning chains and selects the best one based on majority voting. \cite{wang2023selfconsistencyimproveschainthought}

It significantly improves accuracy in a range of arithmetic and commonsense reasoning tasks and is useful for collecting rationales and providing uncertainty estimates. \cite{wang2023selfconsistencyimproveschainthought}

Works with both few-shot and zero-shot CoT. \cite{wang2023selfconsistencyimproveschainthought}

One limitation of self-consistency is that it incurs more computation cost. Small number of paths is enough, as in most cases the performance saturates quickly. \cite{wang2023selfconsistencyimproveschainthought}


\todo{move CoT stuff here}

\subsection{Parallel meta-generation}

Parallel meta-generation involves multiple generations, with one being chosen based on some reward model or with voting. Alternatively, the results can be merged into a single final answer with the language model. \cite{welleck2024decodingmetagenerationinferencetimealgorithms}


Prompting techniques like chain-of-thought can increase answer quality at the cost of longer and more computationally expensive outputs. \cite{brown2024largelanguagemonkeysscaling}

Generating large sample collections is only useful if the correct samples in a collection can be identified. \cite{brown2024largelanguagemonkeysscaling}

It is possible and sometimes cost-effective, to amplify weaker models with many samples and outperform single samples from more capable models. \cite{brown2024largelanguagemonkeysscaling}

Relationship between coverage and the number of samples can often be modeled using an exponentiated power law, suggesting a form of scaling laws for inference-time compute although not as exact as training scaling laws. \cite{brown2024largelanguagemonkeysscaling}

Repeated sampling can make use of high batch sizes and specialized optimizations that improve system throughput relative to single-attempt inference workloads. \cite{brown2024largelanguagemonkeysscaling}

Coverage and precision diverge as number of samples increases on tasks without automatic verifiers, highlighting the need for improving sample verification methods. \cite{brown2024largelanguagemonkeysscaling}


All parallel scaling methods rely on guidance signals to select the optimal token, step, or solution from a set of candidates. \cite{zeng2025revisitingtesttimescalingo1like}

For the same number of generated tokens, parallel scaling provides a significantly larger improvement in coverage compared to sequential scaling. \cite{zeng2025revisitingtesttimescalingo1like}

Practical parallel scaling method must select a final answer from a set of candidate answers. \cite{zeng2025revisitingtesttimescalingo1like}

Shortest majority vote takes into account the fact that correct solution chains are shorter on average and extends Majority vote by weighing solution counts with the log of average solution length for given solution category, outperforming Majority voting on AIME. \cite{zeng2025revisitingtesttimescalingo1like}

\subsection{Step-level meta-generation}
Step-level meta-generation implements search algorithms on the generation state-space, which can be made up of tokens or longer sequences. Many search algorithms and state evaluation functions are possible. \cite{welleck2024decodingmetagenerationinferencetimealgorithms}


Multi-turn inference-time methods with a fixed width exhibit diminishing gains when computational budget is increased, failing to leverage the vast output space of LLMs.\cite{misaki2025widerdeeperscalingllm} 

Unlike the standard tasks typically tackled by tree search algorithms where the number of possible actions at each node is finite, each call to LLM can yield a new output even for the same input, making each node’s branching factor theoretically infinite.\cite{misaki2025widerdeeperscalingllm} 


\subsubsection{Tree-of-thought}
Similar to CoT with self-consistency but the reasoning chain is split up into steps creating a tree. This reasoning tree can then be explored using a graph search algorithm such as DFS.


\subsection{Refinement meta-generation}
Refinement meta-generation generates a revised version of the output based on past versions and additional information such as intrinsic or extrinsic feedback or environment observations. \cite{welleck2024decodingmetagenerationinferencetimealgorithms}
For extrinsic refinement, it is plausible that there are information sources which add new information, and hence lead to a potential gain with refinement but the efficacy of intrinsic refinement has been mixed. \cite{welleck2024decodingmetagenerationinferencetimealgorithms}


Inference-Time Scaling by training models to think before responding is insufficient because these methods include Reinforcement learning with verifiers, making them unsuitable for open-ended tasks. \cite{wang2025dedicatedfeedbackeditmodels}

Authors train dedicated Feedback and Edit models that can be used at inference time to improve model responses to open-ended general domain tasks. \cite{wang2025dedicatedfeedbackeditmodels}

Efficacy of LLMs in providing feedback and making edits to their own responses is unclear as using LLMs that were not specifically trained to provide feedback is ineffective compared to using high quality feedback. \cite{wang2025dedicatedfeedbackeditmodels}


\subsubsection{Reflexion}
Reflexion converts binary or scalar feedback from the environment into verbal feedback in the form 
of a textual summary, which is then added as additional context for the LLM agent, e.g. CoT or ReAct module, in the next episode. \cite{shinn2023reflexionlanguageagentsverbal}

Reflections go into a long-term memory context limited to a sliding window with maximum capacity. \cite{shinn2023reflexionlanguageagentsverbal}

Improves performance over strong baselines on sequential decision making, reasoning and programming tasks. \cite{shinn2023reflexionlanguageagentsverbal}
\subsubsection{ReAct}
Multi-turn prompting technique that forms the basis of agentic LLMs. The model is given a set of tools, such as a Wikipedia search function or a math expression evaluator.
The model can go through several steps of using the tools, which generates "observation". The model uses these observations to generate a final answer and leaves the ReAct chain when ready using a "finish" function.


\section{Prompting techniques}

Prompting techniques encode human priors, making it difficult to assess a language model's intrinsic reasoning abilities \cite{wang2024chainofthoughtreasoningprompting}

\subsection{Prompt engineering}
In many modern LLM applications, prompts have become programs themselves. \cite{schnabel2024symbolicpromptprogramsearch}
Motivation for prompt engineering is to improve the model's capabilities not by changing the underlying weights with training on data but by crafting an optimal instruction string, or a prompt.
This can be done by providing examples of the task as a part of the prompt or by instructing the model how to solve the task.
In its essence, the model is a left-to-right text completion engine. We can make the analogy with human thinking modes, where it is said that humans
have a fast automatic "System 1" mode and a slow and deliberate "System 2" mode \cite{yao2023treethoughtsdeliberateproblem}. 
With a good prompt we can shift the model from "System 1" to "System 2".
\subsubsection{In-context learning}
Prompts are distinguished based on the number of included examples.
\begin{table}[h!]
    \centering
    \begin{tabular}{|c|p{12cm}|}
    \hline
    \textbf{Prompting Type} & \textbf{Description} \\
    \hline
    Zero-shot Prompting & Prompt has no examples, model relies on its pre-trained knowledge. \\
    \hline
    One-shot Prompting & Prompt has one example to guide the model. \\
    \hline
    Few-shot Prompting & Prompt includes a few examples (typically 2 to 5). \\
    \hline
    \end{tabular}
    \caption{Comparison of Zero-shot, One-shot, and Few-shot Prompting}
\end{table}        
Research\cite{brown2020languagemodelsfewshotlearners} has shown that with growing model size the knowledge-generalizing ability of the model increases. Instead of expensive fine-tuning
models can reuse knowledge from pre-training and solve many tasks when provided just by a few examples.

Few-shot prompting highlights that LLMs can be seen as powerful pattern-completion engines. \cite{meyerson2024languagemodelcrossovervariation}

Providing a prompt of examples from a distribution can condition the LLM to generate further high-probability examples from that distribution \cite{meyerson2024languagemodelcrossovervariation}

\subsection{Prompting techniques}
\todo{talk about how we can achieve meta-generation just by updating the prompt}



\section{Optimization methods}
\subsection{Basics}
Optimization is the search for the optimum (maximum or minimum) of an arbitrary function. 
We can divide optimization into two categories based on the nature of the decision variables:
\begin{enumerate}
    \item continuous
    \item discrete.
\end{enumerate}
\subsection{Continuous optimization methods}
Gradient descent, Newton's method, EAs
\subsection{Discrete optimization methods}
Hill-climber, search methods, GAs


\section{Prompt optimization}
\subsection{Soft prompt tuning}
Prompts for models which allow access to gradients, which is not the case for proprietary models accessed via APIs, can be optimized in the high-dimensional embedding space.

This makes the optimization problem continous. Soft prompts however pose the problem of interpretability and are non-transferable across different LLMs \cite{deng2022rlpromptoptimizingdiscretetext}.

Continuous prompt-optimization techniques, although effective, require parameters of LLMs inaccessible to black-box APIs and often fall short of interpretability. \cite{guo2024connectinglargelanguagemodels}

\subsection{Discrete prompt tuning}
The area of optimizing prompts discretely while utilizing language models as optimization operators has attracted significant research interest in recent years.

Natural language prompt engineering is particularly interesting because it is a natural interface for humans to communicate with machines, but plain language prompts do not always produce the desired result. \cite{zhou2023largelanguagemodelshumanlevel}

Natural language program synthesis search space is infinitely large. \cite{zhou2023largelanguagemodelshumanlevel}


Meta-prompts are flexible but studies lack principled guidelines about their design. \cite{tang2024unleashingpotentiallargelanguage}

Reproduces key model parameter learning factors - update direction and update method - in LLMs to seek theoretical foundations. \cite{tang2024unleashingpotentiallargelanguage}

OPRO\cite{yang2024largelanguagemodelsoptimizers} and APO\cite{pryzant2023automaticpromptoptimizationgradient} introduced analogical "gradient" forms. \cite{tang2024unleashingpotentiallargelanguage}

Analogical momentum forms inspired by the momentum method involve including the optimization trajectory in the meta-prompt. To fit into the context limit and reduce noise, trajectory can be summarized or k most recent/relevant/important gradients can be retrieved. \cite{tang2024unleashingpotentiallargelanguage}

To mimic effects or learning rate, prompt variation can be limited by edit distance (maximum words to be changed). Warm-up and decay strategies can be applied to this constraint. \cite{tang2024unleashingpotentiallargelanguage}

New prompt can be created by editing a previous prompt or generate a new one by following a demonstration. \cite{tang2024unleashingpotentiallargelanguage}

In an experiment on BBH, authors found that optimization without reflection performs better and the best momentum method being relevance. For prompt variation control, the best combination was cosine decay and no warm-up.  \cite{tang2024unleashingpotentiallargelanguage}

Summarization-based trajectory is less helpful because it tends to only capture common elements. \cite{tang2024unleashingpotentiallargelanguage}

Task input-output examples are beneficial in the meta-prompt to provide additional context to the LLM to understand the task. \cite{tang2024unleashingpotentiallargelanguage}

GPT-4 can consistently find better task prompts than GPT-3.5-turbo, which suggests the need for a capable model as the prompt optimizer \cite{tang2024unleashingpotentiallargelanguage}

Trajectory-based methods perform very well possible because trajectory helps the prompt optimizer pay more attention to the important information instead of the noise in the current step. \cite{tang2024unleashingpotentiallargelanguage}

\textbf{APE}
LLMs are used to construct a good set of candidate solutions by inferring the most likely instructions from input/output demonstrations. \cite{zhou2023largelanguagemodelshumanlevel}

Local search around the best candidates by resampling - asking the LLM to paraphrase the candidate prompt - this however only provides marginal improvements over just choosing the best-performing prompt from instruction induction. \cite{zhou2023largelanguagemodelshumanlevel}

APE was used to improve on Zero-Shot-CoT \cite{NEURIPS2022_8bb0d291} universal "Let's think by step" prompt"on GSM8k.\cite{zhou2023largelanguagemodelshumanlevel}


Prompt to the LLM optimizer is called the meta-prompt and includes previous prompts with their training accuracies sorted in ascending order along with the task description and training set samples. \cite{yang2024largelanguagemodelsoptimizers}

The main advantage of LLMs for optimization is their ability of understanding natural language, which allows people to describe their optimization tasks without formal specifications. \cite{yang2024largelanguagemodelsoptimizers}

Motivated by linear regression and TSP and on small-scale traveling salesman problems, OPRO performs on par with some hand-crafted heuristic algorithms. \cite{yang2024largelanguagemodelsoptimizers}

Optimization stability can be improved by generating multiple solutions when relying on random ICL samples. \cite{yang2024largelanguagemodelsoptimizers}

To balance between exploration and exploitation, LLM sampling temperature can be tuned. Lower temperature encourages exploitation in the local solution space and higher temperature allows more aggressive exploration of different solutions. \cite{yang2024largelanguagemodelsoptimizers}

Only the top instructions are kept in the meta-prompt to fit in the LLM context limit. \cite{yang2024largelanguagemodelsoptimizers}

New outstanding solution is usually found only all the prompts are of similar quality: first all the worse prompts are purged and substituted by a prompt similar to the current best. \cite{yang2024largelanguagemodelsoptimizers}

Semantically similar instructions have vastly different performance on GSM8k: “Let’s think step by step.” achieves accuracy 71.8, “Let’s solve the problem together.” has accuracy 60.5, while the accuracy of “Let’s work together to solve this problem step by step.” is only 49.4. \cite{yang2024largelanguagemodelsoptimizers}

\subsubsection{Textual gradients}
Naturally there are no gradients in the text space but some researchers try to emulate them using reflection-based operators.

APO mirrors the steps of gradient descent within a text-based Socratic dialogue substituting differentiation with LLM feedback and backpropagation with LLM editing \cite{pryzant2023automaticpromptoptimizationgradient}

Beam search is an iterative optimization process where in current prompt is expanded into many more candidates in each iteration and a selection process decides which will be used in the next iteration. \cite{pryzant2023automaticpromptoptimizationgradient}

Expansion first uses gradients to edit the current prompt and then explores the local monte-carlo search space by paraphrasing the editions \cite{pryzant2023automaticpromptoptimizationgradient}

To limit the computation used on evaluating prompts, an approach inspired by best arm identification in bandit optimization is utilized. \cite{pryzant2023automaticpromptoptimizationgradient}


Applying previous iterative prompt optimization methods, based on prompt+score pairs, to text generation tasks is challenging due to the lack of effective optimization signals. \cite{he2024crispomultiaspectcritiquesuggestionguidedautomatic}

Critiques and suggestions, written in natural language, are more helpful for prompt improvement than a single score.\cite{he2024crispomultiaspectcritiquesuggestionguidedautomatic}

CriSPO uses prompt+score+critique triples for next candidate generation. \cite{he2024crispomultiaspectcritiquesuggestionguidedautomatic}
Unlike APE \cite{pryzant2023automaticpromptoptimizationgradient} prompt generation is decoupled from suggestions and a history of critiques and suggestions as packed into the optimizer for a more stable optimization. \cite{he2024crispomultiaspectcritiquesuggestionguidedautomatic}

CoT is applied in optimization by first asking to compare high-score prompts to low-score ones and draft general ideas. \cite{he2024crispomultiaspectcritiquesuggestionguidedautomatic}

Critique-based optimization explores a larger space, which is indicated by lower similarity of the prompts in lexicons and semantics.\cite{he2024crispomultiaspectcritiquesuggestionguidedautomatic}

CriSPO outperforms OPRO \cite{yang2024largelanguagemodelsoptimizers} both on summarization and QA tasks and metaprompt allows for creating ICL and RAG template prompts. \cite{he2024crispomultiaspectcritiquesuggestionguidedautomatic}

\textbf{DSPy optimizers}

Most prompt optimizer approaches do not apply to multi-stage LLM programs where we lack gold labels or evaluation metrics for individual LLM calls. \cite{opsahlong2024optimizinginstructionsdemonstrationsmultistage}

Proposing a few high-quality instructions is essential due to the intractably large search space. \cite{opsahlong2024optimizinginstructionsdemonstrationsmultistage}

Uses a surrogate Bayesian optimization model, which is updated periodically by evaluating the program on batches, to sample instructions and demonstrations for each stage of the LLM program \cite{opsahlong2024optimizinginstructionsdemonstrationsmultistage}

Optimizing demonstrations alone usually yields better performance than just optimizing instructions, but optimizing both yield the best performance. \cite{opsahlong2024optimizinginstructionsdemonstrationsmultistage}

Optimizing instructions is most valuable for tasks with subtle conditional rules not expressible by a few examples.  \cite{opsahlong2024optimizinginstructionsdemonstrationsmultistage}

For LLM programs, it is beneficial to alternate between optimizing weights (fine-tuning) and optimizing prompts. \cite{soylu2024finetuningpromptoptimizationgreat}

\subsubsection{Evolutionary optimization}
Building upon the inherent ability of LLMs to paraphrase (mutation) and combine (crossover) text, an interesting intersection of traditional evolutionary algorithms and modern LLMs has formed. 


Sequences of phrases can be regarded as gene sequences in typical Evolutionary algorithms. \cite{guo2024connectinglargelanguagemodels}


Considers two widely used EAs: Genetic Algorithm and Differential Evolution with DE outperforming GA on most tasks \cite{guo2024connectinglargelanguagemodels}

Initial population consists of manually-written prompts to leverage human knowledge as well as some prompts generated by LLMs to reflect the fact that EAs start from random solutions to avoid local optima. \cite{guo2024connectinglargelanguagemodels}

DE-inspired approached builds on the idea that the common elements of the current best prompts need to be preserved \cite{guo2024connectinglargelanguagemodels}

Evoprompt performs best with roulette selection when compared with tournament and random selection. \cite{guo2024connectinglargelanguagemodels}

Similar results are achieved when population is initialized with the best and with random prompts, hinting that the crafted design of initial prompts is not essential. \cite{guo2024connectinglargelanguagemodels}


Previous research optimized zero-shot instructions and examples separately, overlooking their interplay and resulting in sub-optimal performance. \cite{cui2024phaseevounifiedincontextprompt}

There is a prevailing notion that prompt engineering sacrifices efficiency for performance due to the lengthening of prompts, but PhaseEvo actively shortens the prompts \cite{cui2024phaseevounifiedincontextprompt}

Current EA applications to prompt optimization suffer from extremely high computational cost and slow convergence speed due to the complexity of the high-dimensional search space. \cite{cui2024phaseevounifiedincontextprompt}

PhaseEvo alternates between two phases: exploration with evolution operators and exploitation using a feedback "gradient". \cite{cui2024phaseevounifiedincontextprompt}

TABLE 1 \todo{recreate} compares all 5 operators.  \cite{cui2024phaseevounifiedincontextprompt}

4 phases: initialization - lamarck or manual, local feedback mutation, global evolution with EDA and CR operators, local semantic mutation (paraphrasing) \cite{cui2024phaseevounifiedincontextprompt}

Candidates for evolution operators are selected based on a "performance vector", combining prompts that do not make the same mistakes.  \cite{cui2024phaseevounifiedincontextprompt}

When the performance improvement with an operator stagnates up to some operator-specific tolerance, the current phase is terminated. \cite{cui2024phaseevounifiedincontextprompt}

Evolution in phases outperforms random operator selection. \cite{cui2024phaseevounifiedincontextprompt}

PhaseEvo is the most cost-effective but still needs around 12 iterations and 4000 API calls. \cite{cui2024phaseevounifiedincontextprompt}


APE \cite{zhou2023largelanguagemodelshumanlevel} ran into problems with diminishing returns and abandoning the iterative approach entirely, Promptbreeder aims to solve this with a diversity-maintaining evolutionary algorithm for self-referential self-improvement of prompts \cite{fernando2023promptbreederselfreferentialselfimprovementprompt}

Prompt optimization techniques utilize the fact that LLMs are effective at generating mutations from examples and can encode human notions of interestingness and can be used to quantify novelty. \cite{fernando2023promptbreederselfreferentialselfimprovementprompt}

Self-referential system should improve the way it is improving, thus Promptbreeder used a "hyper-prompt" to optimize its meta-prompt \cite{fernando2023promptbreederselfreferentialselfimprovementprompt}

Uses a binary tournament genetic algorithm. \cite{fernando2023promptbreederselfreferentialselfimprovementprompt}

Uses a random uniformly sampled mutation operators out of 9 total from 5 broad categories for each replication event. \cite{fernando2023promptbreederselfreferentialselfimprovementprompt}

Zero-order mutation (creating a prompt from task description) generates new task prompts more aligned with the task description in the event the evolution diverges.  \cite{fernando2023promptbreederselfreferentialselfimprovementprompt}

LLMs tend to be biased to examples found later in EDA mutation lists. Lying to the LLM and telling it that the prompts are sorted by performance in a descending order improves diversity.  \cite{fernando2023promptbreederselfreferentialselfimprovementprompt}

Removing any self-referential operator in ablation is harmful under nearly all circumstances \cite{fernando2023promptbreederselfreferentialselfimprovementprompt}


\subsubsection{Metaprompting}
Metaprompting or "prompting to create prompts". Research shows that meta-prompting will always be superior to prompting through category theory\cite{dewynter2024metaprompting}.

\chapter{Methodology}
\section{Inference framework}
Taking inspiration from DSPy\cite{khattab2023dspycompilingdeclarativelanguage}, we first implement a simple LLM-calling framework 
capable of invoking several selected inference strategies. Motivations for this are twofold:
\begin{enumerate}
    \item DSPy is a young and ambitious project aiming at simplifying LLM pipeline design and optimization. 
    As we focus on single-stage prompt program optimization, this capability is not useful for our work. 
    Furthermore, due to the framework's infancy, it lacks proper documentation and sometimes exhibits unexpected behavior.
    \item Implementing the prompting techniques discussed in \ref{sec:inference} provides further insight into their workings and performance.
\end{enumerate}

\subsection{Structured generation}
Following current research trends\cite{zhang2025metapromptingaisystems}, we build our inference framework around a structured JSON template,
or a \texttt{Signature}. The \texttt{Signature} structure consists of input and output fields and additional instructions. 
These fields are populated by a \texttt{Field} data structure.
Of particular interest are the output fields, which hold the output name, desired type and optional description. 

Interactions with LLMs in a structured format benefit from better predictability. By implementing the \texttt{Signature} structure,
we can use LLMs as we would a function in any programming language. Functions in programming languages also have functions signatures which
specify input and output names and types.

When employing good naming practices the model can often deduce the task only by looking at output names and types.
Consider the simple \texttt{Signature} in figure \ref{box:simplesig}, which implicitly instructs the LLM to return a word with a meaning opposite to the provided input.

\begin{figurebox}{Simple Signature}{box:simplesig}
    \hlbox{ctuorange}{Word: \texttt{str}}  
    \hlbox{brown}{Antonym: \texttt{str}}
\end{figurebox}

For more complex tasks, filling the output descriptions or even adding explicit instructions is necessary.
In figure \ref{box:complexsig}, notice that it is possible to specify multiple inputs and outputs, which are then generated in the order given.

\begin{figurebox}{Complex Signature. }{box:complexsig}
    \hlbox{ctuorange}{Text: \texttt{str} (Student text)\\
    Grading guide: \texttt{str} (Steps to follow during evaluation)} 
    \hlbox{brown}{Evaluation: \texttt{str} (Textual feedback)\\
     Grade: \texttt{int} (Numerical grade 1-10)} 
    \hlbox{ctublue}{Grade the text.\\ You are an expert text evaluator. \\
    Use the grading guide to evaluate the test and give a final grade. 
    Use formal language and Markdown formatting in the evaluation\\ and output a 1-10 integer for the grade.}
\end{figurebox}

We will maintain this formatting style whenever showing a \texttt{Signature} structure in the future: \textcolor{ctuorange}{orange inputs}, \textcolor{brown}{brown outputs} and the optional \textcolor{ctublue}{blue instructions}.

Sufficiently large instruction-tuned LLMs are usually good at reliably producing JSON output.
For smaller models or more complex output structures, it might be necessary to use some form of constrained generation as discussed in \ref{sec:inference}.
A JSON schema could be constructed automatically from the \texttt{Signature} and passed into a parser-based sampler.
However, this is not necessary for our use-case and the only safeguard we implement is repeated generation in case of a parsing error.

\subsection{Predict method}
To facilitate \texttt{Signature}-powered generation, we implement a \texttt{predict} method that 
involves 1) prepending a developer prompt to the messages, and 2) parsing of \texttt{Signature} outputs.

\begin{figurebox}{Predict method developer prompt with highlighted prompt sections: directive, context, examples and format specifications.}{box:predictdev}
    \hlbox{ctulightblue}{You are an intelligent function that returns structured JSON outputs matching a given schema.
    }
    
    \hlbox{ctulightblue}{
    You will receive a JSON object containing: \\
        - `inputs`: a dictionary of named inputs \\
        - `outputs`: a dictionary specifying the expected output fields with their types and descriptions\\
        - `instructions`: a task or question to answer (optional)\\

    Your job is to:\\
        1. Understand the task from `instructions` or infer it from `inputs` and `outputs`\\
        2. Use the `inputs` to compute or generate the answer\\
        3. Respond **only** with keys from the `outputs` dictionary and values matching the described types
    }
    \hlbox{ctulightblue}{Only return a flat JSON object like:\\
    \{
    "field1": <value matching type and description>,\\
    "field2": <...>
    \}}
    
    \hlbox{ctulightblue}{Do not add metadata, explanations, or wrap outputs in additional structures.\\
    Do not include type names or field descriptions in the output.\\
    Your output must be strictly valid JSON and fill **all** requested output fields.}
\end{figurebox}

The developer prompt has to clearly explain to the LLM how to work with the JSON-based \texttt{Signature}.
In figure \ref{box:predictdev} notice the sections of the prompt following prompt engineering principles outlined in \ref{sec:preng}.

First, the directive states the task, then further context is added about the \texttt{Signature} data structure and the task.
Next, notice the example showing the proper output. Finally, few more clarifying instructions about the output format are added.
In experiments, this prompt is successful in incentivizing parsable outputs adhering to the \texttt{Signature} specifications.

Parsing the output presents some challenges as the LLM sometimes wraps the JSON output into a Markdown code block
or uses inconsistent escape sequences. We implement a simple parser based on regular expressions that is able to parse 
the majority of outputs. In case of a parsing issue or model failure, such as getting stuck in a generation loop, we add a repeated generation
feature.

\subsection{Inference techniques implementation}
Leveraging the \texttt{predict} method and the modular \texttt{Signature}-based interface, we implement a suite of inference-time prompting techniques. 
Each technique is realized through systematic modifications of the \texttt{Signature} fields, changing the developer prompt and the chaining of multiple generation steps 
and function calls. This design allows for modularity and reuse while preserving transparency.
We implement the following methods.
\begin{enumerate}
    \item \textbf{Chain-of-thought}\cite{NEURIPS2022_8bb0d291}: Prepends a reasoning field to the \texttt{Signature} outputs which forms a scratch pad for the LLM.
    \item \textbf{Chain-of-thought with Self-consistency}\cite{wang2023selfconsistencyimproveschainthought}: Multiple CoT generations with majority-voting.
    \item \textbf{ReAct}\cite{yao2023reactsynergizingreasoningacting}: Adding tools allows the LLM to interleave thoughts and action steps.
    \item \textbf{Program-of-thought}\cite{chen2023programthoughtspromptingdisentangling}: Two-stage CoT with Python-code execution
    \item \textbf{Reflexion}\cite{shinn2023reflexionlanguageagentsverbal}: After an initial generation, the model is prompted to self-critique and revise its output.
    \item \textbf{Tree of Thoughts}\cite{yao2023treethoughtsdeliberateproblem}: The problem is first decomposed and each step is expanded, forming a thought tree, which is then traversed with BFS or DFS.
\end{enumerate}

These techniques are however not the main focus of this bachelor's thesis, and we proceed without further discussion or evaluation.
In our prompt optimization method, we will utilize only the basic Chain-of-thought module.

\section{Datasets}
In this section, we discuss choosing datasets for testing our method and comparing various prompt optimization approaches. 
While searching available datasets, we focus on the following criteria:
\begin{enumerate}
    \item \textbf{Output complexity}: We focus on more complex outputs. Specifically, datasets with multiple-choice or Yes/No answers are omitted. 
    This disqualifies commonly used datasets as MMLU or BigBenchHard.
    \item \textbf{Contamination}: Recently, researchers have expressed concern\cite{white2025livebenchchallengingcontaminationlimitedllm} 
    whether benchmarks are reliable evaluations of models as they might appear in their training data. We omit most common datasets, such as GSM8k\cite{cobbe2021gsm8k}, which has been shown to have inflated scores for some models\cite{testing_language_models_on_a_held_out_high_school_national_finals_exam}.
    \item \textbf{Output verification}: We prefer to use simple automatic verification rather than using LLM-as-a-judge, which has been shown to be 
    biased in some circumstances\cite{ye2024justiceprejudicequantifyingbiases}. Neither do we use human feedback, which defeats the purpose of automatic prompt optimization.
    \item \textbf{Difficulty}: We omit tasks where models already have near-perfect score. 
    \item \textbf{Benefit from non-trivial instruction}: We focus on tasks where helpful hints and step-by-step tutorial-like instructions might increase the probability of successful solution.
\end{enumerate}
We now list the datasets that we will use for evaluation and explain why they were chosen.
\subsection{Livebench}
The Livebench\cite{white2025livebenchchallengingcontaminationlimitedllm} dataset is very recent and has been created with the issue of data contamination in mind.
It also addresses the issues of LLM-as-a-judge verification and all its categories can be verified automatically. It is also very challenging, with top models achieving $65\%$ accuracy\cite{white2025livebenchchallengingcontaminationlimitedllm}.

Out of the tasks available in Livebench, we select the \texttt{Connections} task from the \texttt{Language} subset. 
This task consists of sorting given words into non-trivial groups of four based on semantics, phonetics and other features. 
An ideal prompt would attempt to list multiple possible aspects based on which the words can be sorted and also include a helpful example.

\subsection{Code Contests}
Programming puzzles are a difficult and easily verifiable task. Although \texttt{CodeContests}\cite{li2022competition} is an older dataset, 
we anticipate this dataset presents reduced contamination risks compared to datasets with simpler outputs. With LLM-powered coding assistants on the rise, we
feel this is a relevant application area for our method. 

\subsection{Sequences}
We design a small but challenging dataset based on predicting the next number in an integer sequence.
Each sequence is created according to a formula with randomly selected coefficients. The formulas fall into several categories, for example
\begin{itemize}
    \item \textbf{Linear with modulo}: $s(i) = \operatorname{mod}_{q}(a_{1} i + b_{1})$
    \item \textbf{Sum}: $s(i) = \sum_{j=0}^{i-1}a_{1} j + b_{1}$
    \item \textbf{Alternating}: $s(i) = a_{1}i + a_{2}i(-1)^{i}$.
\end{itemize}
This tests the model's ability to 1. detect and understand patterns and 2. systematically perform simple arithmetic. 
In practice, we will optimize just for a single sequence category and observe whether the optimizer evolves a prompt with a tutorial for the specific sequence category.
Experiments showed that the \texttt{Alternating} class of sequences has a good difficulty balance, and we will use it for evaluation.

\section{Evaluation Metrics}
The evaluation metric defines the optimization goal and thus forms its central component. 
Most evaluation metrics are task-specific and divisible into two categories based on whether they are used in a supervised or self-supervised context.
\subsection{Metrics for Supervised Optimization}
Supervised optimization is supported by gold labels and its underlying metrics all perform comparisons between the results and the gold labels.
These include classification metrics like accuracy or Hamming Loss, regression metrics like Mean Squared Error, and many others.

All three main benchmarks that we will use (\texttt{Connections}, \texttt{CodeContests}, \texttt{Sequences}) 
fall into this category. For each benchmark we use a simple accuracy metric. Given a dataset $\mathcal{D}$ and questions $q$ and gold labels $g$, $(q,g) \in \mathcal{D}$:
\begin{enumerate}
    \item \textbf{Connections}:  $\mathcal{F}_{\mathcal{D}_{\text{Conn}}}(q, g) = \operatorname{Overlap}(\operatorname{Groups}(q), \operatorname{Groups}(g))$
    \item \textbf{CodeContests}: $\mathcal{F}_{\mathcal{D}_{\text{Code}}}(q, g) = \operatorname{FinishesExecution}(q) + \operatorname{PassesAllCases}(q, g)$
    \item \textbf{Sequences}: $\mathcal{F}_{\mathcal{D}_{\text{Seq}}}(q, g) = \operatorname{Equals}(q, g)$
\end{enumerate} 

\subsection{Metrics for Self-Supervised Optimization}\label{sec:ssometrics}
In self-supervised contexts, metrics are usually based on reward models pretrained on human preference or environment data.
To allow our method to be applied to gold label-free problems, we turn to LLM-based direct pairwise comparisons.
Given a dataset $\mathcal{D}$ with queries $q$, output $y$ produced by prompt $P \in \mathscr{P}$ and a set of completions $\mathcal{C}_{q}$ for each query.
\begin{equation}
    \mathcal{F}_{\mathcal{D}}^{\text{pairwise}}(q, y, \mathcal{C}_{q}) = \operatorname{WinRate}(\{\operatorname{Compare}(q,y,c)\vsep c\in \mathcal{C}_{q}\}).
\end{equation}
In practice, we combine the output comparison with comparing the output's respective prompts.
These comparisons are then used as optimization signals in the \texttt{Feedback} operator.

\section{Optimization Framework}
Although our first implementation attempt utilized an evolutionary algorithm, we will use a basic population-based hill-climber algorithm.
This design decision has several reasons.
\begin{enumerate}
    \item Most PO research uses a hill-climber architecture.
    \item EAs suffer from slow convergence compared to state-of-the-art hill-climber PO\cite{xiang2025selfsupervisedpromptoptimization}.
    \item PO is complex as it is, and more complicated architectures only introduce more hyperparameters.
\end{enumerate}


\begin{algorithm}
    \caption{Prompt Optimization Hill-Climber}
    \label{alg:promptoptimloop}
    \KwIn{Dataset $\mathcal{D}$, Population size $S$, Iteration count $I$, Batch size $B$}
    \KwOut{Optimized Prompts $\mathscr{P}^{\star}$}
    $\mathcal{D}_{\text{train}}, \mathcal{D}_{\text{dev}}, \mathcal{D}_{\text{test}} \gets \operatorname{Split}(\mathcal{D})$ \tcp{Generate training splits}
    $\mathscr{P} \gets \operatorname{InstructionInduction}(\mathcal{D}_{\text{train}})$ \tcp{Induce initial prompts}
    $i \gets 0$ \tcp{Initialize iteration count}
    $\mathcal{C} \gets \{\}$ \tcp{Initialize solutions} 
    $\mathcal{E} \gets \{\}$ \tcp{Initialize scores}
    $\mathcal{A} \gets \mathscr{P}$ \tcp{All prompts}
    \While{$i<I$}{
        $Q, G \gets \operatorname{RandomSample}(\mathcal{D}_{\text{dev}}, B)$ \\
        $\mathcal{C} \gets \{\mathcal{C}_{q}^{\mathscr{P}}\vsep q \in Q\}$ \\
        $\mathcal{E} \gets \operatorname{Evaluate}(\mathcal{C}, G)$ \\
        $\mathscr{P} \gets \operatorname{Selection}(\mathscr{P}, \mathcal{E})$ \tcp{Pruning} 
        $\mathscr{P} \gets \operatorname{Expand}(\mathscr{P}, \mathcal{C}, \mathcal{E}, \mathcal{D}_{\text{train}})$ \\
        $\mathcal{A} \gets \mathcal{A} \cup \mathscr{P}$ \\
    }
    %$Q_{\text{test}}, G_{\text{test}} \gets D_{\text{test}}$\\
    %$\mathcal{C}_{\text{test}} \gets \{\mathcal{C}_{q}^{\mathcal{A}}\vsep q \in Q_{\text{test}}\}$\\
    %$\mathcal{E}_{\text{test}} \gets \operatorname{Evaluate}(\mathcal{C}, G_{\text{test}})$\\
    $P^{\star} \gets \underset{P\in\mathcal{A}}{\operatorname{argmax}}(\mathcal{E}_{\mathcal{D}_{\text{test}}}(P))$\\
    \Return{$P^{\star}$}
\end{algorithm}

In Algorithm \ref{alg:promptoptimloop} we iterate on the general algorithm \ref{alg:genoptimloop}. 
We will discuss the design of functions used in \ref{alg:promptoptimloop} in following sections.

\begin{itemize}
    \item \textbf{Expand}: The $\operatorname{Expand}$ function can be filled with various expansion operators, of which $\operatorname{InstructionInduction}$
    is a special case. 
    \item \textbf{Evaluate}: Evaluating and identifying the most promising prompts is handled by the $\operatorname{Evaluate}$ operator, which uses task-specific automatic evaluation or LLM-feedback.
    \item \textbf{Selection}: The $\operatorname{Selection}$ operator prunes the population and should maintain only the most promising and diverse prompts for the next expansion.
\end{itemize}

\subsection{Expansion Operator Design}
Expansion operators' job is extending the optimization population with new prompts. Remember notation from \ref{eq:metaprompting}:
\begin{equation*}
    P = \mathscr{M}_{\text{optim}}(M\vsep \mathcal{R}).
\end{equation*}
Notice the use of $\mathscr{M}_{\text{optim}}$, which utilizes non-zero sampling temperature $\tau > 0$ to encourage output diversity. 
Evidently the prompt generation task can be separated into two independent problems: 1. crafting the optimal \textit{Meta-prompt} $M$ 
and 2. designing a data retrieval function $\mathcal{R} = \mathcal{R}(\mathscr{P}, \mathcal{C}, \mathcal{D}, \mathcal{E})$.
The operators' design should address the following challenges:
\begin{enumerate}
    \item \textbf{Loss of generality}: When using task samples $(q, g) \in \mathcal{D}$, the model $\mathscr{M}_{\text{optim}}$ might focus on a single query $q$ and thus fail to generate general instructions.
    \item \textbf{Loss of diversity}: Even for  $\mathscr{M}_{\text{optim}}$ with $\tau>0$, the resulting prompts can be very similar and fail to explore the prompt space $\mathcal{I}$. 
    This ties into a broader exploration vs. exploitation balance issue.
    \item \textbf{Lack of optimization signal}: Research\cite{he2024crispomultiaspectcritiquesuggestionguidedautomatic}\cite{xiang2025selfsupervisedpromptoptimization} suggests that $\mathscr{M}_{\text{optim}}$ 
    can make use of feedback on prompts' outputs and that these textual signals are more effective than numerical scores.
    \item \textbf{Out of distribution \textit{Meta-prompt}}: Prompt engineering is a novel research area and does not have a substantial support in the LLM's training corpus.
    The \textit{Meta-prompt} $M$ thus has to be carefully constructed to help the model output relevant prompts.
\end{enumerate}

We now discuss the design of each prompt generation operator and display their signatures and \textit{meta-prompts}. 
Note that all operators are ultimately used in a CoT context, where a \texttt{reasoning} field is prepended to each signature's outputs.
\subsubsection{Lamarckian}
Instruction Induction\cite{honovich2022instructioninductionexamplesnatural} is used by many PO methods and often referred to as \texttt{Lamarckian Mutation}. 
We will adopt this terminology from now on and design our \texttt{Lamarckian} operator. 
Design of its meta-prompt, shown in figure \ref{box:lamarcksig}, takes into account the design challenges mentioned earlier by 1. warning the LLM to be general and not to focus on a single example, 
2. clearly states the problem using a directive and formatting specifications. 

The problem with diversity still persists, and we consider two approaches to solving it. 
We can increase the model's creativity by increasing its sampling temperature. Another approach is to use
some kind of \textit{seed}, for example a \textit{persona}. We experiment with using personas from PersonaHub\cite{ge2024scalingsyntheticdatacreation}.
Authors of this paper argued that seeding generation with the persona helps with creating novel synthetic data. 

For the data retrieval part, \texttt{Lamarckian} utilizes only examples of the datasets. 
We randomly sample $N$ examples from a separate training split. So
\begin{equation}
    \mathcal{R}_{\text{L}}(\mathcal{D}) = \operatorname{RandomSample}(\mathcal{D}_{\text{train}}, N)
\end{equation}
\begin{figurebox}{Lamarckian Signature. Formatting specifications trimmed.}{box:lamarcksig}
    \hlbox{ctuorange}{Task examples: \texttt{str} (Samples from a problem class) \\
    Persona (Optional): \texttt{str} (Assume this persona when writing the prompt)} 
    \hlbox{brown}{Prompt proposal: \texttt{str} (Instructions for solving the problem)} 
    \hlbox{ctublue}{Craft a \textbf{general} developer prompt to help an LLM with solving a class of problems.\\
    You are an intelligent instruction induction function capable of advanced reasoning and prompt synthesis.\\
    Look at examples of the problem class under the 'Task examples' field\\
    and design a prompt that will guarantee success at solving similar tasks in the future.\\
    Make sure your instructions are \textbf{TRULY GENERAL} and apply to all given samples \textbf{simultaneously}.
}
\end{figurebox}
\newpage
\subsubsection{Iterative}
The \texttt{Iterative} operator is one of the most common and simplest operators. 
It uses a sequence of prompts and their scores in ascending order.
The hope is for the LLM to deduce the optimization direction by looking at the differences in the prompts and incite it to continue the pattern.

Although some research\cite{yang2024largelanguagemodelsoptimizers} only uses the top prompts, we opt for a roulette selection method
and sort to prompts by score in ascending order. The number $N$ is a hyperparameter dictating how many prompts to sample.
We define the retrieval function as
\begin{equation}
   \mathcal{R}_{\text{I}}(\mathscr{P}, \mathcal{E}) = \operatorname{SortByScore}(\operatorname{RouletteSampling}(\mathscr{P}, \mathcal{E}, N), \mathcal{E})
\end{equation}
Other methods\cite{tang2024unleashingpotentiallargelanguage} additionally augment the \textit{meta-prompt} with task samples, similar to the \texttt{Lamarckian} operator.

In the \textit{meta-prompt}, we instruct the LLM to try to follow the sequence. Also, we specifically say to 'craft a new prompt'
as opposed to 'improve a prompt' to incite more novelty. For formatting, we use the same instruction set as in the \texttt{Lamarckian}.
\begin{figurebox}{Iterative Signature. Formatting specifications trimmed.}{box:itersig}
    \hlbox{ctuorange}{Old prompts: \texttt{list} (List of previous prompts with scores)}
    \hlbox{brown}{Prompt proposal: \texttt{str} (Better prompt)} 
    \hlbox{ctublue}{Craft a new prompt for an LLM.
    
    You are an intelligent pattern continuation function capable of advanced reasoning and prompt synthesis.\\
    You are given a history of past prompts along with their scores.\\
    They are listed in ascending order of fitness.\\
    Follow the sequence and design an improved prompt. \\
}
\end{figurebox}

\subsubsection{Reflective}
Recent PO literature\cite{xiang2025selfsupervisedpromptoptimization} shifts to using LLM outputs as optimization signals
and argues that utilizing only numerical signals is ineffective. To address this, we design an exploitative operator, which
aims to fix faults in the prompt by analyzing its failed attempt at a task sample. 

To achieve this, a more complex \texttt{Signature} is utilized. Its outputs guide the LLM to first critique the original prompt
and then improve it. Instructions are more complete with a step-by-step guide which explains the task clearly.
Note that 1. now we use "improve" wording, 2. we stress to only alter the prompt \textit{slightly}. This is done due to 
frequent observation of the model just creating an entirely different prompt only applicable to the single example task.
For formatting, we use the same instructions as in previous \textit{meta-prompts}.

In $\mathcal{R}$, we want to select the worst possible attempt. This means we optimize "from the bottom up" and try to bootstrap the
worst prompts. The retrieval function is
\begin{equation}
    \mathcal{R}_{\text{R}}(\mathscr{P}, \mathcal{C}, \mathcal{D}, \mathcal{E}) = \operatorname{JoinAttemptWithTask}(\operatorname{FindWorstAttempt}(\mathscr{P}, \mathcal{C}, \mathcal{E}, \mathcal{D})
\end{equation}

\begin{figurebox}{Reflective Signature. Formatting specifications trimmed.}{box:reflexsig}
    \hlbox{ctuorange}{Original prompt: \texttt{str} (Improve this prompt) \\
    Task question: \texttt{str} (Task on which the prompt was used) \\
    Solution: \texttt{str} (What the original prompt produced)}
    \hlbox{brown}{Original prompt critique: \texttt{str} (Faults in the original prompt) \\
    Prompt proposal: \texttt{str} (Improved prompt)} 

    \hlbox{ctublue}{Improve a prompt for an LLM.
    
    You are an intelligent reflection function capable of advanced reasoning and prompt synthesis.\\
    Follow these steps to craft a better prompt:\\
    - Analyze the original prompt and its suboptimal performance on a task sample.\\
    - Find failure points in the solution and cross-reference to identify weaknesses in the prompt.\\
    - Think of a critique that captures your findings\\
    - Apply your critique to \textit{slightly} alter the original prompt to improve it.\\
    Your improved prompt should still be \textbf{widely applicable and generic}.}
\end{figurebox}


\subsubsection{Feedback}
As we mentioned earlier, the \texttt{Feedback} operator is suitable for use in self-supervised settings.
It leverages reasoning traces from pairwise LLM-based comparisons, discussed in \ref{sec:ssometrics}. 
Let $\mathcal{E}_{\text{comp}}$ hold textual comparisons of each prompt and their attempts and
$P_{\text{base}} = \operatorname{RandomSample}(\mathscr{P})$. Then
\begin{equation}
    \mathcal{R}_{\text{F}}(\mathscr{P}, \mathcal{E}_{\text{comp}}) = \{P_{\text{base}}, \operatorname{GetComparisons}(P_{\text{base}}, \mathcal{E}_{\text{comp}})\}
\end{equation}
In the \textit{Meta-prompt}, we frame the task as critique synthesis and use "improve" wording to guide the LLM to start from the base prompt.
We also explain that each comparison has a different verdict and the base prompt might not always be the winner. For the formatting guide, we use the same instructions
as in the previous operators.

For large populations or tasks producing long prompts, we might run into issues with LLM context window length. 
However for our purpose, modern LLMs provide more than sufficient context limits. 
\begin{figurebox}{Feedback Signature. Formatting specifications trimmed.}{box:feedbacksig}
    \hlbox{ctuorange}{
        Base prompt: \texttt{str} (Improve this prompt) \\
        Comparisons: \texttt{str} (Base prompt compared to others)
    }
    \hlbox{brown}{Prompt proposal: \texttt{str} (Improved prompt)} 
    \hlbox{ctublue}{Improve a prompt for an LLM.

    You are an intelligent critique synthesis function capable of advanced reasoning. \\
    You are given a base prompt and a list of comparisons between the base prompt and other prompts.\\
    Some other prompts are better than the base prompt, some are worse.\\
    Your task is to analyze the comparisons and synthesize a new prompt that incorporates the feedback.}
\end{figurebox}

\subsubsection{Paraphrase}
To serve as another baseline for other operators, we implement a simple \texttt{Paraphrase} operator.
This operator performs random search in the prompt space by changing the wording and structure of a prompt.
The prompt is selected via the retrieval function
\begin{equation}
    \mathcal{R}_{\text{P}}(\mathscr{P}, \mathcal{E}) = \operatorname{RouletteSampling}(\mathscr{P}, \mathcal{E}).
\end{equation}
This method uses no optimization signal or improvement instructions and relies on pure chance of finding a more potent prompt.


\begin{figurebox}{Paraphrase Signature. Formatting specifications trimmed.}{box:parasig}
    \hlbox{ctuorange}{Original prompt: \texttt{str} (Prompt to paraphrase)}
    \hlbox{brown}{Prompt proposal: \texttt{str} (Paraphrased prompt)} 
    \hlbox{ctublue}{Paraphrase a prompt for an LLM.
                    
    You are an intelligent paraphrasing function capable of advanced reasoning and prompt synthesis.\\
    You are given a prompt and your task is to paraphrase it. \\
    Use synonyms and change the structure of the prompt but keep it semantically equivalent.}
\end{figurebox}
In figure \ref{box:formatting} we show the formatting specification part, which is identical for all prompt generation \textit{meta-prompts}.

\begin{figurebox}{Formatting specifications, which are the same for all operators}{box:formatting}
    Use Markdown formatting in your final answer to indicate bullet points and whatever else necessary.\\
    As a placeholder for the task question, '<INSERT TASK QUESTION HERE>' should be used exactly ONCE.\\
    In the final answer, do not include a title or any additional data, just the prompt.
\end{figurebox}

\subsection{Selection Operator}
At the start of each optimization step, we select $n_{\text{continue}}$ prompts to continue in the process and purge the rest. 
To achieve better prompt diversity, a method based on edit distance is used. 

This method, outlined in Algorithm \ref{alg:duplicpurge},
removes the closest prompt for each prompt, starting from the best prompts. This ensures that performant prompts are kept and their worse-performing duplicates are deleted.
We opt to use edit distance instead of semantic similarity, like BERT embeddings.

\begin{algorithm}
    \caption{Purge Duplicates}
    \label{alg:duplicpurge}
    \KwIn{Population $\mathscr{P}$, Pruning factor $f_{\text{prune}}$}
    \KwOut{Pruned population $\mathscr{P}_{\text{pruned}}$}
    $\mathscr{P}_{\text{sorted}} \gets \operatorname{SortByScore}(\mathscr{P}, \mathcal{E})$ \\
    $n_{\text{continue}} \gets \vert\mathscr{P}\vert(1-f_{\text{prune}})$
    $i \gets 0$
    \While{$i<n_{\text{continue}}$} {
        $P_{\text{select}} \gets \operatorname{GetFirst}(\mathscr{P}_{\text{sorted}})$ \\ 
        $P_{\text{purge}} \gets \underset{P\in\mathscr{P}\mid P \neq P_{\text{select}}}{\operatorname{argmax}} \operatorname{LevenshteinRatio}(P, P_{\text{select}})$ \\
        $\operatorname{Remove}(\mathscr{P}, P_{\text{purge}})$
    }
    $\mathscr{P}_{\text{pruned}} \gets \mathscr{P}$\\
    \Return{$\mathscr{P}$}
\end{algorithm}


\chapter{Experiments}
\section{Comparative analysis of optimization operators}


\appendix

\printindex


\bibliographystyle{ieeetr}
\bibliography{references}

%\ctutemplate{specification.as.chapter}
\end{document}