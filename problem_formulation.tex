\section{Problem Formulation}\label{sec:notation}
Before we describe our approach and discuss implementation in detail, we will first establish mathematical notation and use it to 
formulate the prompt optimization task.

Let $\mathcal{T}$ be the space of character sequences. Then we will define an LLM as a stochastic mapping
\begin{equation}
    \mathscr{M}: \mathcal{T} \rightarrow \mathcal{L}(\mathcal{T}),
\end{equation}
where $\mathcal{L}(\mathcal{T})$ is a probabilistic language distribution learned during the LLM's training.
This distribution is governed by the LLM's hyperparameters $H$, which affect its behavior. 

Of particular interest is the sampling temperature $\tau \in H$ which interpolates between greedy decoding and uniform sampling.
In theory, $\mathscr{M}$ is deterministic for $\tau=0$, but in practice numerical errors still introduce variance.

We use the lower index to specify the purpose of the LLM instance. This also highlights
the fact that each instance $\mathscr{M}$ can use different hyperparameters and underlying models. 
We will use this to differentiate between $\mathscr{M}_{\text{solve}}$ and $\mathscr{M}_{\text{optim}}$.
The optimizer model $\mathscr{M}_{\text{optim}}$ will be utilized to generate and improve prompts which will be evaluated on $\mathscr{M}_{\text{solve}}$.


We consider a prompt $P \in \mathcal{I}$, where $\mathcal{I} \subseteq \mathcal{T}$ is the prompt space.
When such prompt is used as an input into the LLM, it produces a completion
\begin{equation}
    y \sim \mathscr{M}(P)
\end{equation}
and $y \in \mathcal{U}$, where
$\mathcal{U} \subseteq \mathcal{T}$ is the completion space.
In contexts where $P$ serves as a template for an additional query $q$, we will write
\begin{equation}
    y \sim \mathscr{M}(P\vsep q),
\end{equation}
where $P \vsep q$ denotes the result of inserting query $q$ into a designated placeholder in $P$.

Let $\mathscr{P} \subseteq \mathcal{I}$ be a population of prompts. Then for a query $q$ we define the set of completions 
\begin{equation}
    \mathcal{C}_{q}^{\mathscr{P}} = \{\mathscr{M}_{\text{solve}}(P\vsep q)\vsep P \in \mathscr{P}\}.
\end{equation}
For convenience, we might omit the explicit population superscript and write just $ \mathcal{C}_{q}$.

We can use LLMs to solve a general task
\begin{equation}
    t = (q, g) \in \mathcal{D},
\end{equation}
where $\mathcal{D} \subseteq \mathcal{Q} \times \mathcal{G}$ is a dataset of query-answer pairs, $\mathcal{Q}\subseteq\mathcal{T}$ is the set of queries $q$
and $\mathcal{G}\subseteq\mathcal{U}$ is the set of gold labels $g$.

We consider each dataset to have one or more assigned evaluation metrics $\mathscr{F}_{\mathcal{D}}^{\text{supervised}}: \mathcal{U} \times \mathcal{G} \rightarrow \mathbb{R}$,
which scores the LLM output using the corresponding gold label. 
Extending the set of completions for queries from the entire dataset, we define
\begin{equation}
    \mathcal{C}_{\mathcal{D}} = \{\mathcal{C}_{q}\vsep q \in \mathcal{D}\}
\end{equation}

For open-ended tasks, the gold label does not exist, $G = \varnothing$. To achieve effective evaluation even for such tasks, 
we formulate a metric based on pairwise comparisons and define
\begin{equation}
    \mathscr{F}_{\mathcal{D}}^{\text{pairwise}}: \mathcal{Q} \times 2^\mathcal{U} \rightarrow \mathbb{R},
\end{equation}
which maps a query and a set of outputs to a real number. Here we use the power set notation $2^\mathcal{U}$ to signify the set of all possible pairs of completions $y \in \mathcal{U}$.

To generalize, we define $\mathscr{F}_{\mathcal{D}}$ which combines both $\mathscr{F}_{\mathcal{D}}^{\text{pairwise}}$ and $\mathscr{F}_{\mathcal{D}}^{\text{supervised}}$
Using this, we define the mean performance $\mathcal{E}$ of a prompt $P$ on a dataset $\mathcal{D}$ as
\begin{equation}
    \mathcal{E}_{\mathcal{D}}(P) = \frac{1}{\Vert \mathcal{D} \Vert}\underset{t=(q,g)\in \mathcal{D}}{\sum}\mathscr{F}(\mathscr{M}(P\vsep q), t, \mathcal{C}_{q}).
\end{equation}

For convenience, we will define the scores of the population $\mathscr{P}$ on dataset $\mathcal{D}$ as
\begin{equation}
    \mathcal{E}_{\mathcal{D}}(\mathscr{P}) = \{\mathcal{E}_{\mathcal{D}}(P)\vsep P\in \mathscr{P}\} = \{\mathscr{F}(y, g, \mathcal{C}_{\mathcal{D}})\vsep y \in \mathcal{C}_{\mathcal{D}}\}.
\end{equation}

We can then formally define the problem of prompt optimization for a task dataset $\mathcal{D}$ as finding the optimal prompt 
\begin{equation}
    \label{eq:optimdef}
    P^{\star} = \underset{P\in\mathcal{I}}{\operatorname{argmax}}\,\mathbb{E}_{(q, g) \sim \mathcal{D}}\left[\mathscr{F}(\mathscr{M}_{\text{solve}}(P \vsep q), g, \mathcal{C}_{q}^{\mathscr{P}})\right].
\end{equation}

In Algorithm \ref{alg:genoptimloop} we can see the general outline of a population-based optimization method.
The initialization operator $\mathscr{O}_I$ creates an initial population of individuals $\mathscr{P}$. 
In each iteration, a selection operator $\mathscr{O}_S$ first selects a portion of the population according to some criteria. 
These selected individuals are then used by the expansion operator $\mathscr{O}_E$ to create new individuals.
This process continues until a termination condition $\Phi_{stop}$ is reached.

\begin{algorithm}
    \caption{General optimization loop}
    \label{alg:genoptimloop}
    \KwIn{Initialization Operator $\mathscr{O}_I$, Selection Operator $\mathscr{O}_S$, Expansion Operator $\mathscr{O}_E$, Termination Condition $\Phi_{stop}$}
    \KwOut{Optimized Population $\mathscr{P}$}
    \KwData{$\mathscr{P} \gets \mathscr{O}_I$} \tcp{Initialize the population}
    \While{$\neg \Phi_{stop}(\mathscr{P})$}{
        \tcp{Selection and Expansion Steps}
        $\mathscr{P}_{\text{selected}} \gets \mathscr{O}_S(\mathscr{P})$ \\ 
        $\mathscr{P}_{\text{expanded}} \gets \mathscr{O}_E(\mathscr{P}_{\text{selected}})$ \\
        $\mathscr{P} \gets \mathscr{P}_{\text{expanded}}$ \tcp{Update the population} 
        }
        \Return{$\mathscr{P}$} \tcp{Return the optimized population}
    \end{algorithm}
    
We apply this technique to the problem of prompt optimization by defining the aforementioned operators.
Of particular interest are the initialization operator $\mathscr{O}_I$ and the expansion operator $\mathscr{O}_E$, which
both need to produce new prompts. For this purpose, we utilize an LLM instance $\mathscr{M}_{\text{optim}}$ and leverage its 
text generation and reasoning capability.

Let $M\in\mathcal{I}$ be a \textit{Meta-prompt}, or a prompt-generation prompt. We can now formulate generating a prompt 
\begin{equation}
    \label{eq:metaprompting}
    P = \mathscr{M}_{\text{optim}}(M\vsep \mathcal{R}),
\end{equation}
where $\mathcal{R} = \mathcal{R}(\mathscr{P}, \mathcal{C}, \mathcal{D}, \mathcal{E}_{\mathcal{D}}(\mathscr{P}))$ is a retrieval function that selects data
from the current population, past generations, dataset samples and population scores. 
By changing the \textit{Meta-prompt} and the retrieval function $\mathcal{R}$, a variety of possible operators $\mathscr{O}_I$ and $\mathscr{O}_E$
can be defined, thus shaping the prompt optimization process.  
